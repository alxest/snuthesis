\keyword{C, 모듈성, 컴파일러 나눠서 검증하기, CompCert, 분할 컴파일, 다중 언어 링킹}
%%%% TODO: FIXME: keyword 써놓은거 출력이 안됨

\begin{abstractalt}
현대의 소프트웨어 시스템은 매우 복잡하고, 이 복잡성을 길들이기 위해 모듈별로 나눠 개발하는 것은 매우 중요하다.
이런 시스템을 바닥까지 검증하려면, 모듈별로 나눠 검증하는 검증 기법이 필수적이다.
하지만, 기존의 방법들은 만족스럽지 못하다.
본 박사학위논문에서 우리는 나눠서 검증하는 새로운 기법인 RUSC (Refinement
Under Self-related Contexts)를 개발하고, 그 유용함을 번역기 검증과 프로그램 검증을 통해 입증한다.

여러 언어로 쓰여진 프로그램을 검증하기 위한 첨단의 프레임워크인 \ccx{}와 \ccc{}는 서로 다른 방식을 취한다.
전자는 모듈들 사이에 상호재귀가 없고 특정한 좋은 성질을 만족해야 한다는 제약을 둠으로써 문제를 단순화시킨다.
후자는 새로운 검증 기법을 개발하여 제약 조건 없이 문제를 해결하지만, 검증에 너무 많은 노력이 들어간다.

우리는 RUSC가 적은 검증 노력으로도 아무런 제약 없이 문제를 해결할 수 있음을 보인다.
이를 위해, 우리는 \cc{}를 다중 언어 링킹을 지원하도록 확장하였고 이것을 \ccm{}이라고 명명했다.
추가로, 우리는 C와 assembly로 짜여진 프로그램의 수학적 명세를 모듈별로 나눠서 검증함으로써, RUSC가 프로그램 검증에도 유용하게 사용할 수 있음을 보인다.

\end{abstractalt}
