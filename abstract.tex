\keyword{C, Modularity, Compositional Compiler Verification, CompCert, Separate Compilation, Multi-Language Linking}

\begin{abstract}

%% In modern computer systems, to tame complexity, each component is developed modularly and then composed.

  %% In modern computer systems, each component is developed modularly and then compiled down to assembly,
%% by possibly different compilation passes.
%% In order to support these

  %% multi-language >>stronger>> different compilers. the latter can be embedded like: C_gcc/C_llvm
  %% In modern computer systems, each module is compiled down to assembly by possibly different compilers composed of multiple translation passes.

  %% Modern computer systems are often composed of different languages such as C and handwritten assembly, where each language is compiled with different compilers.
  %% Verifying a compiler in the presence of such multi-language linking requires a compositional notion of compiler correctness.
  %% In this dissertation, I propose a 
  %% For a compiler to support multi-language linking,

%% In order to tame complexity, modern computer systems are developed modularly, where each modules are
%% This disseratation presents  {RUSC and CompCertM: A new foundation for modular verification and its application to compiler verification}
%% In this dissertation,
%% Large-scale systems are developed modularly, and its verification should also be done modularly.
%% Modular verification is essential for realistic, large-scale verification.

%% In order to tame complexity, modern computer systems are developed modularly, where each modules are

Modern software systems are large and complex; they are composed of
multiple modules written in different languages such as C and
handwritten assembly.  Moreover, each module is then compiled down to
machine code by multi-pass compilers utilizing complex translations.
To verify such a system, it is critically important to have a modular
verification technique that allows proof engineers to focus on a
single module and its single translation at a time.
%% Then, to run this system, each module is compiled down to machine code by multi-pass compilers utilizing complex translations.
%% composition in both axes -- horizontal and vertical.
%% The former concerns composing verifications of each module, and the latter concerns that of each translations.



The two state-of-the-art frameworks, CompCertX and Compositional
CompCert, supporting modular verification of heterogeneous systems
take different approaches. The former simplifies the problem by
imposing restrictions that the source modules should have no mutual
dependence and be verified against certain well-behaved
specifications. On the other hand, the latter develops a new
verification technique that directly solves the problem but at the
expense of significantly increasing the verification cost.



In this dissertation, we develop a novel lightweight verification technique,
called RUSC (Refinement Under Self-related Contexts), and demonstrate
how RUSC can solve the problem without any restrictions but still with
low verification overhead. For this, we develop CompCertM, a full
extension of the latest version of CompCert supporting multi-language
linking. Moreover, we demonstrate the power of RUSC as a program
verification technique by modularly verifying interesting programs
consisting of C and handwritten assembly against their mathematical
specifications.




%% Supporting multi-language linking such as linking C and handwritten
%% assembly modules in the verified compiler CompCert requires a more
%% compositional verification technique than that used in CompCert just
%% supporting separate compilation.  The two extensions, CompCertX and
%% Compositional CompCert, supporting multi-language linking take
%% different approaches. The former simplifies the problem by imposing
%% restrictions that the source modules should have no mutual dependence
%% and be verified against certain well-behaved specifications. On the
%% other hand, the latter develops a new verification technique that
%% directly solves the problem but at the expense of significantly
%% increasing the verification cost.

%% In this dissertation, we develop a novel lightweight verification technique,
%% called RUSC (Refinement Under Self-related Contexts), and demonstrate
%% how RUSC can solve the problem without any restrictions but still with
%% low verification overhead. For this, we develop CompCertM, a full
%% extension of the latest version of CompCert supporting multi-language
%% linking. Moreover, we demonstrate the power of RUSC as a program
%% verification technique by modularly verifying interesting programs
%% consisting of C and handwritten assembly against their mathematical
%% specifications.





%% This dissertation presents a new theory modular verification, and its application to CompCert compiler.
%% When developing complex software systems, to tame complexity, each component is developed modularly and then composed.
%% When developing complex software systems, modularity plays a crucial role in taming complexity.
%% -- abstraction -- composition
%% In Part I,
%% In Part II, ...
%% In Part III,

\end{abstract}
