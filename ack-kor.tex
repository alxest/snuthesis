\begin{acknowledgement}
  연구실에 입학하기 위해 처음 교수님을 만나뵈었던 때가 바로 어제 같이 생생합니다. 교수님께서는 세상에 도움이 되는 재미있는 문제를 해결해나가자고 하셨었고, 지난 6년간 정말로 그런 시간을 보낼 수 있어서 감사했습니다.

  저의 지도교수님이신 허충길 교수님께 감사드립니다.
  교수님께서는 논문을 읽고, 문제를 찾고, 해결하고, 구현하고, 쓰고, 발표하는 모든 과정에서 저를 이끌어주셨고, 이를 통해 저는 연구란 무엇인지 조금씩 배워나갈 수 있었습니다.
  무엇보다도 교수님의 연구자로써의 자세를 배워나갈 수 있어 감사했습니다.
  \textbf{깊은 이해}란 무엇인지, 또 이를 위해서 얼마나 노력해야 하는지 배울 수 있었습니다.
  복잡하고 거창한 것을 만드는게 아니라, 실제 문제를 가장 단순하게 해결하려는 태도를 배울 수 있었습니다.
  남들이 해놓은 일에 겁먹지 않고, 자신감을 갖고 문제에 도전하는게 얼마나 중요한지 배울 수 있었습니다.
  %% 또 문제의 본질이 무엇인지
  %% 교수님의 지도가 없었더라면 남들이 만들어둔 지식에 잔뜩 겁먹고 그 안에서 길을 잃었을 것 같습니다.
  %% 남들이 만들어놓은 것에 

  이광근 교수님께 감사드립니다.
  교수님께서 한국 프로그래밍 언어 동네의 토양을 도탑게 가꾸어주셔서, 저도 조그마한 싹을 틔울 수 있었습니다.
  이를테면, 교수님께서는 항상 우리말로 쉽고 정확하게 표현하라고 말씀해주셨고, 저는 그 과정에서 각 단어의 뜻을 더욱 견고하게 이해할 수 있었습니다.
  특히 Compiler, interpreter를 번역기, 실행기로 표현하면 된다는 것을 배웠을 때의 충격은 잊을 수 없습니다.
  %% 허충길 교수님과 마찬가지로 ``개뿔 정신''을 강조해주셨고 이것이 얼마나 중요한지 배울 수 있었습니다.

  강지훈 교수님에게 감사드립니다.
  교수님은 제가 아는 가장 훌륭한 엔지니어이고, 짧지 않은 시간 함께 일하며 저도 훌쩍 성장할 수 있었습니다. 
  특히 생산성과 코드 품질 사이의 오묘한 균형을 잡는 방법, 먼저 뼈대를 잡고 살을 채워나가는 효율적인 개발 방법을 배울 수 있었습니다.
  연구실 선배로써 저에게 정말 많은 조언을 해주셨고, 어려울 때 언제든 의지할 수 있는 든든한 버팀목이 되어주셨습니다.

  조민기에게 감사합니다.
  민기는 비상한 이해력과 문제 해결능력, 논리적 모순을 찾아내는 날카로움으로 제 연구에 큰 도움이 되어주었습니다.
  연구 외적으로도 많은 철학과 과학 상식을 배울 수 있었습니다.
  %% 감사합니다.

  소프트웨어 원리 연구실과 프로그래밍 연구실의 동료분들께 감사드립니다.
  이렇게 뛰어난 동료들과, 이렇게 허물없이, 이렇게 즐겁게 시간을 보낼 수 있는 날은 아마도 다시 오기 힘들 것 같습니다.
  연구실 모든 구성원들이 연구실에 크고 작은 일이 있을 때마다 발벗고 나서 주셔서 감사합니다. 특히 강지훈 교수님, 김윤승 형, 이준영, 김용현에게 감사합니다.
  함께 연구할 때 저를 믿고 잘 따라와준 조민기, 김동주, 김용현에게 감사합니다.
  같이 맛집도 다니고 잡담도 하면서 즐거운 시간을 보내게 해준 이준영, 이동권, 김세훈, 이성환, 조민기에게 감사합니다.
  항상 따뜻하게 반겨주시고, 선배로써 많은 조언 해주시는 이우석 교수님, 허기홍 교수님, 김지응 박사님에게 감사드립니다.

  기나긴 대학원 과정을 잘 마무리할 수 있게 도와주신 부모님과 형에게 이 논문을 바칩니다.
  물심양면으로 지원해주신 덕분에 부족함 없이 편안한 마음으로 공부에 임할 수 있었습니다.
  이것이 얼마나 감사할 일인지 잘 알고 있습니다. 갚아나가며 살겠습니다.


  

  %% 긴 박사과정 동안 많은 분들의 도움을 받았습니다.
  %% 세상에 도움이 되는 재미있는 문제를 같이 해결해나가자는 요지의 말씀을 하셨었습니다.
  
\end{acknowledgement}
