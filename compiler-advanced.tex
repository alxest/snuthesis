\section{Advanced Optimizations with Module-Local Invariants}
\label{sec:compiler:advanced}

To demonstrate the flexibility of our framework -- allowing arbitrary memory relations -- we added a new optimization, \texttt{Unreadglob}, whose verification requires a new memory relation.
We flesh out the details in the following order: first we present {\it enriched memory injection}, a mildly strengthened version of \cc{}'s original injection (\Cref{sec:overview-verification:injection:dynamic}),
then the new memory relation (\Cref{sec:overview-verification:injection:static}), and finally \texttt{Unreadglob} optimization (\Cref{sec:compiler:advanced:unreadglob}).
%% In this section, we present {\it enriched memory injection}, a mild generalization of \cc{}'s original injection (\Cref{sec:overview-verification:injection:static}),
%% the new memory relation \Cref{sec:overview-verification:injection:dynamic}, and \texttt{Unreadglob} optimization \Cref{fig:overview-modulelocal:compiler}.

%% In this section we present what our framework additionally offers about compiler and
%% program verification: verifying more advanced compiler optimizations with module-local
%% invariants (\Cref{sec:overview-modulelocal:compiler}) and verifying program modules against their
%% mathematical specification modules (\Cref{sec:overview-modulelocal:program} and \Cref{sec:overview-modulelocal:utod}).

\myparagraph{Enriched Memory Injection}
\label{sec:overview-verification:injection:dynamic}
%
For open modules, reasoning about dynamically allocated local memory
such as a function's stack frame requires to strengthen the original
memory injection due to the presence of unknown modules.  The reason
is because when reasoning about a module $M$, we have to assume that
an unknown function invoked by $M$ does not change the dynamic local
memory of $M$ and also guarantee that a function of $M$ invoked by an
unknown module does not change the caller's dynamic local memory.

For this purpose, \ccc{} introduces \emph{structured injections} that
enrich the original memory injections with ownership information (\ie
whether owned by the current module or others) for all memory blocks
including public ones.  Using them, structured simulations impose
fine-grained invariants subject to the ownership information and a
concrete leakage protocol based on reachability from pointers.

Unlike \ccc{}, \ccm{} generalizes open simulations and memory injections
in a more abstract way following \cite{DBLP:conf/icfp/DreyerNB10,pb}.

First, we generalize the external call case of the open simulation in \Cref{fig:open-sim}
by allowing \emph{private transitions}, denoted $\sqsupseteq_\weak$,
as follows (\textcolor{red}{in red color}):
\[
\begin{array}{@{}l@{}}
\texttt{ 5:}~\quad \textcolor{red}{\exists w' \sqsupseteq_\weak w},~ (f_\src,f_\tgt)\in \texttt{vrel}(\textcolor{red}{w'}) \land (\vec{v}_\src,\vec{v}_\tgt) \in \overrightarrow{\texttt{vrel}(\textcolor{red}{w'})} \land{} \\[1mm]
\texttt{ 6:}~\quad \textcolor{red}{\forall w'' \sqsupseteq w'},~\forall (\memsrc',\memtgt')\in\texttt{mrel}(\textcolor{red}{w''}),~ \forall (r_\src,r_\tgt)\in \texttt{vrel}(\textcolor{red}{w''}),\\[1mm]
\texttt{ 7:}~\quad \textcolor{red}{\exists w''' \sqsupseteq_\weak w'',~ w''' \sqsupseteq w} \land {} \\
\phantom{\texttt{ 7:}}~\quad ((\memsrc',\mathtt{after\_external}~r_\src~\stsrc),(\memtgt',\mathtt{after\_external}~r_\tgt~\sttgt))\in R(\textcolor{red}{w'''})
\end{array}
\]
Though private transitions are allowed before and after an external function call (\ie
$w' \sqsupseteq_\weak w$ and $w''' \sqsupseteq_\weak w''$),
the overall transition should be \emph{public} (\ie $w''' \sqsupseteq w$)
assuming the external call also makes a public transition (\ie $w'' \sqsupseteq w'$).%
\footnote{We only allow private transitions just before and after external calls for simplicity.
See \Cref{sec:related} for comparison with \cite{DBLP:conf/icfp/DreyerNB10,pb}.}

Second, we extend memory injections to specify others' dynamic local
memories in the source and target that should be unchanged by the current module.
Specifically, an (enriched) memory injection $(\iota, m^\weak_\src, m^\weak_\tgt)$
consists of an original memory injection $\iota$ mapping the source public blocks into target blocks; and additionally
a private (\ie dynamic local) memory of the source $m^\weak_\src$ and that of the target $m^\weak_\tgt$
where $m^\weak_\src$ and $m^\weak_\tgt$ should be disjoint from the public memories specified by~$\iota$.
Then, private transitions from $(\iota, m^\weak_\src, m^\weak_\tgt)$ to
$(\iota', {m'}^\weak_\src, {m'}^\weak_\tgt)$ only require that $\iota'$ should extend $\iota$,
while public transitions additionally require that private memories should be unchanged
(\ie $m^\weak_\src = {m'}^\weak_\src$ and $m^\weak_\tgt = {m'}^\weak_\tgt$).
Note that all the areas of the source and target memories that are not on $m^\weak_\src$, $m^\weak_\tgt$ or the injection map $\iota$
are considered as \emph{private} (\ie dynamic local) memory of the current module.

%% \begin{wrapfigure}{r}{0.45\textwidth}
\begin{minipage}{1\textwidth}
\mbox{}\\[-7mm]    
\begin{Verbatim}
   int f() {          int f() {     
1:   int a0;            int a[2];   
2:   reg a1 = 0;  -->   a[1] = 0;   
3:   g(&a0);            g(&a[0]);   
4:   return a1;         return a[1];
   }                  }
\end{Verbatim}
\mbox{}\\[-10mm]
\end{minipage}
%% \end{wrapfigure}
To show how it works,
we give an example mimicking register spilling
in the presence of address-taken stack variables.
Consider the transformation on the right, where
in the source a memory block for \texttt{a0} and a function-local register for \texttt{a1} are allocated and
the address of \texttt{a0} escapes to \texttt{g},
while in the target a single block for both \texttt{a[0]} and \texttt{a[1]}
is allocated and the address of the block escapes to \texttt{g}.
Here \texttt{a0} can be seen as an address-taken stack variable and \texttt{a1} a spilled register.
The key difference is that, in the source, \texttt{a1} cannot be accessed by
\texttt{g} since it is a function-local register
while, in the target, \texttt{a[1]} can be accessed via the address of \texttt{a[0]}.

We now show how the target \texttt{f} simulates the source \texttt{f}
by logically protecting \texttt{a[1]} from \texttt{g}.
Though we give an informal description here to help understanding,
the formal definition of an open simulation $R$ 
can be easily derived from the description.
At line~$\texttt{1}$, any world $w_0$ and
memories $(m_\src, m_\tgt)$ related at $w_0$ are given. We take a step
to line~$\texttt{2}$ by extending $w_0.\iota$ (\ie the public
injection of $w_0$) to map $\texttt{a0}$ to $\texttt{a[0]}$, say $w_1$,
which is a public transition. At line~$\texttt{2}$, we take a step
to line~$\texttt{3}$ without changing the world $w_1$.
At line~$\texttt{3}$, we first make a private transition from $w_1$
to $w_2$ by extending $w_1.m^\weak_\tgt$
%(\ie the private area of the target memory)
to include the memory chunk $\texttt{a[1]} = 0$.
Then we assume that \texttt{g} makes a public transition from $w_2$ to $w_3$
returning any memories related at $w_3$. Thanks to $w_2.m^\weak_\tgt = w_3.m^\weak_\tgt$,
we know that the chunk $\texttt{a[1]} = 0$ remains the same.
Then we make a private transition from $w_3$ to $w_4$ by
dropping the chunk $\texttt{a[1]} = 0$ from $w_3.m^\weak_\tgt$.
Since $w_4.m^\weak_\tgt = w_1.m^\weak_\tgt$, we have a public transition from $w_1$ to $w_4$.
Finally, at line~$\texttt{4}$, we know that both the register $\texttt{a1}$ and
the memory-allocated variable $\texttt{a[1]}$ contain
$\texttt{0}$ and thus the same value $\texttt{0}$ is returned.

It is important to note that the (others') private memories $w.m^\weak_\src$ and $w.m^\weak_\tgt$ of a
memory injection $w$ are preserved as long as a function accesses
$(i)$ the memory via public addresses, or $(ii)$ its own private memory.
In the former case,
since a public block of the source is fully injected into a block of the target,
%% ---this is why the mapping is called an injection---
whenever a pointer offset goes beyond the public area mapped by the injection $w.\iota$,
the source program accesses an unallocated area thereby raising UB.
In the example above, if \texttt{g} in the target accesses \texttt{*(\&a[0]+1)},
then in the source it accesses \texttt{*(\&a0+1)}, which raises UB.
In the latter case, since the function's own private memory
is disjoint from all the memories specified by~$w$,
accessing it does not affect $w$. In the example above, at line~\texttt{2} in the target, 
the assignment \texttt{a[1] = 0} preserves $w_1.m^\weak_\tgt$ (and also the target public memory of $w_1$) because we know that
the current private memory \texttt{a[1]} is disjoint from the area specified by $w_1$ by construction.

\newrevision{Also note that any part of the public memories cannot be
  converted to a private one since the injection map is only
  extended at each step; and any part of the others' private memories
  (\ie $m^\weak_\src$ and $m^\weak_\tgt$) cannot be
  converted to the current module's private one since all
  \emph{proper} steps (\ie local steps or steps across an external
  call) only allow public transitions (\ie preserving $m^\weak_\src$ and $m^\weak_\tgt$).}

%% $R(w)$ relates any memories related at $w$ and
%% we take a step to line~$\texttt{2}$ by extending $w.\iota$
%% (\ie the (public) injection of $w$)
%% to map $\texttt{a0}$ to $\texttt{a[0]}$. At line~$\texttt{2}$,
%% $R(w)$ requires that $w.\iota$ maps $\texttt{a0}$ to $\texttt{a[0]}$

%% We now show how to logically protect \texttt{a[1]} from \texttt{g} and
%% prove that the two programs are related by an open simulation $R$.
%% First, $R$ relates each corresponding line of \texttt{f} in the source
%% and target.  We will then explain, at each line, how $R$ relates
%% source and target memories and satisfies the open simulation property.
%% At line~$\texttt{1}$, $R(w)$ relates any memories related at $w$ and
%% we take a step to line~$\texttt{2}$ by extending $w.\iota$
%% (\ie the (public) injection of $w$)
%% to map $\texttt{a0}$ to $\texttt{a[0]}$. At line~$\texttt{2}$,
%% $R(w)$ requires that $w.\iota$ maps $\texttt{a0}$ to $\texttt{a[0]}$

%% First,
%% we define an open simulation $R(w)$ to relate any memories related at
%% $w$ and each corresponding line of \texttt{f} in the source and
%% target. Second, we prove that 

%% Explain open simulation

\myparagraph{Memory Injection with Module-Local Invariants}
\label{sec:overview-verification:injection:static}
%
For open modules, reasoning about statically allocated local memory
such as static variables of C requires a further generalization.  The
problem is that when an open module $M$ invokes an unknown function
$f$, one cannot assume that the static memory of $M$ is unchanged
during the call because $f$ may call back a function from $M$, which
may change the static memory. However, since the static memory is only
accessible to the known functions in $M$, one can find a certain
invariant on the static memory by analyzing all the functions of $M$
and expect that an external call preserves the invariant although the
static memory can be changed. Enabling such reasoning is simple:
\ccm{} just adds another component in a memory injection $w$ that
globally imposes a given invariant on selected static variables
disjoint from $w.m^\weak_\src$, $w.m^\weak_\tgt$ and $w.\iota$.
We give examples using module-local invariants in \Cref{sec:overview-modulelocal}.


\begin{figure}[t]
\begin{Verbatim}
static int x = 0;       static int x = 0;                   
int f() {               int f() {              int f() {    
  g();           [CP]     g();          [UG]     g();       
  x = 1;        ----->    x = 1;       ----->               
  return x;               return 1;              return 1;  
}                       }                      }            
\end{Verbatim}
\begin{Verbatim}
static int y = 0;
void g() {
  if (y == 0) {
    y = 1; f();
  }
}                 
\end{Verbatim}
\caption{An example of \texttt{Unreadglob} optimization}
\label{fig:overview-modulelocal:compiler}
\end{figure}

\myparagraph{Unreadglob Optimization}
\label{sec:compiler:advanced:unreadglob}
We developed a new optimization \texttt{Unreadglob} eliminating all
unread static variables and instructions writing to them.
\Cref{fig:overview-modulelocal:compiler} shows an example
optimization, where $(i)$ the first program is optimized to the second
one by constant propagation~(CP) replacing \texttt{return x} by
\texttt{return 1}; and $(ii)$ the second one is optimized to the third
one by \texttt{Unreadglob}~(UG) eliminating the unread static variable
\texttt{x} and the command \texttt{x = 1}.  It is important to note
that across the function call \texttt{g()}, the static variable
\texttt{x} may be updated from \texttt{0} to \texttt{1} because the
function \texttt{g} can indirectly update it by calling \texttt{f} as shown in
the fourth program in \Cref{fig:overview-modulelocal:compiler}.

\revision{In verification of the optimization \texttt{UG} above,
we have to use memory injections $w$ with module-local invariants
introduced in \Cref{sec:overview-verification:injection:static}.
The reason is that the static variable \texttt{x} in the source cannot reside
$(i)$ in the injection map $w.\iota$ since \texttt{x} does not exist in the target; or
$(ii)$ in the source private $w.m^\weak_\src$ since \texttt{x} can be modified
during the external call \texttt{g()}.
To verify \texttt{UG} above, we can impose the trivial invariant
\texttt{Top} on the eliminated static variable \texttt{x}, meaning
that \texttt{x} can be modified arbitrarily, which is sufficient
because \texttt{x} is unread.}

Note that \ccx{} may be
able to verify \texttt{Unreadglob} using memory injections because it
assumes no mutual dependency among modules, so that no static
variables can be accessed via external function calls, unlike the above
example with mutual recursion.
