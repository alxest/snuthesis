\section{CompCertM}
\label{sec:compiler:compcertm}

Based on the theories we presented so far, we develop \ccm{}, an extension of \cc{} with the
repaired interaction semantics and open simulations to support multi-language linking.  We state
\ccm{}'s compositional correctness results (\Cref{sec:results:compiler}) and evaluate its
verification efforts (\Cref{sec:results:evaluation}).  \ccm{} currently supports the x86 backend only.
We do not currently see any technical problem with supporting other architectures.

\subsection{Compositional Correctness}
\label{sec:results:compiler}

\ccm{} uses open simulations with three parameters:
memory relations, symbol relations and memory predicates
(see \Cref{sec:main-verification:opensim} for details).
It supports $(i)$ the memory relations discussed in \Cref{sec:overview-verification:injection}:
identity, extension and (enriched) injections with no or any given module-local invariant;
$(ii)$ two symbol relations: one for keeping identical symbols in the source and target
and the other for allowing elimination of global variables in the target (only allowed for memory injections), needed for \code{Unusedglob} and \code{Unreadglob};
$(iii)$ two memory predicates: one for no analysis and the other for the value analysis of \cc{}.

Let $\rels$ be the set of open simulations with all possible parameters.
To apply RUSC, we prove that the \ccm{} compiler $\mathcal{C}$ transforms the source module with
a series of passes that are independently verified using open simulations in $\rels$.
\begin{lemma}[Pass Correctness]\label{thm:results-passes}
  For any \textrm{Clight} module $S$ and \textrm{Asm} module $T$, if $\mathcal{C}(S) = T$, then
  there exist intermediate modules $M_0, M_1, \cdots, M_n$ such that:
  \begin{enumerate}
  \item $M_0 = S$ and $M_n = T$; and
  \item $\forall i \in [0,n),~ \exists R \in \rels,~ (M_i, M_{i+1}) \in R$~.
  \end{enumerate}
\end{lemma}

We also prove all \textrm{Clight} and \textrm{Asm} modules are self-related.
\begin{lemma}[Self-Relatedness]\label{thm:results-relatedness} For any \textrm{Clight} or \textrm{Asm}
  module $M$, we have $M \in \self{\rels}$.
\end{lemma}
\noindent
\revision{Note that
  since we define illegal interference from Asm
  (\ie causing different behaviors in the source and target) as undefined behaviors (UBs)
  as shown in \Cref{sec:compiler:solution},
  every Asm module can be self-related.}

From \Cref{thm:results-passes,thm:results-relatedness}, the RUSC relation for the compiler follows.
\begin{theorem} [Modular Correctness]\label{thm:results-modular}
  For any \textrm{Clight} module $S$ and \textrm{Asm} module $T$, if $\mathcal{C}(S) = T$:
  \[
    {S} \rusc_\rels {T} \quad\text{with}\quad S,T \in \self{\rels}~.
  \]
\end{theorem}

\noindent
This theorem provides a truly compositional correctness
thanks to the compositionality of RUSC (\Cref{thm:rusc}):
%% The fact that the source and target are related by RUSC
%% implies that they satisfy behavioral refinement by adequacy of RUSC, and moreover
the relation can be freely (\ie vertically or horizontally) composed with any verification using RUSC
including that against mathematical specifications.
As an example, the following compositional correctness follows.
\begin{corollary} [Compositional Correctness 1]\label{thm:results-compiler}
  Let $(S_1,T_1), \ldots, (S_n,T_n)$ be pairs of source and target modules.
  If each pair is either compiled (\ie $\mathcal{C}(S_i) = T_i$ with $S_i$ \textrm{Clight} and $T_i$ \textrm{Asm}), or a self-related context (\ie $S_i = T_i \in \self{\rels}$), then
  \[
    \beh{S_1 \llink \cdots \llink S_n} \supseteq \beh{T_1 \llink \cdots \llink T_n}~.
  \]
\end{corollary}
%
\noindent This correctness theorem is compositional in the sense that behavior is refined in the
presence of any self-related contexts such as arbitrary \textrm{Clight} and \textrm{Asm} modules
(\Cref{thm:results-relatedness}).





Note that \textrm{Clight}, not \textrm{\cc{} C}, is the source language in the above theorems.  One of the
reasons is that \textrm{Clight} is the source language for most verification frameworks based on
\cc{}, such as VST~\cite{VST}, \ccc{}, and \ccx{}.  More importantly, we found that
\textrm{\cc{} C} is incompatible with memory injections.  Specifically,
\textrm{\cc{} C} imposes a strict alignment requirement on memory blocks of size zero, which, however,
%% but the requirement
is not preserved by memory injections.
%% For this reason, we cannot achieve full horizontal compositionality
%% in the presence of both \textrm{\cc{} C} modules and compiler passes verified using memory injections.
In other words, \textrm{\cc{} C} modules are not always self-related by memory injections.\footnote{This problem
  would be solved if one strengthens memory injections with more strict alignment requirements.}

\myparagraph{Supporting \textrm{\cc{} C}}
However, we can still prove a compositional correctness (not modular correctness as in \Cref{thm:results-modular}) for \textrm{\cc{} C}
following \scc{}'s \emph{Level A} technique~\cite{kang:scc},
which exploits the fact that all \textrm{CompCert C} modules are transformed to \textrm{Clight} modules
by the same two passes.
Specifically, the first pass is verified using an open simulation with the memory identity
and the second pass with memory injections, as done in the original \cc{}.
Then the following lemma follows from horizontal compositionality and adequacy of
open simulations (with memory identity and injection) and transitivity of behavioral refinement.

\begin{lemma} [ClightGen Correctness]\label{thm:results-clightgen}
  Let $(S_1,T_1), \ldots, (S_n,T_n)$ be pairs of source and target modules.
  If each pair is either translated (\ie $\textrm{ClightGen}(S_i) = T_i$ with $S_i$ \textrm{\cc{} C} and $T_i$ \textrm{Clight}), or a self-related context (\ie $S_i = T_i \in \self{\rels}$), then
  \[
    \beh{S_1 \llink \cdots \llink S_n} \supseteq \beh{T_1 \llink \cdots \llink T_n}~.
  \]
\end{lemma}



By composing \Cref{thm:results-compiler}, \Cref{thm:results-clightgen} and \Cref{thm:results-relatedness}, we have the following theorem.
\begin{theorem} [Compositional Correctness 2]\label{thm:results-compiler2}
  Let $(S_1,T_1), \ldots, (S_n,T_n)$ be pairs of source and target modules.
  If each pair is either compiled (\ie $\mathcal{C}(S_i) = T_i$ with $S_i$ \textrm{\cc{} C} or \textrm{Clight} and $T_i$ \textrm{Asm}), or a self-related context (\ie $S_i = T_i \in \self{\rels}$), then
  \[
    \beh{S_1 \llink \cdots \llink S_n} \supseteq \beh{T_1 \llink \cdots \llink T_n}~.
  \]
\end{theorem}


\myparagraph{Adequacy w.r.t. Physical Semantics}


We show that the repaired interaction semantics is adequate w.r.t. the physical semantics of \cc{},
where the former uses the language-independent linking $\llink$ and the latter the syntactic linking $\plink$
concatenating modules of the same language.

We prove that the physical semantics refines the repaired interaction semantics for \textrm{Asm} modules
using a closed simulation of \cc{} with memory injections.
\begin{theorem}[Adequacy w.r.t. Assembly]\label{thm:results-adequacy-asm}
  Let $M_1, \cdots, M_n$ be \textrm{Asm} modules.  We have:
  \[   \beh{M_1 \llink \ldots \llink M_n} \supseteq  \beh{M_1 \plink \ldots \plink M_n} ~.\]
\end{theorem}
\noindent
\revision{This theorem allows us to carry verification results on the interaction semantics such as \Cref{thm:results-compiler2}
down to \cc{}'s Asm semantics with syntactic linking.}



Conversely, we prove that the repaired interaction semantics refines the physical semantics for \textrm{\cc{} C} modules
using a closed simulation of \cc{} with memory identity.
This result is useful because we want to allow separate compilation (of C modules) on the compiler side, and on the program verification side, we want to hide complexities from inter-module steps.
\begin{theorem}[Adequacy w.r.t C]\label{thm:results-adequacy-c}
  Let $M_1, \cdots, M_n$ be \textrm{\cc{} C} modules.  We have:
  \[  \beh{M_1 \plink \ldots \plink M_n} \supseteq \beh{M_1 \llink \ldots \llink M_n}  ~.\]
\end{theorem}


In some sense, the \Cref{thm:results-compiler2,thm:results-adequacy-asm,thm:results-adequacy-c} together forms a strong stress-test for a language-independent linking, and our results show strong evidence that our repaired interaction semantics is indeed adequate (in a literal sense).
Specifically, if one of the three desiderata is missing, it is trivial to find language-independent linking satisfying the others.
Without \Cref{thm:results-adequacy-asm}, one can define interaction semantics to always execute UB; then, the other theorems become trivial.
Without \Cref{thm:results-adequacy-c}, one can define the behavior of interaction semantics to an empty set.
Without \Cref{thm:results-compiler2}, one can define $\llink \defeq \plink$.

%% These results mean that the repaired interaction semantics does not give too few behaviors to assembly programs (e.g., missing physically observable behaviors), nor does it give too many behaviors to well-typed C programs (e.g., giving UB to them).


Interestingly, by composing \Cref{thm:results-compiler2,thm:results-adequacy-asm,thm:results-adequacy-c}, we obtain
the same separate compilation correctness result of \scc{}~\cite{kang:scc}:

\begin{corollary}[Separate Compilation Correctness]
  Let $S_1, \ldots, S_n$ be \textrm{\cc{} C} modules and $T_1, \ldots, T_n$ be \textrm{Asm} modules.
  If $\mathcal{C}(S_i) = T_i$ for each $i$, we have:
  \[
    \beh{S_1 \plink \cdots \plink S_n} \supseteq \beh{T_1 \plink \cdots \plink T_n}~.
  \]
\end{corollary}

\youngju{Just mention that \ccc{} does not satisfy upperbound, and explain it in appendix?}
\youngju{here? or appendix?: To this end, we have strengthened \cc{}'s type checker in a number of
  ways, ruling out trivially wrong (according to C standard) programs more than before.  We rule out
  (i) a program that contains an identifier that is not declared in the module (ii) ``return''
  (without value) statement used for non-void function (iii) ``return'' (with value) statement used
  for void function (iv) function arguments containing void type.  (v) has duplicate (function or
  global variable) identifiers (vi) A function argument with size bigger than INT\_MAX (\cc{}
  already aborts on such programs)}





\subsection{Evaluation of Verification Efforts}\label{sec:results:evaluation}

\begin{table}[t]
\footnotesize

\parbox{\linewidth}{
\caption{SLOC of \ccm{} and related works --- compared to its baseline \cc{}, respectively}
\begin{tabu}{@{}l@{\hspace{1.55pt}}|[1.25pt]@{\hspace{1.55pt}} c @{\hspace{1.55pt}}|@{\hspace{1.55pt}} c @{\hspace{1.55pt}}|@{\hspace{1.55pt}} c @{\hspace{1.55pt}}|[1.25pt]@{\hspace{1.55pt}} c @{\hspace{1.55pt}}|@{\hspace{1.55pt}} c @{\hspace{1.55pt}}|[1.25pt]@{\hspace{1.55pt}}}
Portion     & \shortstack{\cc{} \\ 3.5} & \ccr{} 3.5        & \ccm{} pack                                               & \shortstack{\cc{}\\ 2.1} & \ccc{} \\
\hline
Pass Proofs & 34,376    & 35,893 (+4.41\%)  & \newrevision{4,923(+14.32\%)}                                          & 21,215    & 52,140 (+145.77\%) \\
The Rest    & 85,617    & 87,965 (+2.74\%)  & \newrevision{25,558(+29.85\%)}  & 59,365    & 107,910 \hspace{.6mm} (+81.77\%) \\
Total       & 119,993   & 123,858 (+3.22\%) & \newrevision{30,481(+25.40\%)}                                         & 80,580    & 160,050 \hspace{.6mm} (+98.62\%) \\
\end{tabu}
\\
\begin{tabu}{@{}l@{\hspace{1.55pt}}|[1.25pt]@{\hspace{1.55pt}} c @{\hspace{1.55pt}}|@{\hspace{1.55pt}} c @{\hspace{1.55pt}}|[1.25pt]@{\hspace{1.55pt}}}
Portion     & \shortstack{\cc{} \\ 3.0} & \ccx{}             \\
\hline
Pass Proofs & 26,466    & 30,572 (+15.51\%)  \\
The Rest    & 82,312    & 121,532 (+47.65\%) \\
Total       & 108,778   & 152,104 (+39.83\%) \\
\end{tabu}
\label{table:evaluation-ours}
}

%% \parbox{\linewidth}{
%% \label{table:evaluation-ours}
%% }
    
\parbox{0.38\linewidth}{
\vspace{4mm}
\caption{\mbox{Breakdown of \ccm{} pack}}
\begin{tabu}{@{}l | l@{}}
Portion                          & SLOC                                                                                                     \\
\hline
\revision{Proofs about Intermodule Steps} & \newrevision{4,923}                                                                                                    \\
Interaction Semantics/Properties & 1,940                                                                                                    \\
Language Semantics/Properties    & 1,701                                                                                                    \\
Self Simulations                 & \newrevision{5,593}                                                                                                    \\
\cc{}  Metatheory Extension      & \newrevision{4,688}                                                                                                    \\
\ccm{} Metatheory                & \newrevision{7,656}                                                                                                    \\
Mixed Simulation                 & 1,090                                                                                                    \\
Adequacy w.r.t. Asm              & 2,890                                                                                                    \\
\end{tabu}
\label{table:evaluation-breakdown}
}
\hfill
\parbox{0.45\linewidth}{
\vspace{4mm}
\caption{SLOC of additional developments}
%% \begin{tabu}{@{}l @{\;} |[1.25pt] @{\;} r @{\;} | @{\;} r @{\;} | @{\;} r @{\;} | @{\;} r @{\;} | @{\;} r @{}}
%% Portion                          & \shortstack{\texttt{Unreadglob} \\ 3.5} & \shortstack{\texttt{Unreadglob} \\ pack} & \texttt{mutual-sum} & \texttt{utod} & \shortstack{Adequacy \\ w.r.t. C} \\
%% \hline
%% Pass Proofs                      & 1,842                   & 338                      & 3,088               & 361           & -             \\
%% The Rest                         & 260                     & 1,933                    & 2,707               & 424           & 4,044         \\
%% Total                            & 2,102                   & 2,271                    & 5,795               & 785           & 4,044         \\
%% \end{tabu}
\begin{tabu}{@{}l @{\;} |[1.25pt] @{\;} r @{\;} | @{\;} r @{\;} | @{\;} r @{\;} | @{\;} r @{\;} | @{\;} r @{}}
Portion                          & \shortstack{\texttt{Unreadglob} \\ 3.5} & \shortstack{\texttt{Unreadglob} \\ pack} & \shortstack{Adequacy \\ w.r.t. C} \\
\hline
Pass Proofs                      & 1,842                   & 338                      & -             \\
The Rest                         & 260                     & 1,933                    & 4,044         \\
Total                            & 2,102                   & 2,271                    & 4,044         \\
\end{tabu}
\label{table:evaluation-others}
}%
\end{table}

\youngju{There are two ``unreadglob'' columns, one for \cc{} and one for pack. Simplify it}
\youngju{How about reducing caption text size?}
\jeehoon{``Per-pass'', ``Metatheory'', and ``Total'' instead of ``Pass Proofs'', ``The Rest'', and ``Whole''}

To demonstrate that \ccm{} is lightweight,
we compare significant lines of code (SLOC) of \ccm{}, \ccc{}, and \ccx{} with
those of their baseline \cc{} versions 3.5, 2.1, and 3.0, respectively.
Overall, \ccm{} adds less code to \cc{} than \ccc{} and \ccx{} do,
and in particular significantly less code than \ccc{} for the proofs of compiler passes.%
\footnote{\revision{Note that \ccc{} allows horizontal compositionality between any intermediate languages (ILs)
  while \ccm{} only between Clight and Asm since self-relatedness is proven only for the two.
  Though practically unnecessary, supporting linking between arbitrary ILs in \ccm{} would increase SLOC to prove self-relatedness for the other ILs.}}
\newrevision{Also note that \ccr{} uses the enriched memory injections of \Cref{sec:overview-verification:injection:dynamic} instead of the original memory injections
in order to give reusable main lemmas for both closed and open simulations.
Since \ccr{}'s pass proofs are only 4.41\% larger than \cc{}'s, 
the overhead due to handling the private memory components of enriched memory injections is, roughly speaking, at most 4.41\%.}

\Cref{table:evaluation-ours} summarizes the comparison.
For each compiler (\ie each column),
the rows report SLOC for the proofs of all compiler passes (Pass Proofs),
the rest of the development (The Rest),
and their summation (Total).
Note that \ccm{} is split into \ccr{} and \ccm{} pack, for which the former is our refactoring
of \cc{} and the latter is an additional package to support multi-language linking.
We counted SLOC reported by
\code{coqwc}.\footnote{Concretely, we counted ``spec'' and ``proof'' lines reported by \code{coqwc}.
  Because we use a different criteria for line numbers, they are different from those reported in
  prior work~\cite{stewart:ccc,gu:dscal,wang:saccx}.}  When counting SLOC, we excluded the following
code for fair comparison: $(i)$ code for other architectures than x86 because all three projects support
only x86; $(ii)$ code for the parser and type checker introduced in later versions of \cc{}; and $(iii)$ code for \textrm{ClightGen}, which is not supported by both \ccx{} and
\ccc{}.  We also excluded \ccc{}'s legacy proofs for the original compiler correctness.  We used the
latest development branches for the three projects.\footnote{Development as of November 8, 2019, available at: \url{https://github.com/snu-sf/compcertr}, \url{https://github.com/snu-sf/compcertm}, \url{https://github.com/PrincetonUniversity/compcomp}, \url{https://github.com/DeepSpec/dsss17/tree/master/CAL}}








\Cref{table:evaluation-breakdown} analyzes the \newrevision{30,481} SLOC for \ccm{} pack.
\revision{The pass proofs consist of \newrevision{4,923} SLOC for reasoning about intermodule steps, which is
  sometimes nontrivial since they perform the logical instrumentation presented in \Cref{sec:overview-semantics}.
  Note that \ccr{} provides proofs for intramodule steps as main lemmas, which are reused in \ccm{}.
}
The rest consists of
1,940 SLOC for the repaired interaction semantics and its properties;
1,701 SLOC for properties of each language such as determinism and receptiveness;
5,576 SLOC for self-relatedness (\Cref{thm:results-relatedness});
4,687 SLOC for extending the metatheory of CompCert;
7,569 SLOC for open simulations and other metatheory for \ccm{};
1,090 SLOC for mixed simulation; and
2,890 SLOC for adequacy w.r.t. assembly (\Cref{thm:results-adequacy-asm}).

%% \Cref{table:evaluation-others} shows SLOC for the new optimization pass and the verification examples
%% given in the dissertation.  Note that \code{Unreadglob} 3.5 adds the optimization to \ccr{} proving closed simulation
%% and \code{Unreadglob} pack to \ccm{} proving open simulation, which reuses the proof of \code{Unreadglob} 3.5 for intramodule steps.
%% As the verification of \code{mutual-sum} and \code{utod} show, directly proving
%% open simulation between programs and specifications is costly. 
%% We believe that program logics like VST~\cite{VST} can be used to prove such simulation,
%% which could significantly reduce the verification cost.
\Cref{table:evaluation-others} shows SLOC for the new optimization pass.  Note that \code{Unreadglob} 3.5 adds the optimization to \ccr{} proving closed simulation
and \code{Unreadglob} pack to \ccm{} proving open simulation, which reuses the proof of \code{Unreadglob} 3.5 for intramodule steps.
