\section{CompCertR}
\label{sec:compiler:compcertr}

In this section, we list the major differences between \cc{} and \ccr{}.

As discussed, we enriched \cc{}'s existing memory injection so that it can be used in an open simulation.
Roughly speaking, \cc{} only \emph{relies} on private memory protection (by system call axioms) while it does not \emph{guarantee} it.
This is indeed okay because system calls are expected not to make a mutually recursive call and well-behaved.
However, in our open setting, an external call can make a mutually recursive call, and each translation should also guarantee the condition.
For instance, suppose that \code{f} calls \code{g} and \code{g} calls \code{f} again.
Here, for the first f to rely on private memory protection, second f needs to guarantee it.

For technical reasons, we also refactored some language's semantics but \cc{}'s correctness results remain the same.
Specifically, we changed the followings:
\begin{itemize}
\item We changed the \code{Callstate} of each language to carry function pointer instead of identifiers.
  Note that in the open setting, a module can call another module's function via a function pointer.
  %% There are two ways to support it.
  To be specific, suppose that module X's symbol code is \code{[b, c]} and the global symbol code is \code{[a, b, c, d]}.
  Then, the global symbol environment is $[1\mapsto\code{a}, 2\mapsto\code{b}, 3\mapsto\code{c}, 4\mapsto\code{d}]$,
  and module X's is $[2 \mapsto \code{b}, 3 \mapsto \code{c}]$.
  If the \code{Callstate} carries identifiers, X cannot call a function \code{a} via its function pointer $1$ because it first needs to lookup its environment with $1$ and find the corresponding identifier, but it is lacking.
  For this reason, we changed \code{Callstate} to carry more raw information, a function pointer itself, than an identifier.

  It is worth noting that \ccc{} takes a different approach; in their setting X's symbol environment is $[1\mapsto\code{extern a}, 2\mapsto\code{b}, 3\mapsto\code{c}, 4\mapsto\code{extern d}]$,
  where \code{extern} means that it just has an identifier and its body is missing.
  However, this symbol environment is quite different from the one originally used in \cc{}, $[1\mapsto\code{b}, 2\mapsto\code{c}]$.
  We deviated from \ccc{}'s approach because it is desirable to maintain each symbol environment as a mere relocation from the original ones.
  This property is beneficial when proving optimizations that are sensitive to symbol code/symbol environment, such as \code{Unusedglob}, which was missing in the days of \ccc{}.
  %% Also we believe our approach is more resilient to future extensions like \emph{dynamic loading};
  %% every time a new definition is loaded, \ccc{}'s approach needs to update all module's symbol environments while ours does not.

  %% is the same as global symbol environment.
  %% \footnote{Actually, the memory to identifier mapping are the same but identifier to definition mapping is different}

  %% The first approach, taken by \ccc{}, is to make each module's symbol environment to contain all the identifiers.
  %% In our interaction semantics, we wanted to keep each module's symbol environment to be a simple relocation of
\item
  \cc{}'s main function -- the entry point of the whole program -- does not accept any argument, and this fact is exploited in various proofs.
  We relaxed such proofs not to rely on empty arguments so that such proofs can be reused in the open setting where each module can be called with arbitrary arguments. 
  %% Changes are miniscule except for the \code{Stacking.v}.
  \\
  %
  In detail, in \ccm{} we employ a special gadget called \emph{dummy stack} in three consecutive intermediate languages (\emph{LTL, Linear, Mach}) to take initial arguments into account.
  Those languages are unusual because the callee reads arguments from caller's stack frame; higher-level languages directly pass the argument, and a lower-level language (assembly) passes arguments via memory.
  Therefore, unlike \cc{}, where the initial stack frame of those languages is empty, in \ccm{}, we put a \emph{dummy stack} that only contains initial arguments but nothing else.
  \footnote{Correspondingly, the \code{corestep} is defined as follows: $\code{corestep}(st_0, e, st_1) \defeq st_0 \estep{e} st_1 \land \code{get\_stack}(st_0) \neq []$.}
  For this, we relaxed \cc{}'s simulation proof so that it can be used in both \cc{} and \ccm{}.
  The two passes that introduce (\code{Allocation.v}; translation to \emph{LTL}) and eliminates (\code{Asmgen.v}; translation from \emph{Mach}) dummy stack required slight change.
  The other passes whose source and target both have dummy stack (\eg \code{Linearize.v}, \code{Tunneling.v}) are almost unchanged,
  except for the \code{Stacking.v} whose proof is deeply involved with the structure of stack frame.
  %

\item
  In \cc{}'s assembly semantics, the $\code{RSP}$ register is initialized to an integer 0.
  This is okay for \cc{} generated assemblies where the initial value of $\code{RSP}$ is never used, but in our setting, we may include a hand-written assembly module, so a more faithful formalization would be desirable.
  Therefore, we initialize $\code{RSP}$ with a junk pointer.\footnote{we can also use \texttt{undef} value but junk pointer is more convenient in simulation proof}.
  As we did in \Cref{sec:compiler:solution:model}, we massaged the initial memory to contain a junk block (block number 1), and initialize $\code{RSP}$ with that address.
  %% We prefer junk pointer because 

%% 4) Generalizing "step" relation to respect "symbol environment".
\end{itemize}
The correctness result of \cc{} remains the same, and the reason is as follows.
\begin{itemize}
\item The semantics of the source language is unchanged. In \code{ClightBigstep.v} the equivalence between small-step style and big-step style is proven.
  We did not modify the big-step style at all, and the equivalence is still proven.
\item The only thing that is changed in the target language's semantics is the initial value of RSP register, where our version admits more behavior. 
\end{itemize}

Additionally, we have strengthened the C type checker in order to prove \Cref{thm:results-adequacy-c}; our modification is faithful w.r.t modern C compliers.%\footnote{Probably valid w.r.t. C standards too.}
Specifically, our modified type checker additionally rejects the following cases: (1) void function returning a value, (2) non-void function not returning a value, and (3) ill-formed composite types.
With these enhancements, we proved that when executing a well-typed C module, arguments of a function call and a return value of a function are always well-typed.
\footnote{\cc{}'s type soundness theorem already maintains all the value in its state are well-typed.}
This additional property guarantees that nothing weird happens in the inter-module step, which is necessary for proving \Cref{thm:results-adequacy-c}.


%% Note that, in spite of these technical modifications, mathematical meaning of the semantics is unchanged.  
%% - For source languages, see `/cfrontend/ClightBigstep.v` and `/cfrontend/Cstrategy.v` which proves the small-step variant and big-step variant of the semantics are equivalent. We modified small-step variant but it is still equivalent to (unmodified) big-step semantics.
%% - For target language (`/x86/Asm.v`), we didn't modify the semantics at all.


%% 5) Ctyping

%% mathematical meaning is proven to be equivalent as before.

%% Recall that by prohibiting mutual recursion, \ccx{} was able to use closed simulations which do not require the ``guarantee'' condition.

%% 1) Proving "guarantee" conditions of each translations.  
%% In the context of open simulations, CompCert only "rely" on it (by external call axioms) but does not "guarantee" it.  
%% To address this gap, we needed to prove "guarantee" conditions for each per-pass proofs.  
%% We needed to prove additional properties for injection passes only.  

%% 2) Changing "Callstate" of each language to carry function pointer, instead of function definition.  
%% External module's function definition may not be visible in the local module, but it should be able to call it with function pointer.  

%% 3) Relaxing each translation's simulation relation so that first stack (`main`, in CompCert) may have called with arguments.  
%% This change affects only three passes, and except for the Stackingproof the change is very small.  

%% 4) Generalizing "step" relation to respect "symbol environment".  
