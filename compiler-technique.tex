\section{Formalization of Verification Techniques}
\label{sec:compiler:technique}

Now we present the formalization of our verification techniques.
%% As discussed in \Cref{sec:overview-verification:solution},
We parameterize the notion of open simulation presented in
\Cref{sec:rusc:background} with three parameters: memory relations, symbol
relations, and memory predicates.  We present formal details about
mixed simulation (\Cref{sec:main-verification:mixedsim}),
the three parameters (\Cref{sec:main-verification:parameter}),
the parameterized open simulations (\Cref{sec:main-verification:opensim}), and
their horizontal compositionality and adequacy theorems (\Cref{sec:main-verification:theorems}).
Finally, we present some interesting instances for the three parameters (\Cref{sec:main-verification:instances}).
%% Then we introduce mixed simulation that allows forward
%% reasoning in the presence of nondeterminism (\Cref{sec:main-verification:mixedsim}).

\subsection{Mixed Simulation}
\label{sec:main-verification:mixedsim}
In this section, we flesh out technical details about mixed simulation.
Recall that our mixed simulation technique supports three modes; (1) backward, (2) forward with locally deterministic target states, and (3) forward with locally receptive source states and locally determinate target states.
Technically, modes (2) and (3) are implemented as an instance of a more general mode (4) that subsumes the two.
Therefore, we first define that ``general'' mode and then explain the others.

Intuitively, mode (2) imposes a strong restriction on the target and no restriction on the source, whereas mode (3) imposes a moderate restriction on both source and target.
In mode (4), we introduce a parameter ST that controls the amount of restriction imposed on the source and target.
%% $\textrm{SimilarTraces} \eq \{ \sim \; \in \; \option \Events \times \option \Events \;|\; (\forall x, \none \; \sim \; x \implies x \eq \none) \land (\forall x, x \; \sim \; \none \implies x \eq \none) \}$ 
\[
\begin{array}{l@{}l}
\textrm{ST} \in \textrm{SimilarTraces} = & \{\; \sim \; \in \; \option \Events \times \option \Events \;|\; \\
                         & \;\; (\forall \, e, \; \none \sim e \implies e = \none) \land (\forall \, e, \; e \sim \none \implies e = \none) \}
\end{array}
\]
Then, depending on the parameter ST, we define the notion of locally receptive and locally determinate as follows.
\[
\begin{array}{l@{}l}
\textrm{receptive\_at} (s: \texttt{state})   & \; \defeq \; \forall \, e_1, s_1, e_2, (s \estep{e_1} s_1 \land e_1 \sim e_2) \implies \exists s_2, s \estep{e_2} s_2 \\
\textrm{determinate\_at} (s: \texttt{state}) & \; \defeq \; \forall \, e_1, s_1, e_2, s_2, (s \estep{e_1} s_1 \land s \estep{e_2} s_2) \implies \\
                                             & \;\;\;\;\;\;\; (e_1 \sim e_2 \land (e_1 = e_2 \implies s_1 = s_2)) \\
\end{array}
\]
Finally, mode (4) is defined as follows:
%% \begin{itemize}
\begin{enumerate}[resume]
\item There exists ST such that receptive\_at($\mssrc$) holds and\\
  $\forall e, \mssrc',~ \mssrc \estep{e} \mssrc' \implies {} $ \\
  $ \exists \mstgt',~ \mstgt \exstep{\tau}^{\raisebox{-1mm}{\scriptsize$\ast$}} \exstep{e}\exstep{\tau}^{\raisebox{-1mm}{\scriptsize$\ast$}} \mstgt' \land (\mssrc', \mstgt') \in R$\\
  where $\ms \exstep{e} \ms'$ denotes that determinate\_at($\ms$) holds and $\ms \estep{e} \ms'$.
\end{enumerate}
\begin{proof}
We write a rough outline of the adequacy proof of this mode.
If the $\mssrc$ cannot take any step, it is undefined behavior, so the proof is over. Otherwise, we have $e_1$ such that $\mssrc \estep{e_1} \mssrc'$.
By applying mode (4) proof with $e_1$ and $\mssrc'$, we have ($\tau$ steps omitted for brevity) $\mstgt \estep{e_1} \mstgt' \land (\mssrc', \mstgt') \in R$. Now, it suffices\footnote{We omit the details about coinductive reasoning here.} to prove this:
\[\forall e_2, \mstgt'', \mstgt \estep{e_2} \mstgt'' \implies \exists \mssrc', \mssrc \estep{e_2} \mssrc' \land (\mssrc', \mstgt'') \in R.\]

By determinate\_at, we have $(e_1 \sim e_2 \land (e_1 = e_2 \implies \mstgt' = \mstgt''))$.
If $e_1 = e_2$, we have $\mstgt' = \mstgt''$. We finish the proof by instantiating $\exists \mssrc'$ with $\mssrc'$.
Otherwise, we still have $e_1 \sim e_2$.
Then, by receptive\_at, we have $\mssrc''$ such that $\mssrc \estep{e_2} \mssrc''$.
Finally, by applying mode (4) proof with $e_2$ and $\mssrc''$, we get $\mstgt''$ such that $\mstgt \estep{e_1} \mstgt'' \land (\mssrc'', \mstgt'') \in R$.
We finish the proof by instantiating $\exists \mssrc'$ with $\mssrc''$.

\end{proof}
%% We write rough outline of the adequacy proof of this mode.
%% If the $\mssrc$ cannot take any step, it is undefined behavior so the proof ends. Otherwise, we have $e$ such that
%% \begin{itemize}
%% $\mssrc \estep{e} \mssrc'$; \\
%% $\mstgt \estep{e} \mstgt'$ by the mode (4) proof; \\
%% Now, we need to show that $\forall e', \mstgt \estep{e'} \mstgt' \implies \mssrc \estep{e'} \mssrc'$ \\
%% \end{itemize}

We get modes (2) and (3) by instantiating $\sim$ of (4) with proper instances; the former with equality, and the latter with a certain relation called ``match\_traces'' from \cc{}.
%% \end{itemize}

\subsection{Parameters for Open Simulations}
\label{sec:main-verification:parameter}

\Cref{fig:simulation-parameters} presents
the sets of three parameters for open simulations:
the set of memory relations $\MREL$, the set of symbol relations $\SREL$, and the set of memory predicates $\MPRED$.
%% , which we will explain in details.
%% Note that this section will be easier to follow if read in color:
%% \textcolor{myred}{memory relations} are presented in red, \textcolor{darkgreen}{symbol relations} in
%% green, and \textcolor{myblue}{memory predicates} in blue.

\begin{figure}[t!]
\footnotesize

\makebox[\linewidth]{\makebox[1.2\linewidth]{
\begin{minipage}{1.2\linewidth}
  \vspace{3mm}\mbox{}\\
  \mytitle{memory relation}\\
$
  %% \[
  \begin{stackTL}
  %% \MREL = \{
  %% (\ty, \texttt{mem}, \sqsubseteq, \sqsubseteq_\pub, \texttt{vrel}) \; | \;
  %% \ty \in \textrm{Set}, \texttt{mem} \in \ty \rightarrow (\Memory \times \Memory), \sqsubseteq, \sqsubseteq_\pub \; \in \powset{\ty \times \ty}, \texttt{vrel} \in \ty \rightarrow \powset{\Val \times \Val}
  %% %% & \ty \in \textrm{Set}, \texttt{src} \in \ty \rightarrow \Memory, \texttt{tgt} \in \ty \rightarrow \Memory, \sqsubseteq, \sqsubseteq_\pub \; \in \powset{\ty \times \ty}, \texttt{vrel} \in \ty \rightarrow \powset{\Val \times \Val}
  %% \}
  %% \\
  %% \MREL \text{ is a \emph{MemRel} if the following properties hold.}
  %% \\
  %% \\
  %% \caselabel{properties}\\
  %% {\begin{array}{l@{\;}l}
  %%   \caselabel{1} & \sqsubseteq_\pub \text{ is preorder}\\
  %%   \caselabel{2} & \sqsubseteq_\pub \subseteq \sqsubseteq\\
  %%   \caselabel{3} & \forall \mrel, \mrel',\; \mrel \sqsubseteq \mrel' \implies \texttt{vrel}(\mrel) \subseteq \texttt{vrel}(\mrel')\\
  %% \end{array}}
  %% \\
  %% %% \texttt{mrel}: \ty \rightarrow \powset{\Memory \times \Memory} \defeq \lambda \mrel, \mem_\src, \mem_\tgt, ($\MREL{}$.\texttt{src} \; \mrel) = \mem_\src \land ($\MREL{}$.\texttt{tgt} \; \mrel) = \mem_\tgt \\
  %% \caselabel{derived}\\
  %% \begin{array}{r@{\;}l}
  %% \texttt{mrel} & \defeq \lambda \mrel, \mem_\src, \mem_\tgt,\; ($\MREL{}$.\texttt{src} \; \mrel) = \mem_\src \land ($\MREL{}$.\texttt{tgt} \; \mrel) = \mem_\tgt \\
  %% \args_\src \succsim_{\mrel} \args_\tgt & \defeq (\args_\src.\texttt{f}, \args_\tgt.\texttt{f}) \in \texttt{vrel}(\mrel) \land
  %%   (\args_\src.\texttt{mem}, \args_\tgt.\textrm{mem}) \in \texttt{mrel}(\mrel) \land
  %%   (\args_\src.\texttt{vs}, \args_\tgt.\texttt{vs}) \in \overrightarrow{\texttt{vrel}(\mrel)}\\
  %% \retv_\src \succsim_{\mrel} \retv_\tgt & \defeq (\retv_\src.\texttt{v}, \retv_\tgt.\texttt{v}) \in \texttt{vrel}(\mrel) \land
  %%   (\retv_\src.\texttt{mem}, \retv_\tgt.\textrm{mem}) \in \texttt{mrel}(\mrel)\\
  %% \end{array}
  %% \\
  %% .
  %% \\
  %% .
  %% \\
  %% .
  %% \\


  \begin{array}{l@{}l}
  \MREL \in \textrm{MemRel} = \span
  \\\quad
  \setof{
    (&\ty, \sqsubseteq, \sqsubseteq_\weak, \texttt{mrel}, \texttt{vrel}) \in (\textrm{Set} \times \powset{\ty \times \ty} \times \powset{\ty \times \ty} \times (\ty \rightarrow \powset{\Memory \times \Memory}) \times (\ty \rightarrow \powset{\Val \times \Val})) \; |
  \\\quad
  & (\sqsubseteq \text{ is preorder}) \; \land \;
    (\sqsubseteq \, \subseteq \, \sqsubseteq_\weak) \; \land \;
    (\forall \mrel, \mrel',\; \mrel \sqsubseteq \mrel' \implies \texttt{vrel}(\mrel) \subseteq \texttt{vrel}(\mrel')) \; \land \;
  \\\quad
  & (\forall \mrel, i,~ (\textrm{Vint } i, v_\tgt) \in \texttt{vrel}(\mrel) \implies v_\tgt = \textrm{Vint } i)
  }
  \end{array}
  \\
  %% \texttt{mrel}: \ty \rightarrow \powset{\Memory \times \Memory} \defeq \lambda \mrel, \mem_\src, \mem_\tgt, ($\MREL{}$.\texttt{src} \; \mrel) = \mem_\src \land ($\MREL{}$.\texttt{tgt} \; \mrel) = \mem_\tgt \\
  \begin{array}{r@{\;}l}
  \args_\src \succsim_{\mrel} \args_\tgt \defeq &
    (\args_\src.\texttt{m}, \args_\tgt.\texttt{m}) \in \texttt{mrel}(\mrel) \land
    (\args_\src.\texttt{f}, \args_\tgt.\texttt{f}) \in \texttt{vrel}(\mrel) \land
    (\args_\src.\texttt{vs}, \args_\tgt.\texttt{vs}) \in \overrightarrow{\texttt{vrel}(\mrel)} \; \land \\
    & (\args_\src.\texttt{rs}, \args_\tgt.\texttt{rs}) \in \overrightarrow{\texttt{vrel}(\mrel)} \\
  \retv_\src \succsim_{\mrel} \retv_\tgt \defeq &
    (\retv_\src.\texttt{m}, \retv_\tgt.\texttt{m}) \in \texttt{mrel}(\mrel) \land
    (\retv_\src.\texttt{v}, \retv_\tgt.\texttt{v}) \in \texttt{vrel}(\mrel) \land
    (\retv_\src.\texttt{rs}, \retv_\tgt.\texttt{rs}) \in \overrightarrow{\texttt{vrel}(\mrel)} \\
  \end{array}
  \end{stackTL}
$
  %% \]
%% \youngju{\texttt{vrel} should be $\MREL{}$.\texttt{vrel}. (1) change? note that Former (and overview) is textrm but latter is texttt (2) specify $\MREL{}$ only when it is confusing (e.g. \ty)}
\\
\end{minipage}%
}}
\\
\makebox[\linewidth]{\makebox[1.2\linewidth]{
\begin{minipage}{1.2\linewidth}
  \vspace{3mm}\mbox{}\\
  \mytitle{symbol relation}\\
$
  \begin{stackTL}
  \begin{array}{@{}l@{}l@{~}l@{}}
    \SREL \in \textrm{SymbRel} = \span \span
\\\quad
\setof{&(\ty, \sqsubseteq, \simsk, \simskenv) \in
(\textrm{Set} \times \powset{\ty \times \ty} \times (\ty \rightarrow \powset{\Skel \times \Skel}) \times (\ty \rightarrow \MREL.\ty \rightarrow \powset{\Skenv \times \Skenv})) \; | \span
\\\quad
& \caselabel{1} \;&  \sqsubseteq \textrm{is preorder}
\\\quad
& \caselabel{2} \;& \forall \skel_\src, \skel'_\src, \skel''_\src, \skel_\tgt, \skel'_\tgt, \skel''_\tgt,~ \skel''_\src = \skel_\src \plink \skel'_\src \land \skel''_\tgt = \skel_\tgt \plink \skel'_\tgt \implies {}
\\\quad
&                   &
\forall \srel, \srel',~ (\skel_\src, \skel_\tgt) \in \simsk(\srel) \land (\skel'_\src \; \skel'_\tgt) \in \simsk(\srel') \implies {}
\\\quad
&                   &
\exists \srel'',~ (\skel''_\src, \skel''_\tgt) \in \simsk(\srel'') \land \srel \sqsubseteq \srel'' \land \srel' \sqsubseteq \srel''
\\\quad
& \caselabel{3} \; & \forall \skel_\src, \skel_\tgt, \srel,~ (\skel_\src, \skel_\tgt) \in \simsk(\srel) \implies
\\\quad
  &                   & \exists \mrel,~ (\textrm{load\_mem}(\skel_\src), \textrm{load\_mem}(\skel_\tgt)) \in \texttt{mrel}(w) \; \land \;
(\textrm{load\_se}(\skel_\src), \textrm{load\_se}(\skel_\tgt)) \in \simskenv(\srel, \mrel)
\\\quad
& \caselabel{4} \; & \forall \srel, \mrel, \mrel',~ \mrel \sqsubseteq_\weak \mrel' \implies \simskenv(\srel, \mrel) \subseteq \simskenv(\srel, \mrel')
\\\quad
  %% & \caselabel{5} \; (& \forall \srel, \mrel, \skenv_\src, \skenv_\tgt, \val_\src, \val_\tgt,~ (\skenv_\src, \skenv_\tgt) \in \simskenv(\srel, \mrel) \land (\val_\src, \val_\tgt) \in \MREL.\texttt{vrel}(\mrel) \implies \\
  %% &                   & \val_\src \in \textrm{ftns}(\skenv_\src) \iff \val_\tgt \in \textrm{ftns}(\skenv_\tgt))\\
& \caselabel{5} \; & \forall \srel, \mrel, \skenv_\src, \skenv_\tgt,~ (\skenv_\src, \skenv_\tgt) \in \simskenv(\srel, \mrel) \implies \skenv_\src.\texttt{pubs} = \skenv_\tgt.\texttt{pubs} \; \land \;
\\\quad
&                   & \forall (v_\src, v_\tgt) \in \MREL.\texttt{vrel}(\mrel),~ v_\src \in \textrm{ftns}(\skenv_\src) \implies v_\tgt \in \textrm{ftns}(\skenv_\tgt)
\\\quad
& \caselabel{6} \; & \forall \srel, \srel', \mrel, \skel_\src, \skel_\tgt, \skenv_\src, \skenv_\tgt,~

                     \srel \sqsubseteq \srel' \land (\skel_\src, \skel_\tgt) \in \simsk(\srel) \land (\skenv_\src, \skenv_\tgt) \in \simskenv(\srel', \mrel) \implies
\\\quad
&                   & (\skenv_\src \textbar_{\skel_\src}, \skenv_\tgt \textbar_{\skel_\tgt}) \in \simskenv(\srel, \mrel)
\\\quad
& \caselabel{7} \; & \forall \srel, \mrel, \skenv_\src, \skenv_\tgt, \args_\src, \args_\tgt,~ (\skenv_\src, \skenv_\tgt) \in \simskenv(\srel, \mrel) \land \args_\src \succsim_{\mrel{}} \args_\tgt \implies
\\\quad
  &                   & \forall e, \retv_\src,~ \textrm{external\_call} \; \skenv_\src \; \args_\src \; e \; \retv_\src \implies \exists \retv_\tgt,~ \textrm{external\_call} \; \skenv_\tgt \; \args_\tgt \; e \; \retv_\tgt \land
                        \exists \mrel' \sqsupseteq \mrel,~ \retv_\src \succsim_{\mrel'} \retv_\tgt
  }
  \end{array}
  \\
  \end{stackTL}
$
\end{minipage}%
}}
\\
\makebox[\linewidth]{\makebox[1.2\linewidth]{
\begin{minipage}{1.2\linewidth}
  \vspace{3mm}\mbox{}\\
  \mytitle{memory predicate}\\
$
  \begin{stackTL}
  \begin{array}{l@{}l}
  \MPRED \in \textrm{MemPred} = \span \\
  \quad \setof{ (&\ty, \sqsubseteq, \sqsubseteq_\weak, \texttt{mpred}, \texttt{vpred}, \texttt{sepred}) \in (\textrm{Set} \times \powset{\ty \times \ty} \times \powset{\ty \times \ty}
                                                                                                           \times (\ty \!\rightarrow\! \powset{\Memory}) 
                                                                                                           \times (\ty \!\rightarrow\! \powset{\Val}) \times (\ty \!\rightarrow\! \powset{\Skenv})) \; | \\
  & (\sqsubseteq \text{ is preorder}) \; \land \;
  (\sqsubseteq \, \subseteq \, \sqsubseteq_\weak) \; \land \;
  (\forall \mpred, \mpred',\; \mpred \sqsubseteq \mpred' \implies \texttt{vpred}(\mpred) \subseteq \texttt{vpred}(\mpred')) \; \land \; \\
  & (\text{\texttt{sepred} should satisfy the unary version of $\simskenv$'s conditions where $\SREL{}.\!\!\sqsubseteq$ and $\simsk$ are the total relations})
  }
  \end{array}
  \\
  \begin{array}{r@{\;}l}
  %% \texttt{mpred}(\mpred) & \defeq \setof{$\MPRED{}$.\texttt{mem}(\mpred)} \\

  \texttt{cpred}(\mpred) & \defeq \{ \args \in \Args \suchthat \args.\texttt{m} \in \texttt{mpred}(\mpred) \land \args.\texttt{f} \in \texttt{vpred}(\mpred) \land
                                                \args.\texttt{vs} \in \overrightarrow{\texttt{vpred}(\mpred)} \land \args.\texttt{rs} \in \overrightarrow{\texttt{vpred}(\mpred)} \} \\
  \texttt{rpred}(\mpred) & \defeq \{ \retv \in \Retv \suchthat \retv.\texttt{m} \in \texttt{mpred}(\mpred) \land \retv.\texttt{v} \in \texttt{vpred}(\mpred) \land
                                                \retv.\texttt{rs} \in \overrightarrow{\texttt{vpred}(\mpred)} \} \\
  %% \args_\src \in \texttt{cpred}(\mpred) & \defeq \args_\src.\texttt{m} \in \texttt{mpred}(\mpred) \land \args_\src.\texttt{f} \in \texttt{vpred}(\mpred) \land
  %%                                               \args_\src.\texttt{vs} \in \overrightarrow{\texttt{vpred}(\mpred)} \land \args_\src.\texttt{rs} \in \overrightarrow{\texttt{vpred}(\mpred)} \\
  %% \retv_\src \in \texttt{rpred}(\mpred) & \defeq \retv_\src.\texttt{m} \in \texttt{mpred}(\mpred) \land \retv_\src.\texttt{v} \in \texttt{vpred}(\mpred) \land
  %%                                               \retv_\src.\texttt{rs} \in \overrightarrow{\texttt{vpred}(\mpred)} \\

  \end{array}
  \end{stackTL}
$
%% \\
%% \youngju{지금 2번쨰줄 아래가 짤리는데 왜인지 모르겠습니다}\\
%% \youngju{pred 대신 prd?}\\
\end{minipage}
}}

\caption{Three parameters for open simulations}
\label{fig:simulation-parameters}
\end{figure}


\myparagraph{Memory Relation}

The first parameter ranges over Kripke-style memory/value relations in $\MREL$.
%~\cite{DBLP:conf/popl/AhmedDR09}
Following \cite{DBLP:conf/icfp/DreyerNB10,pb}, we model the
evolution of memory relations using \emph{possible worlds} and \emph{private and public transitions}
over the worlds.
Note that this parameter will be instantiated with the three memory relations used in \cc{}---namely memory
identity, extension, and injection---and the memory injection with module-local invariants we introduced.
%% for verifying \texttt{Unreadglob}.

A memory relation in $\MREL$ consists of $(i)$ a set $\code{t}$ of possible worlds; $(ii)$ \emph{public}
and \emph{private} transition relations $\sqsubseteq$ and $\sqsubseteq_\weak$ over the worlds;
and $(iii)$ for each world $w \in \code{t}$, memory relation $\texttt{mrel}(w)$ and
value relation $\texttt{vrel}(w)$.  A world $w$ represents an invariant on the memory, which
can evolve over time according to the public/private transition relations,
as we discussed in \Cref{sec:overview-verification:injection}.
%% public transitions $\sqsubseteq$ represent possible evolution of the world before and after an external function
%% call, and private transitions $\sqsubseteq_\weak$ represent possible evolution between any two interaction points with
%% external modules. %(\eg{} between two invocations of external functions).
%% For example, we use private transitions to capture the permission changes in the arguments area of the stack,
%% which cannot be public transitions according to the \newnewrevision{enriched} memory injection.
%% For example, memory
%% injection's public transition encodes the invariant that injection should be increased and memory
%% access permissions are the same before and after a function call; but its private transition allows
%% certain changes in access permissions, as discussed in \Cref{sec:overview-semantics:solution}.  The
%% public transition relation is reflexive and transitive, and is a subset of the private transition
%% relation.
There are four natural well-formedness conditions, which are self-explanatory.
%% $\texttt{vrel}(w)$ should be monotone w.r.t. the world's private
%% transition, and it relates an integer value in source only with the same value in target.
We can also straightforwardly extend the value/memory relation to relations on $\Args$ and $\Retv$, denoted $\succsim_{\mrel}$.
%% For each
%% world $w \in \code{t}$, relations on the input/output of external function calls are defined in a
%% straightforward manner: function pointers, arguments, and results are related as values, and
%% input/output memories are related as memories.

\myparagraph{Symbol Relation}

The second parameter ranges over symbol relations in $\SREL$ that
relate information about global symbols (\eg which block each global
variable points to) in the source and target.  This parameter is
needed to verify optimizations like \code{Unusedglob},
\code{Unreadglob} that remove unnecessary static variables thereby
having non-identical symbol information in the source and target.

%% enables he verification of such
%% specifications and optimizations as \code{Unusedglob} that change symbol tables.
%% Verification of
%% all \cc{} optimizations, except for \code{unused-globs}, requires only a trivial symbol relation
%% in which the source and target symbols are exactly the same.  The main purpose of $\SREL$ is
%% guaranteeing a \emph{run-time} symbol relation, assuming all its modules satisfy \emph{compile-time}
%% symbol relation.  We allow modular reasoning of symbol relations using states and their
%% compatibility relation w.r.t. linking.  For example, the verification of \code{unused-globs} uses a
%% symbol relation whose state is the set of dropped symbols.  To model the evolution of symbol
%% relations under linking, we say two states are compatible if the former is a subset of the latter.
%% The compile-time and run-time symbol relations check whether the symbols in the state are actually
%% dropped.
%% Specifically,

A symbol relation in $\SREL$ consists of $(i)$ a set $\code{t}$ of symbol relation states; $(ii)$ 
an extension relation $\sqsubseteq$ on the states; $(iii)$ for each state $d$,
a (compile-time) symbol code relation $\texttt{screl}(d)$; and $(iv)$ for each state $d$ and world $w \in \MREL.\code{t}$,
(run-time) symbol environment relation $\texttt{serel}(d,w)$.
There are seven well-formedness conditions:
%% $\SREL$ should satisfy the following conditions:
$(1)$ the extension relation $\sqsubseteq$ is transitive and reflexive;
$(2)$ \texttt{screl} is closed under the syntactic linking;
$(3)$ if symbol codes are related by \texttt{screl}, then \newnewrevision{the initial memories and symbol environments loaded by \textrm{load\_mem} and \textrm{load\_se} are related by \texttt{mrel} and \texttt{serel}, respectively};
$(4)$ \texttt{serel} is monotone w.r.t. private transitions;
$(5)$ \newnewrevision{for symbol environments related by \texttt{serel}, their public symbols are identical and their functions have the same signatures};
$(6)$ \texttt{serel} is compatible with $\sqsubseteq$: for $d \sqsubseteq d'$, $\texttt{serel}(d')$ restricted on $\texttt{screl}(d)$ should be in $\texttt{serel}(d)$; and
%% is preserved w.r.t. compatible restriction on symbol codes; and
$(7)$ the memory and symbol relations should be compatible with \cc{}'s axiom about system calls (\ie \textrm{external\_call}).
%% external function symbols related by \texttt{serel} satisfy the condition for open simulation.

% \cdashbox{darkgreen}{$\texttt{screl}(d)$} statically holds for some $d \sqsubseteq d'$ at compile time.



% (0) \SREL{}는 크게 3가지로 instantiate 된다: identity, drop (unusedglob용), invariant (spec 증명용).
%    t 는 두 symbol code (lanugage-dependent 한 부분을 걷어낸 코드)를 relate (screl) 하는데 쓰인다.
%    identity의 경우 unit 타입이고 screl은 항상 true이다.
%    drop의 경우 set of symbol이고 screl은 (i) 실제로 그 symbol이 빠졌는지 (ii) 그 symbol을 refer하는게 없는지 체크한다. (injection 매핑에서 빼줘야 하기 때문)
%    invariant의 경우 역시 set of symbol이고 screl은 (i) 실제로 그 symbol이 invariant를 만족하는지 (ii) 그 symbol을 refer하는게 없는지 체크한다. (injection 매핑에서 빼줘야 하기 때문)
%    $\sqsubseteq$의 의미는 대충 set inclusion 이다. \\
%    이 interface의 모든 목적은 static하게 $\cfbox{darkgreen}{screl}$ 을 guarantee 해주면 runtime에 $\cdashbox{darkgreen}{serel}$ 를 rely 받는 것이다. \\

% (5) serel이면 (i) public symbol들이 같고 (ii) 어떤 relate되어있는 src/tgt value가 있을 때, src가 function이면 tgt도 function이고 둘의 signature가 같다. \\
% (6) 큰 senv가 큰 d'에 대해 serel이고, 작은 sc가 작은 d에 대해 screl이면, 각각의 큰 senv를 작은 sc로 restrict 한 것도 relate 되어있다. \\
% (7) external call axiom \\



\myparagraph{Memory Predicate}

The third parameter ranges over Kripke-style memory predicates in $\MPRED$,
which are needed to modularly verify \cc{}'s analysis engines such as value analysis (see \Cref{sec:main-verification:opensim}).
%% based on which is proved the soundness of compiler's analyses such as \cc{}'s value analysis.
$\MPRED$ is essentially a unary version of $\MREL$ combined with $\SREL$
where $\SREL{}.{\sqsubseteq}$ and $\simsk$ are taken as the total relations (\ie relating everything):
it consists of
$(i)$ the set $\code{t}$ of possible worlds;
$(ii)$ public and private transition relations $\sqsubseteq$ and $\sqsubseteq_\weak$ over the worlds, respectively; and
$(iii)$ for each world $w \in \code{t}$, a memory predicate $\textrm{mpred}(w)$,
a value predicate $\textrm{vpred}(w)$, and a symbol environment predicate $\textrm{sepred}(w)$.
The well-formedness conditions are self-explanatory.
%% Similarly to $\MREL$, $(i)$ $\MPRED$ has world's public and private transitions; $(ii)$ its value
%% predicate should be monotone w.r.t. the world's private transition; and $(iii)$ for each world
%% $w \in \code{t}$, relations on the input/output of external function calls are defined in a
%% straightforward manner.  Similarly to $\SREL$, $\MPRED$ has a few conditions on the symbol
%% environment predicate, which we omit in \Cref{fig:simulation-parameters} for brevity.

% \youngju{3, 4, 6이라고 해놓은거 다른 표현으로 바꿨습니다}
%% \youngju{
%%       \ccc{}가 개발될 당시에는 Value Analysis가 없었고, \ccx{}는 ($\MREL{}$에서와 마찬가지로) closed simulation이기 때문에 문제가 없다.
%%       \ccc{}의 후속연구인 \cascc{}에서도 Value Analysis가 들어간 pass들은 지원하지 않는다.
%%       그러니까 우리가 이 문제를 처음으로 tackle 하는}


\subsection{Open Simulations with Parameters}
\label{sec:main-verification:opensim}

\begin{figure}[t!]
\footnotesize

\vspace*{-20mm}
\begin{minipage}[t][0.98\textheight]{\linewidth}
\begin{minipage}{\linewidth}
\mytitle{sim:states}\\
$
  \begin{stackTL}
  match\_states \in \textrm{open\_sim}(\msem_\src, \msem_\tgt, sound\_state) \defeq \\
  \forall \mrel, \forall ((\memsrc, \stsrc), (\memtgt, \sttgt)) \in match\_states(\mrel), \cdashbox{myblue}{($\exists \mpred,~ (\memsrc, \stsrc) \in sound\_state(\mpred)$)} \implies \\
  %% \mrel, 
  {\begin{array}{l@{\;}l@{\;}l}
  & \caselabel{STEP} & \msem_{\src}.\texttt{at\_external}(\memsrc, \stsrc) = \none \land \msem_{\src}.\texttt{halted}(\memsrc, \stsrc) = \none \; \land \\
  & & \forall e, \memsrc', \stsrc',~ (\memsrc, \stsrc) \estep{e} (\memsrc',\stsrc') \implies \\[.5mm]
  & & \exists \memtgt',\sttgt',~ (\memtgt,\sttgt) \estep{\tau}^{\raisebox{-1mm}{\scriptsize$\ast$}} \estep{e}\estep{\tau}^{\raisebox{-1mm}{\scriptsize$\ast$}} (\memtgt',\sttgt') \land{}\\[.5mm]
  & & \cfbox{myred}{$\exists \mrel' \sqsupseteq \mrel$},~ ((\memsrc',\stsrc'), (\memtgt',\sttgt')) \in match\_states(\mrel') \\[1.2mm]
  %% \lor & \caselabel{CALL} & \cfbox{myred}{$\exists \args_\src, \args_\tgt, \mrel{}' \sqsupseteq_\weak \mrel{},~ \args_\src \succsim_{\mrel{}'} \args_\tgt$} \land \msem_{\src}.\texttt{at\_external}(\memsrc, \stsrc) = \some{\args_\src} \land \msem_{\tgt}.\texttt{at\_external}(\memtgt, \sttgt) = \some{\args_\tgt} \; \land \; \\
  \lor & \caselabel{CALL} & \cfbox{myred}{$\exists \mrel{}' \sqsupseteq_\weak \mrel{}$},~ \exists \args_\src, \args_\tgt,~ \cfbox{myred}{$\args_\src \succsim_{\mrel{}'} \args_\tgt$} \land \\
  & & \msem_{\src}.\texttt{at\_external}(\memsrc, \stsrc) = \some{\args_\src} \land \msem_{\tgt}.\texttt{at\_external}(\memtgt, \sttgt) = \some{\args_\tgt} \; \land \; \\
  & & \cdashbox{myred}{$\forall \mrel{}'' \sqsupseteq \mrel{}'$}, \forall \retv_\src, \retv_\tgt, \cdashbox{myred}{$\retv_\src \succsim_{\mrel{}''} \retv_\tgt$} \implies \\
  & & \forall \memsrc', \stsrc',~ \msem_{\src}.\texttt{after\_external}(\stsrc, \retv_\src) = \some{(\memsrc', \stsrc')} \implies \\[.5mm]
  & & \exists \memtgt', \sttgt',~ \msem_{\tgt}.\texttt{after\_external}(\sttgt, \retv_\tgt) = \some{(\memtgt', \sttgt')} \; \land \\
  %% & & \cfbox{myred}{$\exists \mrel''', \mrel''' \sqsupseteq \mrel \land \mrel''' \sqsupseteq_\weak \mrel'' \land (\memsrc',\memtgt') \in \mathrm{mrel}(\mrel''')$} \land \\
  & & \cfbox{myred}{$\exists \mrel''' \sqsupseteq_\weak \mrel'',~ \mrel''' \sqsupseteq \mrel$} \land ((\memsrc',\stsrc'), (\memtgt',\sttgt')) \in match\_states(\mrel''') \\[1.8mm]
  \lor & \caselabel{RET} & \cfbox{myred}{$\exists \mrel{}' \sqsupseteq \mrel{}$}, \exists \retv_\src, \retv_\tgt, \cfbox{myred}{$\retv_\src \succsim_{\mrel{}'} \retv_\tgt$} \land \\
       & & \msem_{\src}.\texttt{halted}(\memsrc, \stsrc) = \some{\retv_\src} \land \msem_{\tgt}.\texttt{halted}(\memtgt, \sttgt) = \some{\retv_\tgt} \\
  \end{array}}\\
  \end{stackTL}
$
%% \youngju{First line of (CALL) case is overflowed, we need to reduce text size somehow. I suggest to rollback $\msem$ into S}
\end{minipage}
\vspace{1mm}



\begin{minipage}{1\linewidth}
\mytitle{sim:modsem}\\
$
  \begin{stackTL}
  \msem_{\src} \succsim_{\srel{}, sound\_state} \msem_{\tgt} ~\defeq~ \exists match\_states \in \textrm{open\_sim}(\msem_{src}, \msem_{tgt}, sound\_state),\\
  \begin{array}{@{\quad}l@{\;}l@{\;}l}
  & \caselabel{INIT}
  & \forall \mrel{} \in \MREL{}.\texttt{t},~ \forall \args_\src, \args_\tgt,~ \cdashbox{myred}{$\args_\src \succsim_{\mrel{}} \args_\tgt$} \implies {} \\
  && \args_\src.\texttt{f} \in \textrm{ftns}(\msem_\src.\texttt{senv}) \land \args_\tgt.\texttt{f} \in \textrm{ftns}(\msem_\tgt.\texttt{senv})
    \implies \\
  && \cdashbox{green}{$(\msem_{src}.\texttt{senv}, \msem_{tgt}.\texttt{senv}) \in \simskenv(\srel, \mrel)$} \implies \forall (\memsrc, \stsrc) \in \msem_{\src}.\texttt{init\_core}(\args_\src),\\
  && \exists (\memtgt, \sttgt) \in \msem_{\tgt}.\texttt{init\_core}(\args_\tgt), \\
  && \cfbox{myred}{$\exists \mrel{}' \sqsupseteq \mrel{}$},~ ((\memsrc, \stsrc), (\memtgt, \sttgt)) \in match\_states(\mrel') \\
  %% & ((\memsrc, \stsrc), (\memtgt, \sttgt)) \in \textrm{match\_states}(, \mrel{}) \\
  \end{array}
  \end{stackTL}
$
\end{minipage}
\vspace{1mm}




\begin{minipage}{0.70\linewidth}
\mytitle{sim:mod}\\
$
  \begin{stackTL}
  \module_{\src} \succsim \module_{\tgt} ~\defeq~ \exists \srel{} \in \SREL{}.\texttt{t},~
  \exists sound\_state: \MPRED{}.\texttt{t} \rightarrow \powset{\Memory \times \module_\src.\texttt{state}}, \\[1mm]
  \begin{array}{l@{\;}l@{\;}l}
  %% \qquad & ~ \span \\
  & \caselabel{1} & \cfbox{green}{$(\module_{\src}.\texttt{scode}, \module_{\tgt}.\texttt{scode}) \in \simsk(\srel)$} \vspace{0.5mm}\\
  \land & \caselabel{2} & \cfbox{myblue}{$\forall \skenv_{\src},~ sound\_state \in \text{open\_prsv}(\module_{\src}.\texttt{sem} \; \skenv_{\src})$} \vspace{0.5mm}\\
  \land & \caselabel{3} & \forall \srel' \sqsupseteq \srel,~ \forall \mrel,~ \cdashbox{green}{$\forall (\skenv_\src, \skenv_\tgt) \in \simskenv(\srel', \mrel)$},~ \\
        &               & \module_{\src}.\texttt{sem}\,(\skenv_{\src}) \succsim_{\srel{}, sound\_state} \module_{\tgt}.\texttt{sem}\,(\skenv_{\tgt}) \\
  \end{array}
  \end{stackTL}
$
\end{minipage}%
\begin{minipage}{0.30\linewidth}
\mbox{}\\[14mm]
\mytitle{sim:prog}\\
$
  \begin{stackTL}
  \Prg_{\src} \succsim \Prg_{\tgt} \defeq \\[1mm]
  \quad \forall i \in \mathbb{N},~ \Prg_\src[i] \succsim \Prg_\tgt[i] \\
  \end{stackTL}
$
\end{minipage}


\vspace{1mm}
\begin{minipage}{1\linewidth}
\mytitle{preservation}\\
$
\begin{stackTL}
sound\_state \in \textrm{open\_prsv}(\msem_\src) \defeq \\

\begin{array}{l@{\;}l@{\;}l@{}l}

& \caselabel{INIT} & \forall \mpred \in \MPRED.\ty, \cdashbox{blue}{$\forall \args_\src \in \texttt{cpred}(\mpred)$},~
   \cdashbox{blue}{$\msem_\src.\texttt{senv} \in \texttt{sepred}(\mpred)$} \implies \forall (\memsrc, \stsrc) \in \msem_\src.\texttt{init\_core}(\args_\src), \span \\
& & \cfbox{blue}{$\exists \mpred{}' \sqsupseteq \mpred{}$},~ (\memsrc, \stsrc) \in sound\_state(\mpred') \span \\[1mm]

\land & \caselabel{STEP} & \forall \mpred,~ \forall (\memsrc, \stsrc) \in sound\_state(\mpred), \forall e, \memsrc', \stsrc',~ (\memsrc, \stsrc) \estep{e} (\memsrc', \stsrc') \implies \span \\
& & \cfbox{blue}{$\exists \mpred{}' \sqsupseteq \mpred{}$},~ (\memsrc', \stsrc') \in sound\_state(\mpred') \span \\[1mm]

\land & \caselabel{CALL} & \forall \mpred,~ \forall (\memsrc, \stsrc) \in sound\_state(\mpred), \forall \args_\src,~ \msem_\src.\texttt{at\_external}(\memsrc, \stsrc) = \some{\args_\src} \implies \span \\
& & \cfbox{blue}{$\exists \mpred{}' \sqsupseteq_\weak \mpred{}$},~ \cfbox{blue}{$\args_\src \in \texttt{cpred}(\mpred')$} \; \land \; \span \vspace{0.5mm}\\
& & & \cdashbox{blue}{$\forall \mpred'' \sqsupseteq \mpred{}',~ \forall \retv_\src \in \texttt{rpred}(\mpred'')$},~ \forall \memsrc', \stsrc',~
       \msem_\src.\texttt{after\_external}(\stsrc, \retv_\src) = \some{(\memsrc', \stsrc')} \implies \vspace{0.5mm}\\
& & & \cfbox{blue}{$\exists \mpred''' \sqsupseteq_\weak \mpred'',~ \mpred''' \sqsupseteq \mpred \land \exists \memsrc' \in \texttt{mpred}(\mpred''')$} \land (\memsrc', \stsrc') \in sound\_state(\mpred''')\\[1.5mm]

\land & \caselabel{RET} & \forall \mpred,~ \forall (\memsrc, \stsrc) \in sound\_state(\mpred), \forall \retv_\src,~ \msem_\src.\texttt{halted}(\memsrc, \stsrc) = \some{\retv_\src} \implies \span \\
& & \cfbox{blue}{$\exists \mpred{}' \sqsupseteq \mpred{}$},~ \cfbox{blue}{$\retv_\src \in \texttt{rpred}(\mpred')$} \span \\
\end{array}\\
\end{stackTL}
$
%% \youngju{open sim이랑 open prsv랑 function application 괄호 방식 다름}
\end{minipage}
\vspace*{3mm}
\caption{Parameterized Open Simulations}
\label{fig:full-sim}
\end{minipage}
\\
\\
\\
\\
\\
\end{figure}

%% \youngju{\cfbox{myred}{$\mrel''' \sqsupseteq \mrel''$} it appears in guarantee but not in rely. Remove fbox?}
%% \\
%% \youngju{Need tight box: vspace between two ``after\_external'' is too short}
%% \\
%% \youngju{Function application a(b) and (a b) is mixed. -- vrel(a) and (sc\_compat a b).}
%% \\
%% \youngju{$\msem_\src.\skenv$ is not very readable}
%% \\
%% \youngju{sound\_state is too long}
%% \\
%% \youngju{Need to add analysis developer's interface here}


% \jeehoon{explain rely-guarantee?}

%% TODO: Fig.11에 STEP 케이스가 forward 뿐인데, backward 케이스는 생략되었음을 언급

\Cref{fig:full-sim} presents our parameterized open simulations, which
are given in the form of forward simulation for simplicity though
they are actually in the form of mixed simulation presented in
\Cref{sec:overview-verification:mixedsim}.
In this section, we omit $\MREL$,
$\SREL$, and $\MPRED$ when clear from context (\eg{} $\texttt{vrel}(w)$ for
$\MREL{}.\texttt{vrel}(w)$).  Also, $\cdashbox{black}{R}$ and $\cfbox{black}{G}$ means rely and
guarantee conditions for the external modules.
%% whose edge color indicates which parameters are
%% involved in the rely/guarantee reasoning.

\myparagraph{Simulation of Machine States}

%% \setlist[description]{font=\normalfont\textbullet\space}

A relation $match\_states$ on machine states is an \emph{open simulation} if all related states
either $(i)$ transition to related states, $(ii)$ invoke related external calls (hence the name
``open'' simulation), or $(iii)$ halt with related return values and memories.  Specifically, given
source and target module semantics $\msem_\src$, $\msem_\tgt$ and a (source) soundness predicate $sound\_state$ (discussed later),
%% relations $match\_states(w)$ for each possible world $w \in \MREL.\code{t}$
the relation $match\_states$ over worlds is an open simulation if
the relatedness of $\mssrc$ and $\mstgt$ at a world $w$ with the soundness of $\mssrc$ implies
one of the followings.
\begin{itemize}
\item \caselabel{STEP} The source and target states transition to related states.
  Specifically:
  \begin{itemize}[leftmargin=11mm]
  \item[\textbf{line 1:}] the source machine state takes intramodule steps, and
  \item[\textbf{line 2:}] if the source machine state transitions to a next state emitting an event $e$,
  \item[\textbf{line 3:}] then the target machine state is able to transition to a next state emitting the same
    event $e$, possibly with additional silent transitions, and
  \item[\textbf{line 4:}] the next states are related by $match\_states(w')$ for a public future world $w' \sqsupseteq w$.
  %% \item[\textbf{line 3:}] the
  %%   next source and target memories are related at a public future world $w' \sqsupseteq w$,
  %% \item[\textbf{line 4:}] and the next states are related by $match\_states(w')$.
  \end{itemize}
  \vskip 1mm
\item \caselabel{CALL} The source and target states invoke related external calls.
  Specifically:
  \begin{itemize}[leftmargin=11mm]
  \item[\textbf{line 1:}] certain external functions and arguments in the source and target are related at a private future world $w' \sqsupseteq_\weak w$, and
  \item[\textbf{line 2:}] the source and target machine states invoke the related external functions with the related arguments, and
  \item[\textbf{line 3:}] for any return values and memories related at any public future world $w'' \sqsupseteq w'$,
  \item[\textbf{line 4:}] if the source safely returns from the external call,
  \item[\textbf{line 5:}] then the target also safely returns from the external call, and
  \item[\textbf{line 6:}] the states after return are related by $match\_states(w''')$ for a world $w'''$ that is a private future of $w''$ and a public future of $w$.

  %% \item[\textbf{line 1:}] The source and target machine states are about to invoke external calls, whose
  %%   arguments are related at a private future world $w' \sqsupseteq_\weak w$,
  %% \item[\textbf{line 2:}] for any public future world $w'' \sqsupseteq w'$ and return values and memories related at $w''$,
  %% \item[\textbf{line 3:}] if the source safely returns from the external call,
  %% \item[\textbf{line 4:}] then the target also safely returns from the external call,
  %% \item[\textbf{line 5:}] and there exists a world $w'''$ that $(i)$ is a public future of $w$,
  %%   %% and $(ii)$ a private future of $w''$,
  %%   and $(ii)$ relates the next memories at $w'''$,
  %% \item[\textbf{line 6:}] and the next states are related by $match\_states(w''')$.
  \end{itemize}
  \vskip 1mm
\item \caselabel{RET} The source and target states halt with related values and memories.
  Specifically:
  \begin{itemize}[leftmargin=11mm]
  \item[\textbf{line 1:}] with return values and memories related at~$w'$
    for a public future world $w' \sqsupseteq w$,
  \item[\textbf{line 2:}] the source and target machine states halt.
  \end{itemize}
\end{itemize}


\myparagraph{Simulation of Module Semantics}

Module semantics are related if their initial machine states are related.
Specifically, for a symbol relation $d \in \SREL$ and a (source) soundness predicate $sound\_state$,
a target module semantics $\msem_\tgt$ simulates a source one $\msem_\src$ if for an open simulation $match\_states$:
\begin{itemize}
\item \caselabel{INIT} the initial machine states of $\msem_\src$ and $\msem_\tgt$ are related by $match\_states$.
Specifically: 
\begin{itemize}[leftmargin=11mm]
\item[\textbf{line 1:}] for any source and target call data related at any world $w \in \MREL$,
\item[\textbf{line 2:}] if the functions of the source and target call data belong to the modules and
\item[\textbf{line 3:}] the symbol environments are related at $d$ and $w$, then for any initial machine state of the source function call,
\item[\textbf{line 4:}] there exists an initial machine state of the target function call such that
\item[\textbf{line 5:}] the two initial machine states are related by $match\_states(w')$
  for $w'$ a public future of~$w$.
%\item[\textbf{line 3:}] if source and target symbol environments are related at $d$ and $w$,
%\item[\textbf{line 4:}] then for all possible source initial machine state,
%\item[\textbf{line 5:}] there exists a target initial machine state such that,
%\item[\textbf{line 6:}] the initial memories are related at a public future $w'$ of the current world $w$,
%\item[\textbf{line 7:}] and the initial machine states are again related by $match\_states(w')$.
\end{itemize}
\end{itemize}


\myparagraph{Simulation of Modules}
Modules are related if their module semantics are related. Specifically,
a target module $\module_\tgt$ simulates a source one $\module_\src$
if the following hold for a symbol relation $d \in \SREL$ and a soundness predicate $sound\_state$:
\begin{itemize}[leftmargin=11mm]
\item[\textbf{line 1:}] the source and target symbol codes are related at $d$,
\item[\textbf{line 2:}] $sound\_state$ satisfies the open preservation property (discussed below), and
\item[\textbf{line 3:}] for any symbol environments related at any symbol relation $d'$ extending $d$ and any world~$w$,
\item[\textbf{line 4:}] the source and target module semantics for the related symbol environments are related at $d$ and $w$.
\end{itemize}
Note that the symbol environments are related at $d'$, which represents the possible symbol information after linking with an arbitrary module,
while the module semantics are related at $d$, which represents the module's own symbol information.

\myparagraph{Simulation of Programs}
\newnewrevision{Two programs each of which consists of a list of modules are simulated if each corresponding modules are simulated.}

\myparagraph{Open Preservation with Parameters}
{\newnewrevisioncmd
\cc{} uses a relation $match\_states$ to prove correctness of a translation pass
and a predicate $sound\_state$ to prove correctness of the analyzer performing value analysis,
where $sound\_state$ specifies those states where the analysis results hold.
As we do for $match\_states$,
we perform a similar generalization from a closed setting to an open setting for $sound\_state$.
Specifically, 
we generalize the conditions for $sound\_state$ from preservation to open preservation
(\cf from simulation to open simulation);
and parameterize over memory predicates $\MPRED{}$ (\cf memory relations $\MREL{}$),
which intuitively encodes the analysis results of the analyzer.
Also, as we do for open simulation,
we prove that all \textrm{Clight} and \textrm{Asm} modules satisfy
open preservation with $\MPRED{}$, which intuitively means that
all those context modules preserve the analysis results of the analyzer.
Note that the definition of open preservation, \text{open\_prsv}, is essentially a unary version of
that of open simulation, where the \caselabel{INIT} case corresponds
to that of the module semantics simulation and the \caselabel{STEP},
\caselabel{CALL}, and \caselabel{RET} cases to those of the state simulation.

% For brevity, we omit a detailed explanation of the conditions.

%% Open simulation permits modular verification of compiler analyses and optimizations.
%% analyzer L sound-state open preservation
%% exists sound-state, ... Clight, Asm

%% compiler verification uses sound-state given as a parameter to open simulation, so that the verifier can prove optimizations assuming that the sound-predicate holds.
%% Analyzer prove, varifier uses.

%% >  the (source) soundness predicate `sound_state` [l. 923]
%% What is this predicate?
%% >> First, compiler passes are developed in a modular way so that optimization passes can "query" analysis passes, and rely on the analysis result without any extra obligation.
%% >> Such modular nature is reflected in the proof structure too.
%% >> In CompCert, sound_state quantifies the states that are congruent with analysis result (e.g., if analysis concluded that global variable "x" is always positive, sound_state quantifies only those states).
%% >> Then, the obligation for optimization passes is to establish "simulation" assuming "sound_state", while the obligation for analysis passes obligation is to establish "preservation" of "sound_state".
%% >> We extended such notions into an open setting, where "simulation" resulted in "open simulation" and "preservation" resulted in "open preservation."
%% >> We will add these explanations in our revision.

%% Maybe: In particular, $(iii)$ is necessary for proving that context modules are self-related.

%% For this, we
%% require the analyzer for a language $L$ to be equipped with $(i)$ memory predicate $\MPRED$; and
%% $(ii)$ soundness predicate $sound\_states$ for $L$ that satisfies open preservation for $\MPRED$,
%% and $(iii)$ sound predicates for context languages---\textrm{Clight} and \textrm{Asm}---that
%% respectively satisfy open preservation for $\MPRED$.  Then we can use them for discharging
%% \textsc{(sim:mod)}'s condition (2) in the proof of open simulation.
%% In particular, $(iii)$ is necessary for proving that context modules are self-related.

% ; $(iii)$ open preservation of $A$ for $L$: for all $L$-module $M_\src$ and symbol environment
% $\skenv_{\src}$, we have
% $sound\_states(A(M_\src)) \in \text{open\_prsv}_{\MPRED}(\module_{\src}.\texttt{sem} \;
% \skenv_{\src})$; and $(iii)$ open preservation of $A$ for context languages, \textrm{Clight} and
% \textrm{Asm}, for suitable soundness predicates and $\MPRED$.  

%% Furthermore, while the simulation definitions accept only one memory predicate parameter, we can
%% essentially utilize multiple memory predicates (and analyses) by composing them.  Specifically, we
%% proved the following composibility lemma for memory predicates and soundness predicates:
%% \[
%%   \begin{stackTL}
%%     \forall \MPRED{}_{0}, \MPRED{}_{1} \in \textrm{MemPred},~ \exists \MPRED{}_{2} \in \textrm{MemPred},~ \forall sound\_states_{0}, sound\_states_{1}, \msem, \\
%%     \quad sound\_states_{0} \in \textrm{open\_prsv}_{\MPRED{}_{0}}(\msem) \land sound\_states_{1} \in \textrm{open\_prsv}_{\MPRED{}_{1}}(\msem)  \implies \\
%%     \quad (sound\_states_{0} \land sound\_states_{1}) \in \textrm{open\_prsv}_{\MPRED{}_{2}}(\msem)~.
%%   \end{stackTL}
%% \]
}

\subsection{Horizontal Compositionality and Adequacy}
\label{sec:main-verification:theorems}

To use open simulations in RUSC, we prove their horizontal compositionality and adequacy.  Let $P$
and $Q$ be programs (\ie lists of modules) and we define $P \llink Q$ to be the list concatenation
of $P$ and $Q$.  Let
$\MREL \in \textrm{MemRel}, \SREL \in \textrm{SymbRel}, \MPRED \in \textrm{MemPred}$ be parameters,
and $\succsim$ be the program simulation relation for the parameters,
given in \textsc{(sim:prog)} of \Cref{fig:full-sim}.  Then we have:

\begin{theorem}[HorComp]
  For any programs $P_\src$, $P_\tgt$, $Q_\src$, $Q_\tgt$, if $P_\src \succsim P_\tgt$ and
  $Q_\src \succsim Q_\tgt$:
  \[
  \vspace{-0.8mm}
  P_\src \llink Q_\src \succsim P_\tgt \llink Q_\tgt~.
  \vspace{-0.8mm}
  \]
\end{theorem}
\begin{proof} Immediate from the definition of $\llink$ and \textsc{(sim:prog)}.
\end{proof}
\begin{theorem}[Adequacy]
  For any programs $P_\src$ and $P_\tgt$, if $P_\src \succsim P_\tgt$:
  \[
  \vspace{-0.8mm}
  \beh{P_\src} \supseteq \beh{P_\tgt}~.
  \vspace{-0.8mm}
  \]
\end{theorem}
\begin{proof} By ``weaving'' module simulations as in \cite{pb}.
\end{proof}



\subsection{Instances of Parameters}
\label{sec:main-verification:instances}

In this section, we present intuition behind the three parameters (\Cref{sec:main-verification:instances}) with some interesting instances of them.
%% with their underlying ideas.
%% and try to convey the idea behind them.

The virtue of $\MREL$ is that in the \caselabel{CALL} case of \Cref{fig:full-sim},
possible future worlds are restricted with \cdashbox{myred}{$\mrel{}'' \sqsupseteq \mrel{}'$}.
Actually, all the guarantee conditions \cfbox{myred}{$\mrel' \sqsupseteq \mrel$} elsewhere are required to ensure this condition.
The most interesting instances of $\MREL$ are already presented in \Cref{sec:overview-verification:injection:dynamic,sec:overview-verification:injection:static}.
They are carefully designed so that each function guarantees others' private memories (\eg dynamic local memories) are unchanged (\cdashbox{myred}{$\mrel{}'' \sqsupseteq \mrel{}'$}), and in return they are guaranteed that their own private memories are unchanged after an external function call.

$\SREL$ was introduced to verify optimizations that changes their symbol environments, such as \code{Unusedglob}.
Except for those optimizations, we always use the following trivial quadruple.
%% \[
%% (\Skel \times \Skel, \;
%% \code{fun \_\,\_} \rightarrow \true, \;
%% \code{fun} \; \srel \; \skel_\src \; \skel_\tgt \rightarrow \srel = (\skel_\src, \skel_\tgt), \;
%% \]
%% \[
%% \code{fun \_\,\_ } \; \skenv_\src \; \skenv_\tgt \rightarrow \skenv_\src = \skenv_\tgt)
%% \]
\[
\begin{array}{l@{}l}
(&\Skel \times \Skel, \;
\code{fun \_\,\_} \rightarrow \true, \;
\code{fun} \; \srel \; \skel_\src \; \skel_\tgt \rightarrow \srel = (\skel_\src, \skel_\tgt), \\
&\code{fun \_\,\_ } \; \skenv_\src \; \skenv_\tgt \rightarrow \skenv_\src = \skenv_\tgt)
\end{array}
\]

%% $
%% (\Skel \times \Skel, \;
%% \code{fun \_\,\_} \rightarrow \true, \;
%% \code{fun} \; \srel \; \skel_\src \; \skel_\tgt \rightarrow \srel = (\skel_\src, \skel_\tgt), \;
%% \code{fun \_\,\_ } \; \skenv_\src \; \skenv_\tgt \rightarrow \skenv_\src = \skenv_\tgt)
%% $

For unusedglob, we use following instance.
\[
\begin{array}{l@{}l}
(&(\code{d}, \code{s}, \code{t}) \in \powset{\Ident} \times \Skel \times \Skel), \\
 &\code{fun (\code{d}, \code{s}, \code{t}) (\code{d'}, \code{s'}, \code{t'})} \rightarrow \\
 &\qquad \code{d} \subseteq \code{d'} \land \code{s} \sqsubseteq \code{s'} \land \code{t} \sqsubseteq \code{t'} \land (\forall i \in \Ident, i \in d' \land i \notin d \implies (i \notin \code{s} \land i \notin \code{t})), \\
 &\code{fun (d, s, t)} \; \skel_\src \; \skel_\tgt \rightarrow \\
 &\qquad \code{s} = \skel_\src \land \code{t} = \skel_\tgt \land (\forall i \in \Ident, i \notin d \implies \code{s.}i = \code{t.}i) \land \\
 &\qquad (\forall i \in \Ident, i \in d \implies i \in \code{s.statics} \land \code{t.}i = \none) \\
\end{array}
\]
%% {(t, ⊑, screl, serel) ∈ (Set × P(t × t) × (t → P(Scode × Scode)) × (t → MR.t → P(Senv × Senv))) |
