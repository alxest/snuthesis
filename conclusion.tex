\chapter{\;\;\;\;Conclusion}
\label{sec:conclusion}

We conclude by recalling the main contributions of this dissertation.
We developed a novel modular verification technique called RUSC (\Cref{sec:rusc}) and demonstrated
its usefulness by applying it to both compiler verification (\Cref{sec:compiler}) and program verification (\Cref{sec:program}).
On the one hand, RUSC is a significant step forward for compiler verification; it achieves both the flexibility of \ccx{} and the generality of \ccc{}.
We have applied RUSC to \cc{} and developed \ccm{}, a full extension of \cc{} supporting multi-language linking.
On the other hand, RUSC as a program verification technique is in its early stage but shows considerable potential.
We have applied RUSC to verify interesting examples and discussed its advantages, current limitations, and future research directions.

%% In \Cref{sec:rusc}, we discussed the problems with open simulations and presented our solution, RUSC.

%% In \Cref{sec:compiler}, we discussed the problems with interaction semantics and presented how we fixed it.
%% Also, we showed the flexibility of our framework by adding an advanced optimization to \cc{}, and reported CompCertM, a full extension of \cc{} supporting multi-language linking.
%% Finally, we fleshed out formal details of our development and discussed related works.


%% In \Cref{sec:program},


%% and demonstrated that RUSC its effectivity with
%% how it can be used to achieve an end-to-end verification.
%% In particular, RUSC is a significant step forward for modular verification of compiler (\Cref{sec:compiler}).
%% Specifically, we demonstrated how RUSC 
%% Under Self-related Contexts) (\Cref{sec:rusc}), and demonstrate its usefulness by applying it to
%% both compiler verification (\Cref{sec:compiler}) and program verification (\Cref{sec:program}).
