\chapter{\;\;\;\;Formalization of Verification Techniques}
\label{sec:main-verification}

%% \todo{slight difference in math and coq dev ?}

Now we present the formalization of our verification techniques.
%% As discussed in \Cref{sec:overview-verification:solution},
We parameterize the notion of open simulation presented in
\Cref{sec:overview-verification} with three parameters: memory relations, symbol
relations, and memory predicates.  We present the three parameters
(\Cref{sec:main-verification:parameter}), the parameterized open simulations
(\Cref{sec:main-verification:opensim}), and their horizontal compositionality and adequacy theorems
(\Cref{sec:main-verification:theorems}).
%% Then we introduce mixed simulation that allows forward
%% reasoning in the presence of nondeterminism (\Cref{sec:main-verification:mixedsim}).

\section{Parameters for Open Simulations}
\label{sec:main-verification:parameter}

\Cref{fig:simulation-parameters} presents
the sets of three parameters for open simulations:
the set of memory relations $\MREL$, the set of symbol relations $\SREL$, and the set of memory predicates $\MPRED$.
%% , which we will explain in details.
%% Note that this section will be easier to follow if read in color:
%% \textcolor{myred}{memory relations} are presented in red, \textcolor{darkgreen}{symbol relations} in
%% green, and \textcolor{myblue}{memory predicates} in blue.

\begin{figure}[t!]
\footnotesize

\begin{minipage}{1\linewidth}
  \mytitle{memory relation}\\
$
  %% \[
  \begin{stackTL}
  %% \MREL = \{
  %% (\ty, \texttt{mem}, \sqsubseteq, \sqsubseteq_\pub, \texttt{vrel}) \; | \;
  %% \ty \in \textrm{Set}, \texttt{mem} \in \ty \rightarrow (\Memory \times \Memory), \sqsubseteq, \sqsubseteq_\pub \; \in \powset{\ty \times \ty}, \texttt{vrel} \in \ty \rightarrow \powset{\Val \times \Val}
  %% %% & \ty \in \textrm{Set}, \texttt{src} \in \ty \rightarrow \Memory, \texttt{tgt} \in \ty \rightarrow \Memory, \sqsubseteq, \sqsubseteq_\pub \; \in \powset{\ty \times \ty}, \texttt{vrel} \in \ty \rightarrow \powset{\Val \times \Val}
  %% \}
  %% \\
  %% \MREL \text{ is a \emph{MemRel} if the following properties hold.}
  %% \\
  %% \\
  %% \caselabel{properties}\\
  %% {\begin{array}{l@{\;}l}
  %%   \caselabel{1} & \sqsubseteq_\pub \text{ is preorder}\\
  %%   \caselabel{2} & \sqsubseteq_\pub \subseteq \sqsubseteq\\
  %%   \caselabel{3} & \forall \mrel, \mrel',\; \mrel \sqsubseteq \mrel' \implies \texttt{vrel}(\mrel) \subseteq \texttt{vrel}(\mrel')\\
  %% \end{array}}
  %% \\
  %% %% \texttt{mrel}: \ty \rightarrow \powset{\Memory \times \Memory} \defeq \lambda \mrel, \mem_\src, \mem_\tgt, ($\MREL{}$.\texttt{src} \; \mrel) = \mem_\src \land ($\MREL{}$.\texttt{tgt} \; \mrel) = \mem_\tgt \\
  %% \caselabel{derived}\\
  %% \begin{array}{r@{\;}l}
  %% \texttt{mrel} & \defeq \lambda \mrel, \mem_\src, \mem_\tgt,\; ($\MREL{}$.\texttt{src} \; \mrel) = \mem_\src \land ($\MREL{}$.\texttt{tgt} \; \mrel) = \mem_\tgt \\
  %% \args_\src \succsim_{\mrel} \args_\tgt & \defeq (\args_\src.\texttt{f}, \args_\tgt.\texttt{f}) \in \texttt{vrel}(\mrel) \land
  %%   (\args_\src.\texttt{mem}, \args_\tgt.\textrm{mem}) \in \texttt{mrel}(\mrel) \land
  %%   (\args_\src.\texttt{vs}, \args_\tgt.\texttt{vs}) \in \overrightarrow{\texttt{vrel}(\mrel)}\\
  %% \retv_\src \succsim_{\mrel} \retv_\tgt & \defeq (\retv_\src.\texttt{v}, \retv_\tgt.\texttt{v}) \in \texttt{vrel}(\mrel) \land
  %%   (\retv_\src.\texttt{mem}, \retv_\tgt.\textrm{mem}) \in \texttt{mrel}(\mrel)\\
  %% \end{array}
  %% \\
  %% .
  %% \\
  %% .
  %% \\
  %% .
  %% \\


  \begin{array}{l@{}l}
  \MREL \in \textrm{MemRel} = \span
  \\\quad
  \setof{
    (&\ty, \sqsubseteq, \sqsubseteq_\weak, \texttt{mrel}, \texttt{vrel}) \in (\textrm{Set} \times \powset{\ty \times \ty} \times \powset{\ty \times \ty} \times (\ty \rightarrow \powset{\Memory \times \Memory}) \times (\ty \rightarrow \powset{\Val \times \Val})) \; |
  \\\quad
  & (\sqsubseteq \text{ is preorder}) \; \land \;
    (\sqsubseteq \, \subseteq \, \sqsubseteq_\weak) \; \land \;
    (\forall \mrel, \mrel',\; \mrel \sqsubseteq \mrel' \implies \texttt{vrel}(\mrel) \subseteq \texttt{vrel}(\mrel')) \; \land \;
  \\\quad
  & (\forall \mrel, i,~ (\textrm{Vint } i, v_\tgt) \in \texttt{vrel}(\mrel) \implies v_\tgt = \textrm{Vint } i)
  }
  \end{array}
  \\
  %% \texttt{mrel}: \ty \rightarrow \powset{\Memory \times \Memory} \defeq \lambda \mrel, \mem_\src, \mem_\tgt, ($\MREL{}$.\texttt{src} \; \mrel) = \mem_\src \land ($\MREL{}$.\texttt{tgt} \; \mrel) = \mem_\tgt \\
  \begin{array}{r@{\;}l}
  \args_\src \succsim_{\mrel} \args_\tgt \defeq &
    (\args_\src.\texttt{m}, \args_\tgt.\texttt{m}) \in \texttt{mrel}(\mrel) \land
    (\args_\src.\texttt{f}, \args_\tgt.\texttt{f}) \in \texttt{vrel}(\mrel) \land
    (\args_\src.\texttt{vs}, \args_\tgt.\texttt{vs}) \in \overrightarrow{\texttt{vrel}(\mrel)} \; \land \\
    & (\args_\src.\texttt{rs}, \args_\tgt.\texttt{rs}) \in \overrightarrow{\texttt{vrel}(\mrel)} \\
  \retv_\src \succsim_{\mrel} \retv_\tgt \defeq &
    (\retv_\src.\texttt{m}, \retv_\tgt.\texttt{m}) \in \texttt{mrel}(\mrel) \land
    (\retv_\src.\texttt{v}, \retv_\tgt.\texttt{v}) \in \texttt{vrel}(\mrel) \land
    (\retv_\src.\texttt{rs}, \retv_\tgt.\texttt{rs}) \in \overrightarrow{\texttt{vrel}(\mrel)} \\
  \end{array}
  \end{stackTL}
$
  %% \]
%% \youngju{\texttt{vrel} should be $\MREL{}$.\texttt{vrel}. (1) change? note that Former (and overview) is textrm but latter is texttt (2) specify $\MREL{}$ only when it is confusing (e.g. \ty)}
\\
\end{minipage}%
\\
\begin{minipage}{1\linewidth}
  \mytitle{symbol relation}\\
$
  \begin{stackTL}
  \begin{array}{@{}l@{}l@{~}l@{}}
    \SREL \in \textrm{SymbRel} = \span \span
\\\quad
\setof{&(\ty, \sqsubseteq, \simsk, \simskenv) \in
(\textrm{Set} \times \powset{\ty \times \ty} \times (\ty \rightarrow \powset{\Skel \times \Skel}) \times (\ty \rightarrow \MREL.\ty \rightarrow \powset{\Skenv \times \Skenv})) \; | \span
\\\quad
& \caselabel{1} \;&  \sqsubseteq \textrm{is preorder}
\\\quad
& \caselabel{2} \;& \forall \skel_\src, \skel'_\src, \skel''_\src, \skel_\tgt, \skel'_\tgt, \skel''_\tgt,~ \skel''_\src = \skel_\src \plink \skel'_\src \land \skel''_\tgt = \skel_\tgt \plink \skel'_\tgt \implies {}
\\\quad
&                   &
\forall \srel, \srel',~ (\skel_\src, \skel_\tgt) \in \simsk(\srel) \land (\skel'_\src \; \skel'_\tgt) \in \simsk(\srel') \implies {}
\\\quad
&                   &
\exists \srel'',~ (\skel''_\src, \skel''_\tgt) \in \simsk(\srel'') \land \srel \sqsubseteq \srel'' \land \srel' \sqsubseteq \srel''
\\\quad
& \caselabel{3} \; & \forall \skel_\src, \skel_\tgt, \srel,~ (\skel_\src, \skel_\tgt) \in \simsk(\srel) \implies
\\\quad
  &                   & \exists \mrel,~ (\textrm{load\_mem}(\skel_\src), \textrm{load\_mem}(\skel_\tgt)) \in \texttt{mrel}(w) \; \land \;
(\textrm{load\_se}(\skel_\src), \textrm{load\_se}(\skel_\tgt)) \in \simskenv(\srel, \mrel)
\\\quad
& \caselabel{4} \; & \forall \srel, \mrel, \mrel',~ \mrel \sqsubseteq_\weak \mrel' \implies \simskenv(\srel, \mrel) \subseteq \simskenv(\srel, \mrel')
\\\quad
  %% & \caselabel{5} \; (& \forall \srel, \mrel, \skenv_\src, \skenv_\tgt, \val_\src, \val_\tgt,~ (\skenv_\src, \skenv_\tgt) \in \simskenv(\srel, \mrel) \land (\val_\src, \val_\tgt) \in \MREL.\texttt{vrel}(\mrel) \implies \\
  %% &                   & \val_\src \in \textrm{ftns}(\skenv_\src) \iff \val_\tgt \in \textrm{ftns}(\skenv_\tgt))\\
& \caselabel{5} \; & \forall \srel, \mrel, \skenv_\src, \skenv_\tgt,~ (\skenv_\src, \skenv_\tgt) \in \simskenv(\srel, \mrel) \implies \skenv_\src.\texttt{pubs} = \skenv_\tgt.\texttt{pubs} \; \land \;
\\\quad
&                   & \forall (v_\src, v_\tgt) \in \MREL.\texttt{vrel}(\mrel),~ v_\src \in \textrm{ftns}(\skenv_\src) \implies v_\tgt \in \textrm{ftns}(\skenv_\tgt)
\\\quad
& \caselabel{6} \; & \forall \srel, \srel', \mrel, \skel_\src, \skel_\tgt, \skenv_\src, \skenv_\tgt,~

                     \srel \sqsubseteq \srel' \land (\skel_\src, \skel_\tgt) \in \simsk(\srel) \land (\skenv_\src, \skenv_\tgt) \in \simskenv(\srel', \mrel) \implies
\\\quad
&                   & (\skenv_\src \textbar_{\skel_\src}, \skenv_\tgt \textbar_{\skel_\tgt}) \in \simskenv(\srel, \mrel)
\\\quad
& \caselabel{7} \; & \forall \srel, \mrel, \skenv_\src, \skenv_\tgt, \args_\src, \args_\tgt,~ (\skenv_\src, \skenv_\tgt) \in \simskenv(\srel, \mrel) \land \args_\src \succsim_{\mrel{}} \args_\tgt \implies
\\\quad
  &                   & \forall e, \retv_\src,~ \textrm{external\_call} \; \skenv_\src \; \args_\src \; e \; \retv_\src \implies \exists \retv_\tgt,~ \textrm{external\_call} \; \skenv_\tgt \; \args_\tgt \; e \; \retv_\tgt \land
                        \exists \mrel' \sqsupseteq \mrel,~ \retv_\src \succsim_{\mrel'} \retv_\tgt
  }
  \end{array}
  \\
  \end{stackTL}
$
\end{minipage}%
\\
\begin{minipage}{1\linewidth}
  \mytitle{memory predicate}\\
$
  \begin{stackTL}
  \begin{array}{l@{}l}
  \MPRED \in \textrm{MemPred} = \span \\
  \quad \setof{ (&\ty, \sqsubseteq, \sqsubseteq_\weak, \texttt{mpred}, \texttt{vpred}, \texttt{sepred}) \in (\textrm{Set} \times \powset{\ty \times \ty} \times \powset{\ty \times \ty}
                                                                                                           \times (\ty \!\rightarrow\! \powset{\Memory}) 
                                                                                                           \times (\ty \!\rightarrow\! \powset{\Val}) \times (\ty \!\rightarrow\! \powset{\Skenv})) \; | \\
  & (\sqsubseteq \text{ is preorder}) \; \land \;
  (\sqsubseteq \, \subseteq \, \sqsubseteq_\weak) \; \land \;
  (\forall \mpred, \mpred',\; \mpred \sqsubseteq \mpred' \implies \texttt{vpred}(\mpred) \subseteq \texttt{vpred}(\mpred')) \; \land \; \\
  & (\text{\texttt{sepred} should satisfy the unary version of $\simskenv$'s conditions where $\SREL{}.\!\!\sqsubseteq$ and $\simsk$ are the total relations})
  }
  \end{array}
  \\
  \begin{array}{r@{\;}l}
  %% \texttt{mpred}(\mpred) & \defeq \setof{$\MPRED{}$.\texttt{mem}(\mpred)} \\

  \texttt{cpred}(\mpred) & \defeq \{ \args \in \Args \suchthat \args.\texttt{m} \in \texttt{mpred}(\mpred) \land \args.\texttt{f} \in \texttt{vpred}(\mpred) \land
                                                \args.\texttt{vs} \in \overrightarrow{\texttt{vpred}(\mpred)} \land \args.\texttt{rs} \in \overrightarrow{\texttt{vpred}(\mpred)} \} \\
  \texttt{rpred}(\mpred) & \defeq \{ \retv \in \Retv \suchthat \retv.\texttt{m} \in \texttt{mpred}(\mpred) \land \retv.\texttt{v} \in \texttt{vpred}(\mpred) \land
                                                \retv.\texttt{rs} \in \overrightarrow{\texttt{vpred}(\mpred)} \} \\
  %% \args_\src \in \texttt{cpred}(\mpred) & \defeq \args_\src.\texttt{m} \in \texttt{mpred}(\mpred) \land \args_\src.\texttt{f} \in \texttt{vpred}(\mpred) \land
  %%                                               \args_\src.\texttt{vs} \in \overrightarrow{\texttt{vpred}(\mpred)} \land \args_\src.\texttt{rs} \in \overrightarrow{\texttt{vpred}(\mpred)} \\
  %% \retv_\src \in \texttt{rpred}(\mpred) & \defeq \retv_\src.\texttt{m} \in \texttt{mpred}(\mpred) \land \retv_\src.\texttt{v} \in \texttt{vpred}(\mpred) \land
  %%                                               \retv_\src.\texttt{rs} \in \overrightarrow{\texttt{vpred}(\mpred)} \\

  \end{array}
  \end{stackTL}
$
%% \\
%% \youngju{지금 2번쨰줄 아래가 짤리는데 왜인지 모르겠습니다}\\
%% \youngju{pred 대신 prd?}\\
\end{minipage}

\caption{Three parameters for open simulations}
\label{fig:simulation-parameters}
\end{figure}


\myparagraph{Memory Relation}

The first parameter ranges over Kripke-style memory/value relations in $\MREL$.
%~\cite{DBLP:conf/popl/AhmedDR09}
Following \cite{DBLP:conf/icfp/DreyerNB10,pb}, we model the
evolution of memory relations using \emph{possible worlds} and \emph{private and public transitions}
over the worlds.
Note that this parameter will be instantiated with the three memory relations used in \cc{}---namely memory
identity, extension, and injection---and the memory injection with module-local invariants we introduced.
%% for verifying \texttt{Unreadglob}.

A memory relation in $\MREL$ consists of $(i)$ a set $\code{t}$ of possible worlds; $(ii)$ \emph{public}
and \emph{private} transition relations $\sqsubseteq$ and $\sqsubseteq_\weak$ over the worlds;
and $(iii)$ for each world $w \in \code{t}$, memory relation $\texttt{mrel}(w)$ and
value relation $\texttt{vrel}(w)$.  A world $w$ represents an invariant on the memory, which
can evolve over time according to the public/private transition relations,
as we discussed in \Cref{sec:overview-verification:injection}.
%% public transitions $\sqsubseteq$ represent possible evolution of the world before and after an external function
%% call, and private transitions $\sqsubseteq_\weak$ represent possible evolution between any two interaction points with
%% external modules. %(\eg{} between two invocations of external functions).
%% For example, we use private transitions to capture the permission changes in the arguments area of the stack,
%% which cannot be public transitions according to the \newnewrevision{enriched} memory injection.
%% For example, memory
%% injection's public transition encodes the invariant that injection should be increased and memory
%% access permissions are the same before and after a function call; but its private transition allows
%% certain changes in access permissions, as discussed in \Cref{sec:overview-semantics:solution}.  The
%% public transition relation is reflexive and transitive, and is a subset of the private transition
%% relation.
There are four natural well-formedness conditions, which are self-explanatory.
%% $\texttt{vrel}(w)$ should be monotone w.r.t. the world's private
%% transition, and it relates an integer value in source only with the same value in target.
We can also straightforwardly extend the value/memory relation to relations on $\Args$ and $\Retv$, denoted $\succsim_{\mrel}$.
%% For each
%% world $w \in \code{t}$, relations on the input/output of external function calls are defined in a
%% straightforward manner: function pointers, arguments, and results are related as values, and
%% input/output memories are related as memories.

\myparagraph{Symbol Relation}

The second parameter ranges over symbol relations in $\SREL$ that
relate information about global symbols (\eg which block each global
variable points to) in the source and target.  This parameter is
needed to verify optimizations like \code{Unusedglob},
\code{Unreadglob} that remove unnecessary static variables thereby
having non-identical symbol information in the source and target.

%% enables he verification of such
%% specifications and optimizations as \code{Unusedglob} that change symbol tables.
%% Verification of
%% all \cc{} optimizations, except for \code{unused-globs}, requires only a trivial symbol relation
%% in which the source and target symbols are exactly the same.  The main purpose of $\SREL$ is
%% guaranteeing a \emph{run-time} symbol relation, assuming all its modules satisfy \emph{compile-time}
%% symbol relation.  We allow modular reasoning of symbol relations using states and their
%% compatibility relation w.r.t. linking.  For example, the verification of \code{unused-globs} uses a
%% symbol relation whose state is the set of dropped symbols.  To model the evolution of symbol
%% relations under linking, we say two states are compatible if the former is a subset of the latter.
%% The compile-time and run-time symbol relations check whether the symbols in the state are actually
%% dropped.
%% Specifically,

A symbol relation in $\SREL$ consists of $(i)$ a set $\code{t}$ of symbol relation states; $(ii)$ 
an extension relation $\sqsubseteq$ on the states; $(iii)$ for each state $d$,
a (compile-time) symbol code relation $\texttt{screl}(d)$; and $(iv)$ for each state $d$ and world $w \in \MREL.\code{t}$,
(run-time) symbol environment relation $\texttt{serel}(d,w)$.
There are seven well-formedness conditions:
%% $\SREL$ should satisfy the following conditions:
$(1)$ the extension relation $\sqsubseteq$ is transitive and reflexive;
$(2)$ \texttt{screl} is closed under the syntactic linking;
$(3)$ if symbol codes are related by \texttt{screl}, then \newnewrevision{the initial memories and symbol environments loaded by \textrm{load\_mem} and \textrm{load\_se} are related by \texttt{mrel} and \texttt{serel}, respectively};
$(4)$ \texttt{serel} is monotone w.r.t. private transitions;
$(5)$ \newnewrevision{for symbol environments related by \texttt{serel}, their public symbols are identical and their functions have the same signatures};
$(6)$ \texttt{serel} is compatible with $\sqsubseteq$: for $d \sqsubseteq d'$, $\texttt{serel}(d')$ restricted on $\texttt{screl}(d)$ should be in $\texttt{serel}(d)$; and
%% is preserved w.r.t. compatible restriction on symbol codes; and
$(7)$ the memory and symbol relations should be compatible with \cc{}'s axiom about system calls (\ie \textrm{external\_call}).
%% external function symbols related by \texttt{serel} satisfy the condition for open simulation.

% \cdashbox{darkgreen}{$\texttt{screl}(d)$} statically holds for some $d \sqsubseteq d'$ at compile time.



% (0) \SREL{}는 크게 3가지로 instantiate 된다: identity, drop (unusedglob용), invariant (spec 증명용).
%    t 는 두 symbol code (lanugage-dependent 한 부분을 걷어낸 코드)를 relate (screl) 하는데 쓰인다.
%    identity의 경우 unit 타입이고 screl은 항상 true이다.
%    drop의 경우 set of symbol이고 screl은 (i) 실제로 그 symbol이 빠졌는지 (ii) 그 symbol을 refer하는게 없는지 체크한다. (injection 매핑에서 빼줘야 하기 때문)
%    invariant의 경우 역시 set of symbol이고 screl은 (i) 실제로 그 symbol이 invariant를 만족하는지 (ii) 그 symbol을 refer하는게 없는지 체크한다. (injection 매핑에서 빼줘야 하기 때문)
%    $\sqsubseteq$의 의미는 대충 set inclusion 이다. \\
%    이 interface의 모든 목적은 static하게 $\cfbox{darkgreen}{screl}$ 을 guarantee 해주면 runtime에 $\cdashbox{darkgreen}{serel}$ 를 rely 받는 것이다. \\

% (5) serel이면 (i) public symbol들이 같고 (ii) 어떤 relate되어있는 src/tgt value가 있을 때, src가 function이면 tgt도 function이고 둘의 signature가 같다. \\
% (6) 큰 senv가 큰 d'에 대해 serel이고, 작은 sc가 작은 d에 대해 screl이면, 각각의 큰 senv를 작은 sc로 restrict 한 것도 relate 되어있다. \\
% (7) external call axiom \\



\myparagraph{Memory Predicate}

The third parameter ranges over Kripke-style memory predicates in $\MPRED$,
which are needed to modularly verify \cc{}'s analysis engines such as value analysis (see \Cref{sec:main-verification:opensim}).
%% based on which is proved the soundness of compiler's analyses such as \cc{}'s value analysis.
$\MPRED$ is essentially a unary version of $\MREL$ combined with $\SREL$
where $\SREL{}.{\sqsubseteq}$ and $\simsk$ are taken as the total relations (\ie relating everything):
it consists of
$(i)$ the set $\code{t}$ of possible worlds;
$(ii)$ public and private transition relations $\sqsubseteq$ and $\sqsubseteq_\weak$ over the worlds, respectively; and
$(iii)$ for each world $w \in \code{t}$, a memory predicate $\textrm{mpred}(w)$,
a value predicate $\textrm{vpred}(w)$, and a symbol environment predicate $\textrm{sepred}(w)$.
The well-formedness conditions are self-explanatory.
%% Similarly to $\MREL$, $(i)$ $\MPRED$ has world's public and private transitions; $(ii)$ its value
%% predicate should be monotone w.r.t. the world's private transition; and $(iii)$ for each world
%% $w \in \code{t}$, relations on the input/output of external function calls are defined in a
%% straightforward manner.  Similarly to $\SREL$, $\MPRED$ has a few conditions on the symbol
%% environment predicate, which we omit in \Cref{fig:simulation-parameters} for brevity.

% \youngju{3, 4, 6이라고 해놓은거 다른 표현으로 바꿨습니다}
%% \youngju{
%%       \ccc{}가 개발될 당시에는 Value Analysis가 없었고, \ccx{}는 ($\MREL{}$에서와 마찬가지로) closed simulation이기 때문에 문제가 없다.
%%       \ccc{}의 후속연구인 \cascc{}에서도 Value Analysis가 들어간 pass들은 지원하지 않는다.
%%       그러니까 우리가 이 문제를 처음으로 tackle 하는}


\section{Open Simulations with Parameters}
\label{sec:main-verification:opensim}

\begin{figure}[t!]
\footnotesize

\vspace*{-20mm}
\begin{minipage}[t][0.98\textheight]{\linewidth}
\begin{minipage}{\linewidth}
\mytitle{sim:states}\\
$
  \begin{stackTL}
  match\_states \in \textrm{open\_sim}(\msem_\src, \msem_\tgt, sound\_state) \defeq \\
  \forall \mrel, \forall ((\memsrc, \stsrc), (\memtgt, \sttgt)) \in match\_states(\mrel), \cdashbox{myblue}{($\exists \mpred,~ (\memsrc, \stsrc) \in sound\_state(\mpred)$)} \implies \\
  %% \mrel, 
  {\begin{array}{l@{\;}l@{\;}l}
  & \caselabel{STEP} & \msem_{\src}.\texttt{at\_external}(\memsrc, \stsrc) = \none \land \msem_{\src}.\texttt{halted}(\memsrc, \stsrc) = \none \; \land \\
  & & \forall e, \memsrc', \stsrc',~ (\memsrc, \stsrc) \estep{e} (\memsrc',\stsrc') \implies \\[.5mm]
  & & \exists \memtgt',\sttgt',~ (\memtgt,\sttgt) \estep{\tau}^{\raisebox{-1mm}{\scriptsize$\ast$}} \estep{e}\estep{\tau}^{\raisebox{-1mm}{\scriptsize$\ast$}} (\memtgt',\sttgt') \land{}\\[.5mm]
  & & \cfbox{myred}{$\exists \mrel' \sqsupseteq \mrel$},~ ((\memsrc',\stsrc'), (\memtgt',\sttgt')) \in match\_states(\mrel') \\[1.2mm]
  %% \lor & \caselabel{CALL} & \cfbox{myred}{$\exists \args_\src, \args_\tgt, \mrel{}' \sqsupseteq_\weak \mrel{},~ \args_\src \succsim_{\mrel{}'} \args_\tgt$} \land \msem_{\src}.\texttt{at\_external}(\memsrc, \stsrc) = \some{\args_\src} \land \msem_{\tgt}.\texttt{at\_external}(\memtgt, \sttgt) = \some{\args_\tgt} \; \land \; \\
  \lor & \caselabel{CALL} & \cfbox{myred}{$\exists \mrel{}' \sqsupseteq_\weak \mrel{}$},~ \exists \args_\src, \args_\tgt,~ \cfbox{myred}{$\args_\src \succsim_{\mrel{}'} \args_\tgt$} \land \\
  & & \msem_{\src}.\texttt{at\_external}(\memsrc, \stsrc) = \some{\args_\src} \land \msem_{\tgt}.\texttt{at\_external}(\memtgt, \sttgt) = \some{\args_\tgt} \; \land \; \\
  & & \cdashbox{myred}{$\forall \mrel{}'' \sqsupseteq \mrel{}'$}, \forall \retv_\src, \retv_\tgt, \cdashbox{myred}{$\retv_\src \succsim_{\mrel{}''} \retv_\tgt$} \implies \\
  & & \forall \memsrc', \stsrc',~ \msem_{\src}.\texttt{after\_external}(\stsrc, \retv_\src) = \some{(\memsrc', \stsrc')} \implies \\[.5mm]
  & & \exists \memtgt', \sttgt',~ \msem_{\tgt}.\texttt{after\_external}(\sttgt, \retv_\tgt) = \some{(\memtgt', \sttgt')} \; \land \\
  %% & & \cfbox{myred}{$\exists \mrel''', \mrel''' \sqsupseteq \mrel \land \mrel''' \sqsupseteq_\weak \mrel'' \land (\memsrc',\memtgt') \in \mathrm{mrel}(\mrel''')$} \land \\
  & & \cfbox{myred}{$\exists \mrel''' \sqsupseteq_\weak \mrel'',~ \mrel''' \sqsupseteq \mrel$} \land ((\memsrc',\stsrc'), (\memtgt',\sttgt')) \in match\_states(\mrel''') \\[1.8mm]
  \lor & \caselabel{RET} & \cfbox{myred}{$\exists \mrel{}' \sqsupseteq \mrel{}$}, \exists \retv_\src, \retv_\tgt, \cfbox{myred}{$\retv_\src \succsim_{\mrel{}'} \retv_\tgt$} \land \\
       & & \msem_{\src}.\texttt{halted}(\memsrc, \stsrc) = \some{\retv_\src} \land \msem_{\tgt}.\texttt{halted}(\memtgt, \sttgt) = \some{\retv_\tgt} \\
  \end{array}}\\
  \end{stackTL}
$
%% \youngju{First line of (CALL) case is overflowed, we need to reduce text size somehow. I suggest to rollback $\msem$ into S}
\end{minipage}
\vspace{1mm}



\begin{minipage}{1\linewidth}
\mytitle{sim:modsem}\\
$
  \begin{stackTL}
  \msem_{\src} \succsim_{\srel{}, sound\_state} \msem_{\tgt} ~\defeq~ \exists match\_states \in \textrm{open\_sim}(\msem_{src}, \msem_{tgt}, sound\_state),\\
  \begin{array}{@{\quad}l@{\;}l@{\;}l}
  & \caselabel{INIT}
  & \forall \mrel{} \in \MREL{}.\texttt{t},~ \forall \args_\src, \args_\tgt,~ \cdashbox{myred}{$\args_\src \succsim_{\mrel{}} \args_\tgt$} \implies {} \\
  && \args_\src.\texttt{f} \in \textrm{ftns}(\msem_\src.\texttt{senv}) \land \args_\tgt.\texttt{f} \in \textrm{ftns}(\msem_\tgt.\texttt{senv})
    \implies \\
  && \cdashbox{green}{$(\msem_{src}.\texttt{senv}, \msem_{tgt}.\texttt{senv}) \in \simskenv(\srel, \mrel)$} \implies \forall (\memsrc, \stsrc) \in \msem_{\src}.\texttt{init\_core}(\args_\src),\\
  && \exists (\memtgt, \sttgt) \in \msem_{\tgt}.\texttt{init\_core}(\args_\tgt), \\
  && \cfbox{myred}{$\exists \mrel{}' \sqsupseteq \mrel{}$},~ ((\memsrc, \stsrc), (\memtgt, \sttgt)) \in match\_states(\mrel') \\
  %% & ((\memsrc, \stsrc), (\memtgt, \sttgt)) \in \textrm{match\_states}(, \mrel{}) \\
  \end{array}
  \end{stackTL}
$
\end{minipage}
\vspace{1mm}




\begin{minipage}{0.70\linewidth}
\mytitle{sim:mod}\\
$
  \begin{stackTL}
  \module_{\src} \succsim \module_{\tgt} ~\defeq~ \exists \srel{} \in \SREL{}.\texttt{t},~
  \exists sound\_state: \MPRED{}.\texttt{t} \rightarrow \powset{\Memory \times \module_\src.\texttt{state}}, \\[1mm]
  \begin{array}{l@{\;}l@{\;}l}
  %% \qquad & ~ \span \\
  & \caselabel{1} & \cfbox{green}{$(\module_{\src}.\texttt{scode}, \module_{\tgt}.\texttt{scode}) \in \simsk(\srel)$} \vspace{0.5mm}\\
  \land & \caselabel{2} & \cfbox{myblue}{$\forall \skenv_{\src},~ sound\_state \in \text{open\_prsv}(\module_{\src}.\texttt{sem} \; \skenv_{\src})$} \vspace{0.5mm}\\
  \land & \caselabel{3} & \forall \srel' \sqsupseteq \srel,~ \forall \mrel,~ \cdashbox{green}{$\forall (\skenv_\src, \skenv_\tgt) \in \simskenv(\srel', \mrel)$},~ \\
        &               & \module_{\src}.\texttt{sem}\,(\skenv_{\src}) \succsim_{\srel{}, sound\_state} \module_{\tgt}.\texttt{sem}\,(\skenv_{\tgt}) \\
  \end{array}
  \end{stackTL}
$
\end{minipage}%
\begin{minipage}{0.30\linewidth}
\mbox{}\\[14mm]
\mytitle{sim:prog}\\
$
  \begin{stackTL}
  \Prg_{\src} \succsim \Prg_{\tgt} \defeq \\[1mm]
  \quad \forall i \in \mathbb{N},~ \Prg_\src[i] \succsim \Prg_\tgt[i] \\
  \end{stackTL}
$
\end{minipage}


\vspace{1mm}
\begin{minipage}{1\linewidth}
\mytitle{preservation}\\
$
\begin{stackTL}
sound\_state \in \textrm{open\_prsv}(\msem_\src) \defeq \\

\begin{array}{l@{\;}l@{\;}l@{}l}

& \caselabel{INIT} & \forall \mpred \in \MPRED.\ty, \cdashbox{blue}{$\forall \args_\src \in \texttt{cpred}(\mpred)$},~
   \cdashbox{blue}{$\msem_\src.\texttt{senv} \in \texttt{sepred}(\mpred)$} \implies \forall (\memsrc, \stsrc) \in \msem_\src.\texttt{init\_core}(\args_\src), \span \\
& & \cfbox{blue}{$\exists \mpred{}' \sqsupseteq \mpred{}$},~ (\memsrc, \stsrc) \in sound\_state(\mpred') \span \\[1mm]

\land & \caselabel{STEP} & \forall \mpred,~ \forall (\memsrc, \stsrc) \in sound\_state(\mpred), \forall e, \memsrc', \stsrc',~ (\memsrc, \stsrc) \estep{e} (\memsrc', \stsrc') \implies \span \\
& & \cfbox{blue}{$\exists \mpred{}' \sqsupseteq \mpred{}$},~ (\memsrc', \stsrc') \in sound\_state(\mpred') \span \\[1mm]

\land & \caselabel{CALL} & \forall \mpred,~ \forall (\memsrc, \stsrc) \in sound\_state(\mpred), \forall \args_\src,~ \msem_\src.\texttt{at\_external}(\memsrc, \stsrc) = \some{\args_\src} \implies \span \\
& & \cfbox{blue}{$\exists \mpred{}' \sqsupseteq_\weak \mpred{}$},~ \cfbox{blue}{$\args_\src \in \texttt{cpred}(\mpred')$} \; \land \; \span \vspace{0.5mm}\\
& & & \cdashbox{blue}{$\forall \mpred'' \sqsupseteq \mpred{}',~ \forall \retv_\src \in \texttt{rpred}(\mpred'')$},~ \forall \memsrc', \stsrc',~
       \msem_\src.\texttt{after\_external}(\stsrc, \retv_\src) = \some{(\memsrc', \stsrc')} \implies \vspace{0.5mm}\\
& & & \cfbox{blue}{$\exists \mpred''' \sqsupseteq_\weak \mpred'',~ \mpred''' \sqsupseteq \mpred \land \exists \memsrc' \in \texttt{mpred}(\mpred''')$} \land (\memsrc', \stsrc') \in sound\_state(\mpred''')\\[1.5mm]

\land & \caselabel{RET} & \forall \mpred,~ \forall (\memsrc, \stsrc) \in sound\_state(\mpred), \forall \retv_\src,~ \msem_\src.\texttt{halted}(\memsrc, \stsrc) = \some{\retv_\src} \implies \span \\
& & \cfbox{blue}{$\exists \mpred{}' \sqsupseteq \mpred{}$},~ \cfbox{blue}{$\retv_\src \in \texttt{rpred}(\mpred')$} \span \\
\end{array}\\
\end{stackTL}
$
%% \youngju{open sim이랑 open prsv랑 function application 괄호 방식 다름}
\end{minipage}
\vspace*{3mm}
\caption{Parameterized Open Simulations}
\label{fig:full-sim}
\end{minipage}
\\
\\
\\
\\
\\
\end{figure}

%% \youngju{\cfbox{myred}{$\mrel''' \sqsupseteq \mrel''$} it appears in guarantee but not in rely. Remove fbox?}
%% \\
%% \youngju{Need tight box: vspace between two ``after\_external'' is too short}
%% \\
%% \youngju{Function application a(b) and (a b) is mixed. -- vrel(a) and (sc\_compat a b).}
%% \\
%% \youngju{$\msem_\src.\skenv$ is not very readable}
%% \\
%% \youngju{sound\_state is too long}
%% \\
%% \youngju{Need to add analysis developer's interface here}


% \jeehoon{explain rely-guarantee?}

%% TODO: Fig.11에 STEP 케이스가 forward 뿐인데, backward 케이스는 생략되었음을 언급

\Cref{fig:full-sim} presents our parameterized open simulations, which
are given in the form of forward simulation for simplicity though
they are actually in the form of mixed simulation presented in
\Cref{sec:overview-verification:mixedsim}.
In this section, we omit $\MREL$,
$\SREL$, and $\MPRED$ when clear from context (\eg{} $\texttt{vrel}(w)$ for
$\MREL{}.\texttt{vrel}(w)$).  Also, $\cdashbox{black}{R}$ and $\cfbox{black}{G}$ means rely and
guarantee conditions for the external modules.
%% whose edge color indicates which parameters are
%% involved in the rely/guarantee reasoning.

\myparagraph{Simulation of Machine States}

%% \setlist[description]{font=\normalfont\textbullet\space}

A relation $match\_states$ on machine states is an \emph{open simulation} if all related states
either $(i)$ transition to related states, $(ii)$ invoke related external calls (hence the name
``open'' simulation), or $(iii)$ halt with related return values and memories.  Specifically, given
source and target module semantics $\msem_\src$, $\msem_\tgt$ and a (source) soundness predicate $sound\_state$ (discussed later),
%% relations $match\_states(w)$ for each possible world $w \in \MREL.\code{t}$
the relation $match\_states$ over worlds is an open simulation if
the relatedness of $\mssrc$ and $\mstgt$ at a world $w$ with the soundness of $\mssrc$ implies
one of the followings.
\begin{itemize}
\item \caselabel{STEP} The source and target states transition to related states.
  Specifically:
  \begin{itemize}[leftmargin=11mm]
  \item[\textbf{line 1:}] the source machine state takes intramodule steps, and
  \item[\textbf{line 2:}] if the source machine state transitions to a next state emitting an event $e$,
  \item[\textbf{line 3:}] then the target machine state is able to transition to a next state emitting the same
    event $e$, possibly with additional silent transitions, and
  \item[\textbf{line 4:}] the next states are related by $match\_states(w')$ for a public future world $w' \sqsupseteq w$.
  %% \item[\textbf{line 3:}] the
  %%   next source and target memories are related at a public future world $w' \sqsupseteq w$,
  %% \item[\textbf{line 4:}] and the next states are related by $match\_states(w')$.
  \end{itemize}
  \vskip 1mm
\item \caselabel{CALL} The source and target states invoke related external calls.
  Specifically:
  \begin{itemize}[leftmargin=11mm]
  \item[\textbf{line 1:}] certain external functions and arguments in the source and target are related at a private future world $w' \sqsupseteq_\weak w$, and
  \item[\textbf{line 2:}] the source and target machine states invoke the related external functions with the related arguments, and
  \item[\textbf{line 3:}] for any return values and memories related at any public future world $w'' \sqsupseteq w'$,
  \item[\textbf{line 4:}] if the source safely returns from the external call,
  \item[\textbf{line 5:}] then the target also safely returns from the external call, and
  \item[\textbf{line 6:}] the states after return are related by $match\_states(w''')$ for a world $w'''$ that is a private future of $w''$ and a public future of $w$.

  %% \item[\textbf{line 1:}] The source and target machine states are about to invoke external calls, whose
  %%   arguments are related at a private future world $w' \sqsupseteq_\weak w$,
  %% \item[\textbf{line 2:}] for any public future world $w'' \sqsupseteq w'$ and return values and memories related at $w''$,
  %% \item[\textbf{line 3:}] if the source safely returns from the external call,
  %% \item[\textbf{line 4:}] then the target also safely returns from the external call,
  %% \item[\textbf{line 5:}] and there exists a world $w'''$ that $(i)$ is a public future of $w$,
  %%   %% and $(ii)$ a private future of $w''$,
  %%   and $(ii)$ relates the next memories at $w'''$,
  %% \item[\textbf{line 6:}] and the next states are related by $match\_states(w''')$.
  \end{itemize}
  \vskip 1mm
\item \caselabel{RET} The source and target states halt with related values and memories.
  Specifically:
  \begin{itemize}[leftmargin=11mm]
  \item[\textbf{line 1:}] with return values and memories related at~$w'$
    for a public future world $w' \sqsupseteq w$,
  \item[\textbf{line 2:}] the source and target machine states halt.
  \end{itemize}
\end{itemize}


\myparagraph{Simulation of Module Semantics}

Module semantics are related if their initial machine states are related.
Specifically, for a symbol relation $d \in \SREL$ and a (source) soundness predicate $sound\_state$,
a target module semantics $\msem_\tgt$ simulates a source one $\msem_\src$ if for an open simulation $match\_states$:
\begin{itemize}
\item \caselabel{INIT} the initial machine states of $\msem_\src$ and $\msem_\tgt$ are related by $match\_states$.
Specifically: 
\begin{itemize}[leftmargin=11mm]
\item[\textbf{line 1:}] for any source and target call data related at any world $w \in \MREL$,
\item[\textbf{line 2:}] if the functions of the source and target call data belong to the modules and
\item[\textbf{line 3:}] the symbol environments are related at $d$ and $w$, then for any initial machine state of the source function call,
\item[\textbf{line 4:}] there exists an initial machine state of the target function call such that
\item[\textbf{line 5:}] the two initial machine states are related by $match\_states(w')$
  for $w'$ a public future of~$w$.
%\item[\textbf{line 3:}] if source and target symbol environments are related at $d$ and $w$,
%\item[\textbf{line 4:}] then for all possible source initial machine state,
%\item[\textbf{line 5:}] there exists a target initial machine state such that,
%\item[\textbf{line 6:}] the initial memories are related at a public future $w'$ of the current world $w$,
%\item[\textbf{line 7:}] and the initial machine states are again related by $match\_states(w')$.
\end{itemize}
\end{itemize}


\myparagraph{Simulation of Modules}
Modules are related if their module semantics are related. Specifically,
a target module $\module_\tgt$ simulates a source one $\module_\src$
if the following hold for a symbol relation $d \in \SREL$ and a soundness predicate $sound\_state$:
\begin{itemize}[leftmargin=11mm]
\item[\textbf{line 1:}] the source and target symbol codes are related at $d$,
\item[\textbf{line 2:}] $sound\_state$ satisfies the open preservation property (discussed below), and
\item[\textbf{line 3:}] for any symbol environments related at any symbol relation $d'$ extending $d$ and any world~$w$,
\item[\textbf{line 4:}] the source and target module semantics for the related symbol environments are related at $d$ and $w$.
\end{itemize}
Note that the symbol environments are related at $d'$, which represents the possible symbol information after linking with an arbitrary module,
while the module semantics are related at $d$, which represents the module's own symbol information.

\myparagraph{Simulation of Programs}
\newnewrevision{Two programs each of which consists of a list of modules are simulated if each corresponding modules are simulated.}

\myparagraph{Open Preservation with Parameters}
{\newnewrevisioncmd
\cc{} uses a relation $match\_states$ to prove correctness of a translation pass
and a predicate $sound\_state$ to prove correctness of the analyzer performing value analysis,
where $sound\_state$ specifies those states where the analysis results hold.
As we do for $match\_states$,
we perform a similar generalization from a closed setting to an open setting for $sound\_state$.
Specifically, 
we generalize the conditions for $sound\_state$ from preservation to open preservation
(\cf from simulation to open simulation);
and parameterize over memory predicates $\MPRED{}$ (\cf memory relations $\MREL{}$),
which intuitively encodes the analysis results of the analyzer.
Also, as we do for open simulation,
we prove that all \textrm{Clight} and \textrm{Asm} modules satisfy
open preservation with $\MPRED{}$, which intuitively means that
all those context modules preserve the analysis results of the analyzer.
Note that the definition of open preservation, \text{open\_prsv}, is essentially a unary version of
that of open simulation, where the \caselabel{INIT} case corresponds
to that of the module semantics simulation and the \caselabel{STEP},
\caselabel{CALL}, and \caselabel{RET} cases to those of the state simulation.

% For brevity, we omit a detailed explanation of the conditions.

%% Open simulation permits modular verification of compiler analyses and optimizations.
%% analyzer L sound-state open preservation
%% exists sound-state, ... Clight, Asm

%% compiler verification uses sound-state given as a parameter to open simulation, so that the verifier can prove optimizations assuming that the sound-predicate holds.
%% Analyzer prove, varifier uses.

%% >  the (source) soundness predicate `sound_state` [l. 923]
%% What is this predicate?
%% >> First, compiler passes are developed in a modular way so that optimization passes can "query" analysis passes, and rely on the analysis result without any extra obligation.
%% >> Such modular nature is reflected in the proof structure too.
%% >> In CompCert, sound_state quantifies the states that are congruent with analysis result (e.g., if analysis concluded that global variable "x" is always positive, sound_state quantifies only those states).
%% >> Then, the obligation for optimization passes is to establish "simulation" assuming "sound_state", while the obligation for analysis passes obligation is to establish "preservation" of "sound_state".
%% >> We extended such notions into an open setting, where "simulation" resulted in "open simulation" and "preservation" resulted in "open preservation."
%% >> We will add these explanations in our revision.

%% Maybe: In particular, $(iii)$ is necessary for proving that context modules are self-related.

%% For this, we
%% require the analyzer for a language $L$ to be equipped with $(i)$ memory predicate $\MPRED$; and
%% $(ii)$ soundness predicate $sound\_states$ for $L$ that satisfies open preservation for $\MPRED$,
%% and $(iii)$ sound predicates for context languages---\textrm{Clight} and \textrm{Asm}---that
%% respectively satisfy open preservation for $\MPRED$.  Then we can use them for discharging
%% \textsc{(sim:mod)}'s condition (2) in the proof of open simulation.
%% In particular, $(iii)$ is necessary for proving that context modules are self-related.

% ; $(iii)$ open preservation of $A$ for $L$: for all $L$-module $M_\src$ and symbol environment
% $\skenv_{\src}$, we have
% $sound\_states(A(M_\src)) \in \text{open\_prsv}_{\MPRED}(\module_{\src}.\texttt{sem} \;
% \skenv_{\src})$; and $(iii)$ open preservation of $A$ for context languages, \textrm{Clight} and
% \textrm{Asm}, for suitable soundness predicates and $\MPRED$.  

%% Furthermore, while the simulation definitions accept only one memory predicate parameter, we can
%% essentially utilize multiple memory predicates (and analyses) by composing them.  Specifically, we
%% proved the following composibility lemma for memory predicates and soundness predicates:
%% \[
%%   \begin{stackTL}
%%     \forall \MPRED{}_{0}, \MPRED{}_{1} \in \textrm{MemPred},~ \exists \MPRED{}_{2} \in \textrm{MemPred},~ \forall sound\_states_{0}, sound\_states_{1}, \msem, \\
%%     \quad sound\_states_{0} \in \textrm{open\_prsv}_{\MPRED{}_{0}}(\msem) \land sound\_states_{1} \in \textrm{open\_prsv}_{\MPRED{}_{1}}(\msem)  \implies \\
%%     \quad (sound\_states_{0} \land sound\_states_{1}) \in \textrm{open\_prsv}_{\MPRED{}_{2}}(\msem)~.
%%   \end{stackTL}
%% \]
}

\section{Horizontal Compositionality and Adequacy}
\label{sec:main-verification:theorems}

To use open simulations in RUSC, we prove their horizontal compositionality and adequacy.  Let $P$
and $Q$ be programs (\ie lists of modules) and we define $P \llink Q$ to be the list concatenation
of $P$ and $Q$.  Let
$\MREL \in \textrm{MemRel}, \SREL \in \textrm{SymbRel}, \MPRED \in \textrm{MemPred}$ be parameters,
and $\succsim$ be the program simulation relation for the parameters,
given in \textsc{(sim:prog)} of \Cref{fig:full-sim}.  Then we have:

\begin{theorem}[HorComp]
  For any programs $P_\src$, $P_\tgt$, $Q_\src$, $Q_\tgt$, if $P_\src \succsim P_\tgt$ and
  $Q_\src \succsim Q_\tgt$:
  \[
  \vspace{-0.8mm}
  P_\src \llink Q_\src \succsim P_\tgt \llink Q_\tgt~.
  \vspace{-0.8mm}
  \]
\end{theorem}
%% \begin{proof} Immediate from the definition of $\llink$ and \textsc{(sim:prog)}.
%% \end{proof}
\begin{theorem}[Adequacy]
  For any programs $P_\src$ and $P_\tgt$, if $P_\src \succsim P_\tgt$:
  \[
  \vspace{-0.8mm}
  \beh{P_\src} \supseteq \beh{P_\tgt}~.
  \vspace{-0.8mm}
  \]
\end{theorem}
%% \begin{proof} By ``weaving'' module simulations as in \cite{pb}.
%% \end{proof}

% Our parameterized open simulations satisfy two conditions that are required by RUSC: horizontal
% compositionality and adequacy.  Formally speaking, for any
% $\MREL \in \textrm{MemRel}, \SREL \in \textrm{SymbRel}, \MPRED \in \textrm{MemPred},~ \succsim$
% (the one in SIM:PROG) is horizontally compositional and adequate.  In terms of linking algebra,
% our linking operator is merely a list append operation and horizontal compositionality holds
% trivially by the definition of (SIM:PROG).  The interesting part is adequacy.  adequacy는 간단히
% 말하면 rely/guarantee condition이 \emph{아구가 잘 맞는다}는 뜻이다(이라고 볼 수 있다).  It is
% similar with overview so we omit details.



%% 우리 framework의 포인트는, closed simulation에서와 마찬가지로, preservation 증명과 simulation 증명이 \emph{완전히} 독립적이라는 것이다.
%% 구체적으로, (i) 분석을 한번 증명하면 n개의 pass에서 공짜로 쓸 수 있고 (ii) 한 pass에서 분석 n개를 쓸 수도 있다.
%% 분석기 개발자는 sound\_state를 정의하고 이것의 open\_preservation을 증명하면 되고 최적화 개발자는 해당 sound\_state를 아무런 obligation 없이 가져다 쓸 수 있다.

%% 분석기 개발자의 obligation은 SIM:MOD의 (2)번과 같다. 구체적으로: \\

%% $
%% (i) \text{현재 상태가 분석한 결과와 consistent 함을 나타내는 predicate인 sound\_state를 정의한다.}
%% \\
%% (ii) \cfbox{myblue}{$\forall \skenv_{\src},~ sound\_state \in \text{open\_prsv}(\module_{\src}.\texttt{sem} \; \skenv_{\src})$} \\
%% $

% \youngju{$\land$ things: }
% 이렇게 (static 분석을 지원하기 위해) $\MPRED{}$의 instance가 여러개로 늘어났다고 하자. 그리고 한 pass에서 서로 다른 $\MPRED{}$를 써야하는 분석 두개가 있다고 한다면, 이를 어떻게 지원할 것인가?
% 이를 위해 metatheory에서 다음을 증명하였다.


% \begin{enumerate}
% \item mixed sim 정의
% \item determinate/receptive (Pilsner와 비교)
% \item self stuttering
% \end{enumerate}

% As discussed in the \Cref{sec:background-verification}, instead of establishing backward simulation
% directly, \cc{} (mostly) establishes the exact dual of it: forward simulation.
% \[
% \forall (\stsrc{}, \sttgt{}) \in R,~ \forall e, \stsrc',~ \stsrc{} \estep{e} \stsrc' \Rightarrow \exists \sttgt',~  \sttgt{} \step^\ast \estep{e}\step^\ast \sttgt' \land (\stsrc', \sttgt') \in R~.
% \]
% For complier translations, forward simulation is usually much easier
% to establish than backward simulation. This is because a single
% instruction in the source is usually compiled down to multiple
% instructions in the target, and forward simulation naturally allows
% reasoning of such translation (it suffices to, given single source
% step, find multiple target steps) while backward does not.

% However, as forward simulation is the dual of backward simulation, it
% also implies the dual of behavioral refinement: \eg if $P$ and $Q$ are
% related by forward simulation, then $\beh{P} \subseteq \beh{Q}$. Thus,
% in order to use forward simulation, the target language should be
% \emph{deterministic} --- only a single transition is possible from
% each state --- which is a sufficient condition for forward simulation
% to imply behavioral refinement. Roughly speaking, a deterministic
% language has only one observable behavior, so $\beh{P} = \beh{Q}$
% holds.

% To this end, \cc{}'s languages (except for the source, C) are
% designed to be deterministic.

% \hide{

%   deterministic --> intuition 설명이 쉬움 (beh singleton). flip이야기가 나올 필요가 없음. mixed 설명이 어려움
%   receptive/determinate --> intuition 설명이 힘듦 (flipping). mixed 설명이 쉬움

% %% In order to establish behavioral
% %% refinement, forward simulation additionally requires target language
% %% to be deterministic.  which is a sufficient condition to for forward
% %% simulation to imply backward simulation (or behavioral refinement).

% \footnote{Technically, \cc{} uses a weaker
%   condition, namely that the source language is \emph{receptive} and
%   the target is \emph{determinate}.}

% Nevertheless, this seemingly inadequate relation can be ``flipped''
% into backward simulation if certain condition is satisfied: the target
% language should be deterministic, meaning that there is at most one
% step from each state.
% }

% Nevertheless, requiring a language to be deterministic is rather a harsh condition, since
% nondeterminism is an essential semantic ingredient to explain some ``textbook'' optimizations
% (\eg global value numbering) \cite{lee:freeze}, concurrent behaviors, integer to pointer castings
% \cite{TODO}, etc. Non-determinism also plays a central role in properly modeling interaction
% between C and assembly (which we will discuss in \Cref{sec:overview-semantics}), so we have
% introduced it into interaction semantics.


% \hide{
% The nondeterminism we introduced in \Cref{sec:overview-semantics} can
% potentially cause significant overhead: without determinism, we cannot
% ``flip'' a forward simulation into a backward simulation thus we might
% need to re-prove each pass using a backward simulation.
% }

\hide{
  minki's sentences
  
The intuition of forward simulation is that, since the
target state has at most one possible next execution, if there exists
next target states related with the next source state by R, then the
"all" next possible target state is related with the next source
state. However, this intuition can apply not only for when the whole
target language is deterministic, but also for when the current taget
state is deterministic.
We do not flip a "whole" forward simulation into
backward simulation, but we flip a "step" when the current target
state is locally deterministic - having at most one posstble next
state.
  }

% \myparagraph{Our Solution}


% %% we allow forward reasoning when current target \emph{state} is deterministic.
% We address the problem with our new proof technique called
% \textit{mixed simulation}, which allows forward reasoning (locally)
% even when the target language (as a whole) is not deterministic. The
% key idea is to exploit \textit{local determinism}: for each step,
% \xsim{} allows a user to decide which simulation technique to use
% (either forward or backward) where forward is only allowed when the
% current target \textit{state} is deterministic. \Cref{fig:mixedsim}
% illustrates mixed simualtion.  \todo{erase?: To put it another way,
% while vanilla \fsim{} requires determinism of the \textit{whole
%   language} to allow forward reasoning for \textit{whole proof},
% \xsim{} requires determinism of a \textit{state} and allows forward
% reasoning for a \textit{step}.} As a proof technique, mixed simulation
% subsumes both forward and backward simulation.

% \todo{I said target ``state'' is deterministic, but in figure there are multiple states}

% \vspace{3mm}

% $\forall (\mssrc, \mstgt) \in R$

% $(\forall e, \mssrc',~ \mssrc \estep{e} \mssrc' \implies {} \exists \mstgt',~
% \mstgt \dstep{\tau}^{\raisebox{-1mm}{\scriptsize$\ast$}} \dstep{e}\dstep{\tau}^{\raisebox{-1mm}{\scriptsize$\ast$}} \mstgt'
% \land (\mssrc', \mstgt') \in R)~ \lor$

% $(\forall e, \mstgt',~ \mstgt \estep{e} \mstgt' \implies {} \exists \mssrc',~
% \mssrc \estep{\tau}^{\raisebox{-1mm}{\scriptsize$\ast$}} \estep{e}\estep{\tau}^{\raisebox{-1mm}{\scriptsize$\ast$}} \mssrc'
% \land (\mssrc', \mstgt') \in R)~ .$

% where
% $\stt{} \dstep{e} \stt{}' \triangleq \stt{} \oset{e}{\rightarrow} \stt{}' \; \wedge \; $ deterministic\_at($\stt{}$)

% \todo{just one: figure or formula?}

% \todo{(1) choose better 'deterministic step' notation (2) synch
%   notations of fig and formula }

% Mixed simulation implies behavioral refinement. Intuitively, mixed
% simulaiton is made by ``stitching'' forward and backward simulations.
% In the viewpoint of each fragmented forward simulations, whole
% language is deterministic, and it is sufficient to imply behavioral
% refinement for each fragment, which in sum establishes behavioral
% refinement.



% \todo{mixed sim is a unifying theory: (1) backward (C -> Clight), forward (others)
% --> 사실 open backward만 있어도 됨}

%% \begin{definition}\label{def:xsim} (Mixed Simulation)

%%   $ \psrc{} \succsim_{x} \ptgt{} \triangleq \exists \; R: \src{}.(state) \rightarrow \tgt{}.(state) \rightarrow Prop $ such that

%%     1) $ (load(\psrc{}), \; load(\ptgt{})) \in R \; $

%%     2) For any $ (\stsrc{}, \, \sttgt{}) \in R $, one of the following conditions hold

%%       2-a) For any $ \sttgt{} \oset{e}{\rightarrow} \sttgt{}' $, there exists $ \stsrc{}' $ s.t. $ \stsrc{} \oset{e}{\rightarrow}^{+} \stsrc{}' $ and $ (\stsrc{}', \; \sttgt{}') \in R $

%%       2-b) For any $ \stsrc{} \oset{e}{\rightarrow} \stsrc{}' $, there exists $ \sttgt{}' $ s.t. $ st_{tgt} \oset{e}{\twoheadrightarrow}^{+} \sttgt{}' $ and $ (\stsrc{}', \; \sttgt{}') \in R $

%% where
%% $\stt{} \oset{e}{\twoheadrightarrow} \stt{}' \triangleq \stt{} \oset{e}{\rightarrow} \stt{}' \; \wedge \; $ deterministic\_at($\stt{}$)
%% \end{definition}




\hide{
To establish such a backward simulation for each pass, \cc{} uses two
different techniques: one is unrestricted but harder to use, and the
other is easier but restricted to use. Specifically, the former is to
establish a backward simulation directly, and the latter is to
establish a forward simulation, which is the reverse of a backward
simulation (\ie the target simulates the source).  A forward
simulation is much easier to establish because a compiler pass
typically compiles a single instruction in the source down to several
instructions in the target. However, a forward simulation is
restricted to use because it implies a backward simulation (or
behavioral refinement) only when the target language is
deterministic\footnote{Technically, \cc{} uses a weaker condition, namely that
  the source language is \emph{receptive} and the target is \emph{determinate}.},
and even then it may be too strong to establish.

In fact, \cc{} uses a forward simulation for all the passes except for
the first one, for which a forward simulation is too strong to
establish.
}

%% -----

%% \jeehoon{It seems overview sections make many forward references. Make sure they are all treated in
%%   the main sections.}

%% \youngju{\cc{}의 최적화들은 서로 다른 relation으로 증명되어있는데, 우리의 open simulation은 이를 모두 포함하도록 정의했다.}
%% \youngju{그러는 motivation 설명이 있어야 함. 그래서 개발 양이 적다는 이야기를 앞에서 or 여기서 or later (results)}
%% \youngju{\ccx{}는 하나의 big relation이 필요했다는 이야기 어디선가 언급: overview-RUSC?}
%% \youngju{왜 3개씩이나 쓰냐? \cc{}가 ill-design된거고 애초에 princeton처럼 하나 쓰면 되는거 아니냐? 라는 궁금증에 대해 어느 정도 답을 줘야?}
%% \youngju{
%%       우리의 open simulation 또한 global environment 사이 서술을 해야 하는데, 이것을 Unusedglob을 지원하도록 정의하면, 다른 최적화들의 simulation relation을 사용하지 못한다.
%%       내 모듈의 최적화에서 symbol을 건드리지 않았다고 하더라도, 링킹하는 다른 모듈이 symbol을 건드렸다면 (unusedglob으로 최적화를 했다면), 내 mapping도 identity가 아니게 된다.
%%       %% 예를 들어 내 모듈은 { c } 를 { c } 그대로 내버려뒀고, 링킹하는 다른 모듈은 { a, b }를 { a }로 최적화했다고 하면, mapping은 오른쪽 그림처럼 될 것이고 c 사이의 mapping도 identity가 아니게 된다.
%%       \ccc{}와 \ccx{}가 개발될 당시에는 Unusedglob이 없었고, 각각의 후속 연구들 (\cascc{}, \saccx{})이 개발될 당시에는 Unusedglob이 있었지만 지원하지 않는다.
%%       이들의 simulation relation에는 global environment mapping이 identitiy로 baked in 되어있다.
%%       이 문제를 tackle하는건 우리가 처음이다.
%% }

\hide{
%% \begin{enumerate}
%% \item open simulation\\
  %% 앞서 이야기했듯이 우리는 RUSC를 만족하는 set of relation --- open simulation --- 을 정의했다.
  %% \cc{}의 최적화들은 서로 다른 relation으로 증명되어있는데, 우리의 open simulation은 이를 모두 포함하도록 정의했다. \todo{그러는 motivation 설명이 있어야 함. 그래서 개발 양이 적다는 이야기를 앞에서 or 여기서}
  %% \cc{}의 relation들은 크게 3가지 방향에서 다르고, 따라서 3가지 방향으로 parameterize 시켰다: $\MREL{}, \SREL{}, \MPRED{}$.
  %% 먼저 Context (\cc{}의 relation들이 어떻게 다른지) Gap (이를 어떻게 open sim으로 옮길지) 이야기하고, 우리의 정의를 이야기하겠다. \\
  %% \textbf{Problems (Context+Gap)} \\

  %% \begin{enumerate}
  %%   \item memory relation이 다르다.\\
  %%     \youngju{\ccx{}는 하나의 big relation이 필요했다는 이야기 어디선가 언급: overview-RUSC?}\\
  %%     \myparagraph{Context}
  %%     identity, extension, injection 3개가 있고 순서대로 복잡해지는데, 각각의 최적화는 simplest but sufficient relation을 쓴다.
  %%     \youngju{왜 3개씩이나 쓰냐? \cc{}가 ill-design된거고 애초에 princeton처럼 하나 쓰면 되는거 아니냐? 라는 궁금증에 대해 어느 정도 답을 줘야 함.}
  %%     injection이 가장 복잡하고 가장 강력하다.
  %%     extension은 injection과 달리 auxiliary data를 들고 다니지 않아서 더 간단한 대신 더 약하다.
  %%     identity는 lebiniz equality라 무척 편하다.

    %% \item Unusedglob \\
      %% \myparagraph{Context}
      %% 일반적인 최적화들은 symbol을 건드리지 않는다.
      %% 그러면 src/tgt global environment 사이 서술이 간단하다: 그냥 identity mapping을 사용하면 되기 때문이다. (\Cref{fig:symbrel}-왼쪽)
      %% 그런데 Unusedglob 최적화는 안쓰이는 symbol을 지우기 때문에 이를 깨뜨린다 \Cref{fig:symbrel}.

      %% \myparagraph{Gap}
      %% 우리의 open simulation 또한 global environment 사이 서술을 해야 하는데, 이것을 Unusedglob을 지원하도록 정의하면, 다른 최적화들의 simulation relation을 사용하지 못한다.
      %% 내 모듈의 최적화에서 symbol을 건드리지 않았다고 하더라도, 링킹하는 다른 모듈이 symbol을 건드렸다면 (unusedglob으로 최적화를 했다면), 내 mapping도 identity가 아니게 된다.
      %% 예를 들어 내 모듈은 { c } 를 { c } 그대로 내버려뒀고, 링킹하는 다른 모듈은 { a, b }를 { a }로 최적화했다고 하면, mapping은 오른쪽 그림처럼 될 것이고 c 사이의 mapping도 identity가 아니게 된다.

      %% \ccc{}와 \ccx{}가 개발될 당시에는 Unusedglob이 없었고, 각각의 후속 연구들 (\cascc{}, \saccx{})이 개발될 당시에는 Unusedglob이 있었지만 지원하지 않는다.
      %% 이들의 simulation relation에는 global environment mapping이 identitiy로 baked in 되어있다.
      %% 이 문제를 tackle하는건 우리가 처음이다.

    %% \item ValueAnalysis \\

    %%   \myparagraph{Context}

%% \cc{}'s optimizations often make use of static analyses (\eg value and
%% alias analysis). The static analyzers are independent of
%% optimizations, so that an analyzer can be shared among multiple
%% optimizations and an optimization can use multiple analyzers. The
%% following example shows how an optimization pass and an analyzer
%% cooperates. The constant propagation pass will query value analysis,
%% asking the abstract value of each variables for each line. Then, the
%% analysis result will say, ``x must be 0 at line 3''. Accordingly, the
%% pass will optimize as follows \emph{assuming} the analysis result is
%% correct.

%% %
%% \[
%% \begin{minipage}{0.5\textwidth}
%% \begin{verbatim}
%% 1: int x = 0;
%% 2: f();
%% 3: out(x); // optimizes to "out(0)"
%% \end{verbatim}
%% \end{minipage}
%% \]
%% %

%% \cc{} maintains such modular structure for verification too: the
%% correctness of an analysis is verified separately, and for each
%% optimization using such analysis, its simulation is established
%% \emph{assuming} the analysis result is correct. Formally, the
%% correctness of an analysis is expressed as \emph{preservation} of
%% \emph{sound state}, where $\text{sound-state } P \text{ } s$ basically
%% means the state $s$ is consistent with the analysis result of a
%% program $P$.

%% \[
%% %% \forall \, prg, \,
%% \text{sound-state } P \text{ } \stt,~ \stt \estep{e} \stt' \Rightarrow \text{sound-state } P \text{ } \stt'~.
%% \]

%% \noindent
%% Then, the (forward) simulation is established assuming the source
%% state is sound, \emph{without} proving the subsequent state is sound.

%% \[
%% \begin{array}{r@{\ \ }l}
%%   \forall (\mssrc, \mstgt) \in R,~ \text{sound-state } P \text{ } \mssrc, \forall e, \mssrc',~&
%%   \mssrc \estep{e} \mssrc' \implies {}\\[1mm]
%%   \exists \mstgt',~&
%%   \mstgt \estep{\tau}^{\raisebox{-1mm}{\scriptsize$\ast$}} \estep{e}\estep{\tau}^{\raisebox{-1mm}{\scriptsize$\ast$}} \mstgt' \land (\mssrc', \mstgt') \in R~.
%% \end{array}
%% \]

%% \todo{better align, backward로 바꾸기}

%%       \myparagraph{Gap}

%% Now, we need to extend such reasoning principle for open
%% simulation. The problem is that, correctness of analyses also are
%% established with essential assumptions on external modules (\eg it
%% cannot access an unleaked memory block), but the memory relation
%% (thus, rely/guarantee condition) an optimization is using does not
%% necessarily imply such. %% Moreover, even if an optimization is
%% already using a powerful
%% memory relation, memory blocks that the optimization is aware of does
%% not necessarily include ones that the analyzer is aware of.

  %%     \ccc{}가 개발될 당시에는 Value Analysis가 없었고, \ccx{}는 ($\MREL{}$에서와 마찬가지로) closed simulation이기 때문에 문제가 없다.
  %%     \ccc{}의 후속연구인 \cascc{}에서도 Value Analysis가 들어간 pass들은 지원하지 않는다.
  %%     그러니까 우리가 이 문제를 처음으로 tackle 하는거다.

  %% \end{enumerate}

%% \hide{
%%   우리가 정의한 $\MREL{}, \SREL{}, \MPRED{}$는 이 3가지 방향에서 open simulation을 parameterize한다.
%%   예를 들어 (i) $\MREL{}$ 인자는 사용할 memory relation(rely/guarantee)을 나타내며, 그 인자로 만들어진 open simulation이 RUSC를 만족하기 위해 필요한 minimal한 성질들 또한 quantify 한다.
%%   \cc{}의 memory relation들은 이 interface를 만족시키고, 따라서 우리는 \cc{}의 simulation relation과 compatible한 (그대로 가져다 쓸 수 있는) open simulation instance를 얻는다.
%%   (ii) $\SREL{}$ interface는 symbol relation을 나타내는데, \cc{}를 위해서는 두가지로 instantiate 시켰다 --- identity와 drop.
%%   \eg{} identitiy interface를 쓸 경우, 링킹에 참여하는 각 module pair들은 symbol을 건드리지 않았다고 (guarantee) 해줘야 하고, 그 결과로 identical한 global environment mapping을 얻는다 (rely).
%%   %% drop interface를 쓸 경우, symbol을 뺄 수 있지만 (Unusedglob) 더이상 identical한 mapping은 얻지 못한다.
%%   $\MREL{}$의 rely/guarantee는 실행 중에 call/return때마다 계속 일어나지만, $\SREL{}$의 rely/guarantee는 loading할 때 한번 일어난다.
%%   (iii) $\MPRED{}$는 각각 분석을 위한 별도의 rely/guarantee를 나타낸다. 여기서의 rely/guarantee는 twofold다.
%%   첫째로, sound\_state가 \emph{open preservation}이라는 것을 만족함을 증명하면 (guarantee), \Cref{fig:simstates}에서 sound\_state를 가정하고 (rely) simulation 할 수 있다.
%%   \cc{} whole program proof에서 그랬던 것처럼 두 증명은 완전히 분리되어있다.
%%   두번째로, open preservation 정의 자체에서 rely/guarantee가 있는데, 이는 기본적으로 $\MREL{}$의 rely/guarantee의 unary version이라고 보면 된다.
%%   $\MREL{}$의 rely/guarantee가 (최적화의) context에 대한 essential assumption을 인코딩했듯이, $\MPRED{}$의 rely/guarantee는 (분석의) context에 대한 essential assumption을 인코딩한다.
%%   \\
%%   \\
%%   %% \begin{figure}[t!]
\footnotesize

\vspace*{-20mm}
\begin{minipage}[t][0.98\textheight]{\linewidth}
\begin{minipage}{\linewidth}
\mytitle{sim:states}\\
$
  \begin{stackTL}
  match\_states \in \textrm{open\_sim}(\msem_\src, \msem_\tgt, sound\_state) \defeq \\
  \forall \mrel, \forall ((\memsrc, \stsrc), (\memtgt, \sttgt)) \in match\_states(\mrel), \cdashbox{myblue}{($\exists \mpred,~ (\memsrc, \stsrc) \in sound\_state(\mpred)$)} \implies \\
  %% \mrel, 
  {\begin{array}{l@{\;}l@{\;}l}
  & \caselabel{STEP} & \msem_{\src}.\texttt{at\_external}(\memsrc, \stsrc) = \none \land \msem_{\src}.\texttt{halted}(\memsrc, \stsrc) = \none \; \land \\
  & & \forall e, \memsrc', \stsrc',~ (\memsrc, \stsrc) \estep{e} (\memsrc',\stsrc') \implies \\[.5mm]
  & & \exists \memtgt',\sttgt',~ (\memtgt,\sttgt) \estep{\tau}^{\raisebox{-1mm}{\scriptsize$\ast$}} \estep{e}\estep{\tau}^{\raisebox{-1mm}{\scriptsize$\ast$}} (\memtgt',\sttgt') \land{}\\[.5mm]
  & & \cfbox{myred}{$\exists \mrel' \sqsupseteq \mrel$},~ ((\memsrc',\stsrc'), (\memtgt',\sttgt')) \in match\_states(\mrel') \\[1.2mm]
  %% \lor & \caselabel{CALL} & \cfbox{myred}{$\exists \args_\src, \args_\tgt, \mrel{}' \sqsupseteq_\weak \mrel{},~ \args_\src \succsim_{\mrel{}'} \args_\tgt$} \land \msem_{\src}.\texttt{at\_external}(\memsrc, \stsrc) = \some{\args_\src} \land \msem_{\tgt}.\texttt{at\_external}(\memtgt, \sttgt) = \some{\args_\tgt} \; \land \; \\
  \lor & \caselabel{CALL} & \cfbox{myred}{$\exists \mrel{}' \sqsupseteq_\weak \mrel{}$},~ \exists \args_\src, \args_\tgt,~ \cfbox{myred}{$\args_\src \succsim_{\mrel{}'} \args_\tgt$} \land \\
  & & \msem_{\src}.\texttt{at\_external}(\memsrc, \stsrc) = \some{\args_\src} \land \msem_{\tgt}.\texttt{at\_external}(\memtgt, \sttgt) = \some{\args_\tgt} \; \land \; \\
  & & \cdashbox{myred}{$\forall \mrel{}'' \sqsupseteq \mrel{}'$}, \forall \retv_\src, \retv_\tgt, \cdashbox{myred}{$\retv_\src \succsim_{\mrel{}''} \retv_\tgt$} \implies \\
  & & \forall \memsrc', \stsrc',~ \msem_{\src}.\texttt{after\_external}(\stsrc, \retv_\src) = \some{(\memsrc', \stsrc')} \implies \\[.5mm]
  & & \exists \memtgt', \sttgt',~ \msem_{\tgt}.\texttt{after\_external}(\sttgt, \retv_\tgt) = \some{(\memtgt', \sttgt')} \; \land \\
  %% & & \cfbox{myred}{$\exists \mrel''', \mrel''' \sqsupseteq \mrel \land \mrel''' \sqsupseteq_\weak \mrel'' \land (\memsrc',\memtgt') \in \mathrm{mrel}(\mrel''')$} \land \\
  & & \cfbox{myred}{$\exists \mrel''' \sqsupseteq_\weak \mrel'',~ \mrel''' \sqsupseteq \mrel$} \land ((\memsrc',\stsrc'), (\memtgt',\sttgt')) \in match\_states(\mrel''') \\[1.8mm]
  \lor & \caselabel{RET} & \cfbox{myred}{$\exists \mrel{}' \sqsupseteq \mrel{}$}, \exists \retv_\src, \retv_\tgt, \cfbox{myred}{$\retv_\src \succsim_{\mrel{}'} \retv_\tgt$} \land \\
       & & \msem_{\src}.\texttt{halted}(\memsrc, \stsrc) = \some{\retv_\src} \land \msem_{\tgt}.\texttt{halted}(\memtgt, \sttgt) = \some{\retv_\tgt} \\
  \end{array}}\\
  \end{stackTL}
$
%% \youngju{First line of (CALL) case is overflowed, we need to reduce text size somehow. I suggest to rollback $\msem$ into S}
\end{minipage}
\vspace{1mm}



\begin{minipage}{1\linewidth}
\mytitle{sim:modsem}\\
$
  \begin{stackTL}
  \msem_{\src} \succsim_{\srel{}, sound\_state} \msem_{\tgt} ~\defeq~ \exists match\_states \in \textrm{open\_sim}(\msem_{src}, \msem_{tgt}, sound\_state),\\
  \begin{array}{@{\quad}l@{\;}l@{\;}l}
  & \caselabel{INIT}
  & \forall \mrel{} \in \MREL{}.\texttt{t},~ \forall \args_\src, \args_\tgt,~ \cdashbox{myred}{$\args_\src \succsim_{\mrel{}} \args_\tgt$} \implies {} \\
  && \args_\src.\texttt{f} \in \textrm{ftns}(\msem_\src.\texttt{senv}) \land \args_\tgt.\texttt{f} \in \textrm{ftns}(\msem_\tgt.\texttt{senv})
    \implies \\
  && \cdashbox{green}{$(\msem_{src}.\texttt{senv}, \msem_{tgt}.\texttt{senv}) \in \simskenv(\srel, \mrel)$} \implies \forall (\memsrc, \stsrc) \in \msem_{\src}.\texttt{init\_core}(\args_\src),\\
  && \exists (\memtgt, \sttgt) \in \msem_{\tgt}.\texttt{init\_core}(\args_\tgt), \\
  && \cfbox{myred}{$\exists \mrel{}' \sqsupseteq \mrel{}$},~ ((\memsrc, \stsrc), (\memtgt, \sttgt)) \in match\_states(\mrel') \\
  %% & ((\memsrc, \stsrc), (\memtgt, \sttgt)) \in \textrm{match\_states}(, \mrel{}) \\
  \end{array}
  \end{stackTL}
$
\end{minipage}
\vspace{1mm}




\begin{minipage}{0.70\linewidth}
\mytitle{sim:mod}\\
$
  \begin{stackTL}
  \module_{\src} \succsim \module_{\tgt} ~\defeq~ \exists \srel{} \in \SREL{}.\texttt{t},~
  \exists sound\_state: \MPRED{}.\texttt{t} \rightarrow \powset{\Memory \times \module_\src.\texttt{state}}, \\[1mm]
  \begin{array}{l@{\;}l@{\;}l}
  %% \qquad & ~ \span \\
  & \caselabel{1} & \cfbox{green}{$(\module_{\src}.\texttt{scode}, \module_{\tgt}.\texttt{scode}) \in \simsk(\srel)$} \vspace{0.5mm}\\
  \land & \caselabel{2} & \cfbox{myblue}{$\forall \skenv_{\src},~ sound\_state \in \text{open\_prsv}(\module_{\src}.\texttt{sem} \; \skenv_{\src})$} \vspace{0.5mm}\\
  \land & \caselabel{3} & \forall \srel' \sqsupseteq \srel,~ \forall \mrel,~ \cdashbox{green}{$\forall (\skenv_\src, \skenv_\tgt) \in \simskenv(\srel', \mrel)$},~ \\
        &               & \module_{\src}.\texttt{sem}\,(\skenv_{\src}) \succsim_{\srel{}, sound\_state} \module_{\tgt}.\texttt{sem}\,(\skenv_{\tgt}) \\
  \end{array}
  \end{stackTL}
$
\end{minipage}%
\begin{minipage}{0.30\linewidth}
\mbox{}\\[14mm]
\mytitle{sim:prog}\\
$
  \begin{stackTL}
  \Prg_{\src} \succsim \Prg_{\tgt} \defeq \\[1mm]
  \quad \forall i \in \mathbb{N},~ \Prg_\src[i] \succsim \Prg_\tgt[i] \\
  \end{stackTL}
$
\end{minipage}


\vspace{1mm}
\begin{minipage}{1\linewidth}
\mytitle{preservation}\\
$
\begin{stackTL}
sound\_state \in \textrm{open\_prsv}(\msem_\src) \defeq \\

\begin{array}{l@{\;}l@{\;}l@{}l}

& \caselabel{INIT} & \forall \mpred \in \MPRED.\ty, \cdashbox{blue}{$\forall \args_\src \in \texttt{cpred}(\mpred)$},~
   \cdashbox{blue}{$\msem_\src.\texttt{senv} \in \texttt{sepred}(\mpred)$} \implies \forall (\memsrc, \stsrc) \in \msem_\src.\texttt{init\_core}(\args_\src), \span \\
& & \cfbox{blue}{$\exists \mpred{}' \sqsupseteq \mpred{}$},~ (\memsrc, \stsrc) \in sound\_state(\mpred') \span \\[1mm]

\land & \caselabel{STEP} & \forall \mpred,~ \forall (\memsrc, \stsrc) \in sound\_state(\mpred), \forall e, \memsrc', \stsrc',~ (\memsrc, \stsrc) \estep{e} (\memsrc', \stsrc') \implies \span \\
& & \cfbox{blue}{$\exists \mpred{}' \sqsupseteq \mpred{}$},~ (\memsrc', \stsrc') \in sound\_state(\mpred') \span \\[1mm]

\land & \caselabel{CALL} & \forall \mpred,~ \forall (\memsrc, \stsrc) \in sound\_state(\mpred), \forall \args_\src,~ \msem_\src.\texttt{at\_external}(\memsrc, \stsrc) = \some{\args_\src} \implies \span \\
& & \cfbox{blue}{$\exists \mpred{}' \sqsupseteq_\weak \mpred{}$},~ \cfbox{blue}{$\args_\src \in \texttt{cpred}(\mpred')$} \; \land \; \span \vspace{0.5mm}\\
& & & \cdashbox{blue}{$\forall \mpred'' \sqsupseteq \mpred{}',~ \forall \retv_\src \in \texttt{rpred}(\mpred'')$},~ \forall \memsrc', \stsrc',~
       \msem_\src.\texttt{after\_external}(\stsrc, \retv_\src) = \some{(\memsrc', \stsrc')} \implies \vspace{0.5mm}\\
& & & \cfbox{blue}{$\exists \mpred''' \sqsupseteq_\weak \mpred'',~ \mpred''' \sqsupseteq \mpred \land \exists \memsrc' \in \texttt{mpred}(\mpred''')$} \land (\memsrc', \stsrc') \in sound\_state(\mpred''')\\[1.5mm]

\land & \caselabel{RET} & \forall \mpred,~ \forall (\memsrc, \stsrc) \in sound\_state(\mpred), \forall \retv_\src,~ \msem_\src.\texttt{halted}(\memsrc, \stsrc) = \some{\retv_\src} \implies \span \\
& & \cfbox{blue}{$\exists \mpred{}' \sqsupseteq \mpred{}$},~ \cfbox{blue}{$\retv_\src \in \texttt{rpred}(\mpred')$} \span \\
\end{array}\\
\end{stackTL}
$
%% \youngju{open sim이랑 open prsv랑 function application 괄호 방식 다름}
\end{minipage}
\vspace*{3mm}
\caption{Parameterized Open Simulations}
\label{fig:full-sim}
\end{minipage}
\\
\\
\\
\\
\\
\end{figure}

%% \youngju{\cfbox{myred}{$\mrel''' \sqsupseteq \mrel''$} it appears in guarantee but not in rely. Remove fbox?}
%% \\
%% \youngju{Need tight box: vspace between two ``after\_external'' is too short}
%% \\
%% \youngju{Function application a(b) and (a b) is mixed. -- vrel(a) and (sc\_compat a b).}
%% \\
%% \youngju{$\msem_\src.\skenv$ is not very readable}
%% \\
%% \youngju{sound\_state is too long}
%% \\
%% \youngju{Need to add analysis developer's interface here}

%%   %% 이 parameter들이 overview 에서 이야기한 structural simulation과의 주된 차이이고, 각각의 정확한 정의는 section에서 더 자세히 설명할 것이다.\\
%% }
  %% 이렇게 parameterize 되어있음에도, 우리가 quantify한 minimal한 성질들만 있으면, open simulation은 RUSC의 성질 두가지를 만족한다. \\
  %% (i) horizontal composition: \Cref{fig:simmod}를 보자. program 사이의 relation은 각 module 사이에 pairwise relation으로 되어있다. 따라서 by definition 성립한다. \\
  %% (ii) adequacy: adequacy는 간단히 말하면 rely/guarantee condition이 \emph{아구가 잘 맞는다}는 뜻이다(이라고 볼 수 있다).
  %% 이제 우리가 정의한 $\MREL{}, \SREL{}, \MPRED{}$ 각각의 rely/guarantee 조건이 open simulation 에서 어떻게 드러나고 어떻게 \emph{아구가 잘 맞는지}만 설명할 것이고, 나머지는 (overview에서 다뤘던) standard open simulation과 비슷하기 때문에 설명은 생략한다.
  %% \todo{\Cref{fig:simmod}랑 \Cref{fig:simmodsem} 간단히는 설명? BSTEP이 overview랑 다른데, safe\_modsem은 ..., PRSV는 ..., coinductive (greatest fixpoint)로 정의했고...}
  %% 먼저 $\SREL{}$의 경우, \Cref{fig:simmod}-(1)에서 두 module의 symbol들이 $\SREL{}.\succsim{}$을 만족함을 guarantee 한다. 그러면, \Cref{fig:simmodsem}에서 내가 불렸을 때 global environment가 $\SREL{}.\succsim{}$에 있음을 rely 할 수 있다.
  %% $\MPRED{}$의 경우, \Cref{fig:simmodsem}-(2)에서 내 모듈이 \emph{open preservation}을 만족함을 guarantee 했다면, \Cref{fig:simstates}에서 내가 만나는 모든 source state임을 이야기할 수 있다.
  %% $\MREL{}$의 경우, 대충 앞이랑 비슷하다.

%%   \begin{enumerate}
%%   \item $\MREL{}$ 정의 및 설명 \\
%%     %% \input{fig-mrel}\\
%%     4가지로 instantiate: \cc{} 3개, 우리 demo를 위해서 1개 (injection with invariant)\\
%%     injection을 예로 들어 figure 설명. \todo{적절히 abstraction 필요}\\
%%     t는 src/tgt 메모리와 injection mapping (여기서 처음 나오나? 설명 필요한지?), \emph{부모 메모리} 그리고 src/tgt이 실제로 injection 관계에 있다는 Prop까지 dependent type으로 있다고 보면 된다. (구현은 wf으로 빼놓음)
%%     부모 메모리는 (i) 부모의 private 영역을 명시하기 위해서 (ii) 부모 영역에 함부로 injection 연결하지 않기 위해서 필요하다.
%%     \\
%%     mrel이 vrel까지 같이 이야기하는 이유 설명?: memory에서 load하면 value니까
%%   \item $\MPRED{}$ 정의 및 설명 \\
%%     %% \input{fig-prsv}
%%     %% \input{fig-mpred}
%%     \mbox{}\\
%%     2가지로 instantiate: top, unreach(value analysis)\\
%%     value analysis(unreach)을 예로 들어 설명 --- \Cref{fig:mpred}는 그냥 \Cref{fig-mrel}의 unary version이다. main dish는 \Cref{fig:prsv}인데, 이것도 거의 \Cref{fig:simmodsem}-(3)의 unary version이다.
%%     한가지 차이는 opensim은 coinductive하게 정의되어있지만 $sound\_state$는 그렇지 않다는 것이다.
%%     만약 coinductive하게 정의했다면 strong coinduction을 써서 $sound\_state$가 중간 state를 뛰어넘고 듬성듬성 성립해도 되는데, 그건 우리가 원하는 것이 아니다.
%%     (우리는 모든 state가 $sound\_state$가 성립하기를 원한다.)
%%     또 한가지 clarify하자면 \Cref{fig:simmodsem}-(SIM) 3번째 줄을 보면 $\exists su$로 되어있는데, 그냥 존재만 해도 되는건지 의문이 들 수 있다.
%%     이 $\MPRED{}.\texttt{t}$는 open\_preservation 에서만 관심있는 데이터고, 최적화 증명에서는 $su$ 값에 관심이 없다.
%%     그래서 $\exists su$로 충분하다.
%%     \\
%%     우리 framework의 포인트는, closed simulation에서와 마찬가지로, preservation 증명과 simulation 증명이 \emph{완전히} 독립적이라는 것이다.
%%     구체적으로, (i) 분석을 한번 증명하면 n개의 pass에서 공짜로 쓸 수 있고 (ii) 한 pass에서 분석 n개를 쓸 수도 있다.
%%     (i)는 정의상 당연하므로 생략하겠다. (ii)를 위해서 $sidx$가 있는데, 각각이 open preservation을 만족하면 ... 잘 된다.
%%     \youngju{사실 technical하게는 $sidx$들 중 하나만 full open preservation (rely/guarantee 둘다) 보이고, 나머지는 rely만 해도 되도록 구현되어있다. 그런데 이걸 활용해서 재밌는걸 한게 없음.
%%       다만 아래에서 reach-closed 이야기할 때, 걔네는 reach-closedness를 언어에 대해 한번만 증명하면 되는데 우리는 분석마다 매번 증명해야되는 것처럼 보일 수도 있음.}
%%     \\
%%     사실 우리가 instantiate한게 unreach 뿐인데, 이건 \ccc{}의 reach-closed semantics 이야기랑 흡사하지만 차이가 있다: 우리의 방식은 static 변수와 관련된 분석으로 확장이 가능하지만 reach-closedness로는 어떻게 확장할지 unclear하다
%%     static 변수와 관련된 논증은 \Cref{sec:results}-program verification에서 보여줄 것이고, 그걸 unary로 확장하는건 그냥 하면 되는 일이다. \youngju{forward reference 하면서 이렇게 얘기하면 좀 약해보이기는 하는 듯}
%%     static 변수와 관련된 분석은 충분히 있음직하지만 우연히 \emph{\cc{}에 없었}던 것 뿐이고 그래서 우리도 instantiate를 안시킨 것이다. %(그래서 우리도 가급적 개발을 lightweight하게 하기 위해 $\mathcal{SP}$ 같은걸 안만든 것이고)
%%     우리 framework의 flexibility를 보여주기 위해 실제로 간단한 static variable 분석기를 구현 중이다.
%%     Also (i) we give verification condition/technique while RC is just a property on semantics (ii) we can naturally support ``global'' analysis.
%%     구체적으로, RC는 ``guarantee'' 만 체크하는거고 ``rely'' 하는게 없다.
%%     %% reach-closed랑 비교, static 변수 언급 --- 우리는 $\SREL{}$ 똑같이 따라해서 $\mathcal{SP}$ 같은거 만들면 충분히 가능하다. 안한 이유는 단순히 \emph{\cc{}에 그런 분석이 없기 때문}이다.
%%     %% gcc/clang은 static 변수에 대한 분석을 당연히 하고, \todo{remove?: 우리의 flexibility를 보여주기 위해 (demo) 현재 구현 중이다}.
%%     \\
%%   \end{enumerate}
%% \end{enumerate}
}

%% \note{overview에서는 verification - semantics 순서지만 여기는 반대로 가는게 좋을 듯.
%%   semantics가 정의되어있어야 simulation 정의 가능}

%% \[ X \: Y \: Z \: W \]
%% \[ X \; Y \; Z \; W \]
%% \[ X \ Y \ Z \ W \]
%% \[ X\text{ }Y\text{ }Z\text{ }W \]
%% \mbox{}\\
     %% ;


%% \begin{figure}
%%   \small
%%   \begin{flushleft}
%%     \fbox{\checker{\insrc{\Prg}}{\intgt{\Prg}}}
%%   \end{flushleft}
%%   \resizebox{\textwidth}{!}{
%%     \vspace*{-10pt}
%%     \begin{tabular}{@{}l@{}}
%%       %% \hspace*{-20pt}
%%       \infer
%%       {\checker{\insrc{\Prg}}{\intgt{\Prg}}}
%%       {\deduce{\forall F\!,\!B,\!B'\!.\, \hoare{\Psi[F].\alpha[B,\!-1]}{\insrc{\Prg}[F].\phi[B,\!B']\!}{\!\intgt{\Prg}[F].\phi[B,\!B']}{\Psi[F].\alpha[B'\!,\!0]}}
%%       {\deduce{\forall F\!,\!B,\!i.\, \hoare{\Psi[F].\alpha[B,\!i]}{\insrc{\Prg}[F].\zeta[B,\!i]}{\intgt{\Prg}[F].\zeta[B,\!i]}{\Psi[F].\alpha[B,\!i\!+\!1]}}
%%       {\deduce{\textrm{CheckCFG}(\insrc{\Prg}, \intgt{\Prg}) \quad \quad \forall F \in \insrc{\Prg}.~\textrm{CheckInit}(\Psi[F].\alpha[\textrm{Entry}(F),0])}
%%       {\textsc{(Sim)}\phantom{}}}}}
%%     \end{tabular}
%%   }

%%   \vspace*{-4pt}
%%   \begin{flushleft}
%%     \fbox{\hoare{P}{\insrc{I}}{\intgt{I}}{Q}}
%%   \end{flushleft}
%%   \vspace*{-11pt}
%%   \begin{tabular}{@{\qquad\qquad\qquad}ll}
%%     \infer
%%     {\hoare{P}{\insrc{I}}{\intgt{I}}{Q}}
%%     {\deduce{Q = \textrm{CalcPostAssn}(P, \insrc{I}, \intgt{I})}
%%     {\deduce{\textrm{CheckEquivBeh}(P, \insrc{I}, \intgt{I})}
%%     {\textsc{(PostAssn)}\phantom{}}}}
%%     &
%%       \infer
%%       {\hoare{P}{\insrc{I}}{\intgt{I}}{Q'}}
%%       {\deduce{\viewshift{Q}{Q'}}
%%       {\deduce{\hoare{P}{\insrc{I}}{\intgt{I}}{Q}}
%%       {\textsc{(Consequence)}\phantom{aaa}}}}
%%   \end{tabular}

%%   \begin{flushleft}
%%     \fbox{\viewshift{Q}{Q'}}
%%   \end{flushleft}
%%   \vspace*{-22pt}
%%   \resizebox{\columnwidth}{!}{
%%     \begin{tabular}{@{~~\qquad\qquad}lll@{}}
%%       \infer
%%       {\viewshift{Q}{Q''}}
%%       {\deduce{\viewshift{Q'}{Q''}}
%%       {\deduce{\viewshift{Q}{Q'}}
%%       {\deduce{\textsc{(Trans)}\phantom{aa}}{\phantom{a}}}}}
%%       &
%%       \infer
%%       {\viewshift{Q}{Q'}}
%%       {\deduce{Q' = \textrm{ApplyInf}(rule, Q)}
%%       {\deduce{rule \in \textrm{CustomRules}}
%%       {\deduce{\textsc{(ApplyInf)}\phantom{aaaaaaaaaaa}}{\phantom{a}}}}}
%%       &
%%       \infer
%%       {\viewshift{Q}{Q'}}
%%       {\deduce{\textrm{CheckIncl}(Q, Q')}
%%       {\deduce{\phantom{R}}
%%       {\deduce{\textsc{(Incl)}\phantom{aaaaaaaaaaa}}{\phantom{a}}}}}
%%     \end{tabular}
%%     }
%% \vspace*{2pt}
%%   \hrule
%%   \caption{Proof Rules of \erhl{}}
%%   \label{fig:proofchecker:rule}
%% \end{figure}


%% \input{fig-inj}

%%% Local Variables:
%%% mode: latex
%%% TeX-master: "main"
%%% End:
