\section{Module-Local Invariants and Specification Modules}
\label{sec:overview-modulelocal}

In \Cref{sec:overview-verification} we presented how to achieve compositional compiler correctness
%% ---namely adequacy, horizontal compositionality, and vertical compositionality---
in our framework.  In this section we present what our framework additionally offers about compiler and
program verification: verifying more advanced compiler optimizations with module-local
invariants (\Cref{sec:overview-modulelocal:compiler}) and verifying program modules against their
mathematical specification modules (\Cref{sec:overview-modulelocal:program} and \Cref{sec:overview-modulelocal:utod}).
%
%% As a small case study,
%% we also verify \code{utod}, a handwritten assembly function casting long to double, whose correctness
%% against its specification is axiomatized in CompCert but not any more in \ccm{}
%% (\Cref{sec:overview-modulelocal:builtin}).
%
\revision{To the best of our knowledge, our framework is the
first, in the context of \cc{}, that is capable of verifying
the \emph{mutually recursive} example presented in \Cref{sec:overview-modulelocal:program}.}

\subsection{Advanced Optimizations with Module-Local Invariants}
\label{sec:overview-modulelocal:compiler}

\begin{figure}[t]
\begin{Verbatim}
static int x = 0;       static int x = 0;                   |  static int y = 0;
int f() {               int f() {              int f() {    |  void g() {
  g();           [CP]     g();          [UG]     g();       |    if (y == 0) {
  x = 1;        ----->    x = 1;       ----->               |      y = 1; f();
  return x;               return 1;              return 1;  |    }
}                       }                      }            |  }
\end{Verbatim}  
\caption{An example of \texttt{Unreadglob} optimization}
\label{fig:overview-modulelocal:compiler}
\end{figure}

We developed a new optimization \texttt{Unreadglob} eliminating all
unread static variables and instructions writing to them.
\Cref{fig:overview-modulelocal:compiler} shows an example
optimization, where $(i)$ the first program is optimized to the second
one by constant propagation~(CP) replacing \texttt{return x} by
\texttt{return 1}; and $(ii)$ the second one is optimized to the third
one by \texttt{Unreadglob}~(UG) eliminating the unread static variable
\texttt{x} and the command \texttt{x = 1}.  It is important to note
that across the function call \texttt{g()}, the static variable
\texttt{x} may be updated from \texttt{0} to \texttt{1} because the
function \texttt{g} can indirectly update it by calling \texttt{f} as shown in
the fourth program in \Cref{fig:overview-modulelocal:compiler}.

\revision{In verification of the optimization \texttt{UG} above,
we have to use memory injections $w$ with module-local invariants
introduced in \Cref{sec:overview-verification:injection:static}.
The reason is that the static variable \texttt{x} in the source cannot reside
$(i)$ in the injection map $w.\iota$ since \texttt{x} does not exist in the target; or
$(ii)$ in the source private $w.m^\weak_\src$ since \texttt{x} can be modified
during the external call \texttt{g()}.
To verify \texttt{UG} above, we can impose the trivial invariant
\texttt{Top} on the eliminated static variable \texttt{x}, meaning
that \texttt{x} can be modified arbitrarily, which is sufficient
because \texttt{x} is unread.}

Note that \ccx{} may be
able to verify \texttt{Unreadglob} using memory injections because it
assumes no mutual dependency among modules, so that no static
variables can be accessed via external function calls, unlike the above
example with mutual recursion.

%% one has to assume that the static variable \texttt{x} resides in a private area
%% of the memory since it does not exist in the optimized code.
%% However, since the memory
%% injection of \cc{} requires that the private area of the memory should
%% be unchanged across any function call, it is contradictory to the
%% above optimization \texttt{UG}, where the private variable \texttt{x}
%% is updated across the call \texttt{g()}. Therefore, one cannot verify
%% \texttt{Unreadglob} using memory injection.

%% In order to verify \texttt{Unreadglob}, we define a new memory
%% relation, memory injection with module-local invariant $P$. The
%% relation is basically the same as memory injection except that instead
%% of requiring the private area to be unchanged across function calls,
%% we allow it to be modified as long as it satisfies the invariant $P$.
%% For example, to verify \texttt{Unreadglob}, we can use the trivial
%% invariant \texttt{Top} meaning that \texttt{x} can be modified
%% arbitrarily, which is sufficient because \texttt{x} is unread.

%% \youngju{not all, just injection}
%% All the verification techniques of \cc{} (and \ccc{}) impose the following strict requirement on
%% optimizations: $(i)$ in both the source and the target, memory is logically split into \emph{public}
%% and \emph{private} parts; $(ii)$ the public parts should be the same in the source and the target,
%% but the private parts may differ; $(iii)$ the private parts should not have been ``leaked'' in the
%% public parts; and $(iv)$ the private parts should not be changed during a (possibly unknown)
%% function call.  All CompCert optimizations---as well as most of the standard compiler
%% optimizations---indeed satisfy the above requirement.

%% However, some advanced optimizations do not satisfy the requirement, and thus are beyond the reach
%% of \cc{}'s verification techniques.  \Cref{fig:overview-modulelocal:compiler} presents a realistic
%% example of such optimizations.  In the left module, the function \code{f()} initially writes to the
%% static global variable \code{a} and then immediately reads from it.  The first transformation
%% performs constant propagation that replaces the read value with constant $1$.  Now \code{a} becomes
%% unread at all.  The second transformation performs \code{Unreadglob}, which removes the unread
%% static global variable \code{a} and all writes to it.  In the verification of the second
%% transformation, \code{a} should be private because it is removed in the target, but as opposed to
%% the requirement, the value of \code{a} in the source may be changed during the call to \code{g()},
%% which invokes \code{f()} that writes $1$ to \code{a}.

%% To verify \code{Unreadglob} and other optimizations that change private memory, we introduce
%% \emph{module-local invariants}: instead of requiring that private memory are not changed during a
%% function call, we allow private memory to change as far as it maintains the specified invariant of
%% the enclosing module.  This idea is embodied in our new memory relation, memory injection with
%% module-local invariants (on private memory).  As the module-local invariant in the verification of
%% \code{Unreadglob}, we require nothing on the value of the unread global variables (\eg{} \code{a})
%% because, after all, they are unread.

%% It is interesting to note that \code{Unreadglob} is provable in \ccx{}, but not because it supports
%% module-local invariants.  The reason is it disallows mutual recursion, thereby preventing the
%% complicated interaction among multiple modules as in \Cref{fig:overview-modulelocal:compiler}.

\subsection{Verification against Specification Modules}
\label{sec:overview-modulelocal:program}

\begin{figure}[t]
\fbox{\begin{minipage}{1.15pc}\mbox{}\\[27.25mm]$\texttt{a.c}$\\[23.35mm]\mbox{}\end{minipage}}
\hspace*{-1.9mm}
\begin{minipage}{0.423\textwidth}
\begin{Verbatim}[frame=single]

static int memoized1[1000] = {0};
int f(int i)   {
  int sum;
  if (i == 0) return 0;
  sum = memoized1[i];
  if (sum == 0) {
    sum = g(i-1) + i;
    memoized1[i] = sum;
  }
  return sum;  
}


\end{Verbatim}
\end{minipage}
\hspace*{-2.0mm}
$\mbox{}~\mathlarger{\mathlarger{\mathlarger{\mathlarger{\mathlarger{\llink}}}}}~\mbox{}$
\hspace*{-2.2mm}
\fbox{\begin{minipage}{2.05pc}\mbox{}\\[26.21mm]$\texttt{b.asm}$\\[24.41mm]\mbox{}\end{minipage}}
\hspace*{-1.9mm}
\begin{minipage}{0.41\textwidth}
\begin{Verbatim}[frame=single]
// hand-optimized in assembly
static int memoized2[2] = {0,0};
int g(int i) {
  int sum;
  if (i == 0) return 0;
  if (i == memoized2[0]) {
    sum = memoized2[1];
  } else {
    sum = f(i-1) + i;
    memoized2[0] = i;
    memoized2[1] = sum;
  }
  return sum;
}
\end{Verbatim}
\end{minipage}
\\
\mbox{\fbox{\begin{minipage}{6.5pc}\mbox{}\\[2.13mm]$\texttt{a.spec}$ \\$\text{   with} ~ \textbf{X} = \texttt{f}, \textbf{Y} = \texttt{g}$\\[3.5mm]
      $\texttt{b.spec}$\\ $\text{   with} ~ \textbf{X} = \texttt{g}, \textbf{Y} = \texttt{f}$\\[0.33mm]\mbox{}\end{minipage}}
\hspace*{-1.9mm}
\fbox{\begin{minipage}{0.72\textwidth}
    $
  \texttt{States} := \setof{ \texttt{Init}\ i \ | \ 0 \leq i } \uplus \setof{ \texttt{Ecall} \ i\  |\  0 \leq i } \uplus \setof{ \texttt{Ret}\  r\  | \ 0 \leq r } \\
  \texttt{init\_core} := \setof{ (\textbf{X}, [i], \texttt{Init}\  i) \ |\  0 \leq i < 1000 } \\
  \texttt{at\_external} := \setof{ (\texttt{Ecall} \ i, \textbf{Y}, [i-1])\  |\  0 < i < 1000 } \\
  \texttt{after\_external} := \setof{ (\texttt{Ecall} \ i, \textrm{sum}(i-1), \texttt{Ret} \ \textrm{sum}(i))\  |\  0 < i < 1000 } \\
  \texttt{halted} := \setof{ (\texttt{Ret}\  r, r) \ |\  0 \leq r } \\
  \begin{aligned}
    \texttt{step} :=& \setof{ ((\texttt{Init} \ i, m), \tau, (\texttt{Ret}\  \textrm{sum}(i), m))\  |\  0 \leq i < 1000 } \ \cup \\[-1.5mm]
    &\setof{ ((\texttt{Init} \ i, m), \tau, (\texttt{Ecall}\  i, m))\  |\  0 < i < 1000 }\\[-1mm]
  \end{aligned}
  $
\end{minipage}}}
\\
\mbox{\fbox{\begin{minipage}{2.9pc}\mbox{}\\[8.73mm]$\texttt{ab.spec}$\\[6.93mm]\mbox{}\end{minipage}}
\hspace*{-1.9mm}
\fbox{\begin{minipage}{0.60\textwidth}
    $
  \texttt{States} := \setof{ \texttt{Init}\ i \ | \ 0 \leq i } \uplus \setof{ \texttt{Ret}\  r\  | \ 0 \leq r } \\
  \texttt{init\_core} := \setof{ (\texttt{f}, [i], \texttt{Init}\  i) } \ \cup \  \setof{ (\texttt{g}, [i], \texttt{Init}\  i) } \\
  \texttt{at\_external} := \setof{ } \\
  \texttt{after\_external} := \setof{ } \\
  \texttt{halted} := \setof{ (\texttt{Ret}\  r, r) \ |\  0 \leq r } \\
  \begin{aligned}
    \texttt{step} :=& \setof{ ((\texttt{Init} \ i, m), \tau, (\texttt{Ret}\  \textrm{sum}(i), m))\  |\  0 \leq i < 1000 }\\[-1mm]
  \end{aligned}
  $
\end{minipage}}}
\caption{The \code{mutual-sum} example}
\label{fig:modulelocal}
\end{figure}


\Cref{fig:modulelocal} shows a C module, \texttt{a.c}; a handwritten
assembly module, \texttt{b.asm} (presented in C syntax for
readability); their open specification modules, \texttt{a.spec} and
\texttt{b.spec}; and the combined closed specification module
\texttt{ab.spec}.  Both functions \texttt{f} in \texttt{a.c} and
\texttt{g} in \texttt{b.asm} mutually recursively compute the
summation from $0$ up to the given argument integer $i$ (denoted
$\mathrm{sum}(i)$), performing different memoization optimizations.
The function \code{f} memoizes the result of \code{f(i)} in the static
variable \code{memoized1[i]}, which is initialized with zero
representing invalid value.  The function call \code{f(i)} first reads
the memoized value, and returns it if it is valid; otherwise, it
calculates, memoizes, and returns \code{g(i-1)}, expected to be
$\mathrm{sum}(\code{i}-1)$, plus \code{i}.  On the other hand, the
function \code{g} memoizes only the result of the latest call
\code{g(i)} with the index \code{i}, where \code{memoized2[0]} =
\code{i} and \code{memoized2[1]} = \code{g(i)}.  The code of \texttt{g}
is self-explanatory under the assumption that the call \code{f(i-1)}
returns $\mathrm{sum}(\code{i}-1)$.

%% we have $\code{g(memoized[0])} = \code{memoized[1]}$.
%% The function \code{g(i)} is the same with
%% \code{f(i)}, except for memoization scheme, the callee of recursion,
%% and the language.

%% Our framework is flexible on the choice of languages to the degree that a module's (open)
%% specification can be represented as another module written in Coq's Gallina language.  Such
%% specification module facilitates modular verification of multi-language programs, as illustrated in
%% the example presented in \Cref{fig:modulelocal}.

%% Both of the function \code{f(i)} in the module \code{a.c} and \code{g(i)} in \code{b.asm} compute
%% $\textrm{sum}(\code{i})$, which is the summation from 1 to \code{i}, but with memoization and mutual
%% recursion on each other.\footnote{The module \code{b.asm} is written in hand-optimized assembly, but
%%   for presentational purposes we present it in C syntax.}  The module \code{a.c} memoizes the result
%% of \code{f(i)} in \code{memoized[i]}, which is initialized with zero representing invalid value.
%% The function \code{f(i)} first reads the memoized value, and returns it if valid; otherwise, it
%% calculates, memoizes, and then returns the summation of \code{g(i-1)}---which is expected to be
%% $sum(\code{i-1})$---and \code{i}.  On the other hand, \code{b.asm} memoizes only one result of
%% \code{g()}: we have $\code{g(memoized[0])} = \code{memoized[1]}$.  The function \code{g(i)} is the
%% same with \code{f(i)}, except for memoization scheme, the callee of recursion, and the language.


The open specification modules \texttt{a.spec} and \texttt{b.spec} are
the same except that the names of the internal and external functions
are swapped. This is natural because the two functions \code{f} and
\code{g} compute the same summation. The open specification
\texttt{a.spec} is an abstract, nondeterministic, version of the
function \texttt{f} in \texttt{a.c} including all the observable
behaviors of \texttt{f}.  It has three kinds of states,
$\code{Init}~i$, $\code{Ecall}~i$ and $\code{Ret}~r$, representing the
initial state with argument $i$, the call state executing
$\code{g}(i-1)$, and the halt state returning $r$, respectively. Then
\code{init\_core} starts with $\code{Init}~i$ when \texttt{f} is
invoked with argument $i$ if $0 \le i < 1000$, otherwise UB;
\code{at\_external} recognizes $\code{Ecall}~i$ as the state invoking
\texttt{g} with $i-1$; \code{after\_external} transitions
from $\code{Ecall}~i$ to $\code{Ret}~\mathrm{sum}(i)$ only
when the return value from the external call $\texttt{g}(i-1)$
is $\mathrm{sum}(i-1)$, otherwise UB, which
means that this module gives a conditional specification under the
assumption that $\texttt{g}(i)$ returns $\mathrm{sum}(i)$;
\code{halted} recognizes $\code{Ret}~r$ as the halted state returning
$r$; and finally $\code{step}$ transitions from $\code{Init}~i$ to
either $\code{Ret}~\mathrm{sum}(i)$ or $\code{Ecall}~i$
nondeterministically (without updating the memory), where the former
abstracts reading from memoization and the latter recursively
computing the sum. The same applies to \texttt{b.spec}.
Finally, the combined specification \code{ab.spec} does not make any
external function call and simply returns the summation.

Then, we perform our verification as follows.
First, we prove $\texttt{a.spec} \rusc_\rels \texttt{a.c}$
using memory injections with the following invariant:\\
%% \[
\mbox{}\hfill$\forall 0 \le i < 1000,~
\code{memoized1}[i] = 0 \lor \code{memoized1}[i] = \mathrm{sum}(i)~.$\hfill\mbox{}
%% \]
\\
Second, we prove $\texttt{b.spec} \rusc_\rels \texttt{b.asm}$
using memory injections with the following invariant:\\
%% \[
\mbox{}\hfill$
\exists 0 \le i < 1000,~
\code{memoized2}[0] = i \land \code{memoized2}[1] = \mathrm{sum}(i)~.
$\hfill\mbox{}
%% \]
\\
Finally, we prove $\texttt{ab.spec} \rusc_\rels \texttt{a.spec} \llink
\texttt{b.spec}$ using the memory identity.
\revision{Note that $\rels$ is the set containing open simulations with the three memory relations
  used in the above verification
  (\ie memory injections with the two invariants above and the memory identity).}

%% The module \code{ab.spec} represents a specification module for $\code{a.c} \llink \code{b.asm}$.
%% The specification essentially says \code{f(i)} and \code{g(i)} returns $sum(\code{i})$ if
%% $0 \le \code{i} < 1000$; otherwise, the behavior is undefined.  Note that the condition on \code{i}
%% comes from the fact that a bigger \code{i} causes buffer overrun at the access to \code{memoized[i]}
%% in \code{f(i)}.  Concretely, \code{ab.spec} has two kinds of states: \code{Init i} representing the
%% initial state with argument \code{i}, and \code{Ret res} representing the halted state with result
%% \code{res}; the module initializes a core with the initial state \code{Init i} when \code{f(i)} or
%% \code{g(i)} is invoked; an initial state \code{Init i} transitions to a return state \code{Ret
%%   $sum(\code{i})$} if $0 \le \code{i} < 1000$; and the module invokes no external function calls.

%% The verification of \code{a.c} and \code{b.asm} amounts to proving
%% $\beh{\texttt{ab.spec}} \supseteq \beh{\texttt{a.c} \llink \texttt{b.asm}}$, which comes from
%% linking the following verifications:
%% \[
%% \begin{array}{c}
%% \texttt{a.spec} \rusc_\rels \texttt{a.c}\quad
%% \texttt{b.spec} \rusc_\rels \texttt{b.asm}
%% \\
%% \texttt{ab.spec} \rusc_\rels \texttt{a.spec} \llink \texttt{b.spec}
%% \end{array}
%% \]
%% where $\rusc_\rels$ is the RUSC relation for a suitable set, $\rels$, of module relations, and
%% \code{a.spec} and \code{b.spec} are themselves (open) specification modules for \code{a.c} and
%% \code{b.asm}, respectively.  The specification module \code{a.spec} essentially says \code{f(i)},
%% provided that \code{i} is in a valid range, either $(i)$ returns $sum(\code{i})$; or $(ii)$ invokes
%% an external call \code{g(i-1)}, receives $sum(\code{i}-1)$ as a result, and then returns
%% $sum(\code{i})$.  To model the interaction with the external function call to \code{g(i-1)}, the
%% specification module has a state, \code{Ecall i}, that represents the call to \code{g(i-1)}.
%% Crucially, If the call does not return $sum(\code{i-1})$, the behavior is undefined.  The
%% specification module \code{b.spec} is the same with \code{a.spec}, except that \code{f} and \code{g}
%% are switched.

%% Verifications of $\texttt{a.spec} \rusc_\rels \texttt{a.c}$ and
%% $\texttt{b.spec} \rusc_\rels \texttt{b.asm}$ require module-local invariants, because the memoized
%% values in \code{memoized} should be private, as they exist only in the target, but they may be
%% changed during an external function call via mutual recursion.  As the module-local invariant, we
%% require that memoized values, if valid, are indeed correct.  On the other hand, verification of
%% $\texttt{ab.spec} \rusc_\rels \texttt{a.spec} \llink \texttt{b.spec}$ does not require module-local
%% invariants and any other complications from programming language semantics, but require reasoning
%% about multiple modules.

%% Specification module not only facilitates modular verification of multi-language programs, thereby
%% reducing the total verification cost, but also enables verification of handwritten assembly
%% functions whose correctness is axiomatized in \cc{}, reducing its trusted computing base (TCB).  See
%% \Cref{sec:utod-verification} for details.

{\revisioncmd
\subsection{Verification of \texttt{utod}}
\label{sec:overview-modulelocal:utod}

\verb|__compcert_i64_utod| is one of the \cc{}'s internal handwritten
assembly functions, which converts \verb|unsigned long| to
\verb|double| by utilizing architecture-specific instructions like
\verb|cvtsi2sdq|. \cc{} currently axiomatizes the behaviors of such runtime libraries as the following axiom.
\[
\begin{minipage}{\textwidth}
\begin{coqdoccode}
\coqdocnoindent
\coqdockw{Axiom} \coqdoccst{i64\_helpers\_correct} : ... \ensuremath{\land}\coqdoceol
\coqdocindent{0.5em}
(\coqdockw{\ensuremath{\forall}} \coqdocvar{x} \coqdocvar{z}, \coqdocmod{Val}.\coqdoccst{floatoflongu} \coqdocvar{x} = \coqdocconstr{Some} \coqdocvar{z} \ensuremath{\rightarrow}
\coqdoccst{external\_implements} "\_\_compcert\_i64\_utod" \coqdoccst{sig\_l\_f} [\coqdocvar{x}] \coqdocvar{z})
\end{coqdoccode}
\end{minipage}
\]

We demonstrate that such axioms can be essentially removed in \ccm{} by proving the axiom for \verb|__compcert_i64_utod|.
We first turn the axiom for \verb|__compcert_i64_utod| into a specification module
and then establish an open simulation with memory injections between the assembly module containing \verb|__compcert_i64_utod| and the specification module.
}

%%% Local Variables:
%%% mode: latex
%%% TeX-master: "main"
%%% End:
