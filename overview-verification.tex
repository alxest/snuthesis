\chapter{\;\;\;\;Verification Techniques}
\label{sec:overview-verification}

\section{Background}
\label{sec:overview-verification:background}

\section{Problems}
\label{sec:overview-verification:problems}
%% \myparagraph{Problems}
%
%   9051 /   2839 = 3.2 (id)
%  22124 /   8129 = 2.7 (ext)
%  23564 /   9128 = 2.6 (inj)
%
% Total
%  54739 /  20096 = 2.7  (pass)
% 152363 /  92029 = 1.65 (meta)
% 207102 / 112125 = 1.84 (whole)
%
%% Structured simulations of \ccc{} suffer from the problem that
%% verification using them is significantly more costly than that using
%% closed simulations of \cc{}: the Coq scripts for the verification of
%% all passes in \ccc{} is roughly \todo{2.7} times as large as that in
%% the original \cc{} in terms of lines of code~(LOC).
%% \jeehoon{How about reporting significant lines of code (SLOC)?}


%\section{Refinement Under Self-related Contexts (RUSC)}
\section{Our Solution}
\label{sec:overview-verification:solution}

%% \myparagraph{Our Solution at High Level}
%

% BEGIN REVISION



%% Using mixed simulation, we verify the compositional correctness of \cc{} optimizations by
%% performing forward reasoning on deterministic target states, thereby reusing all the simulation proofs in \cc{}
%% and performing backward reasoning on nondeterministic target states which we
%% introduce in \ccm{}.
%% Furthermore, using mixed simulation, we can reduce the trusted computing base
%% of the original \cc{} by removing the assumption that external function calls are deterministic.
%% Mixed simulation is the first embodiment of the idea of exploiting determinism at the granularity of
%% machine states in the context of \cc{}, while the idea itself is first presented in
%% \cite{neis:pilsner}.

%% Specifically, \cc{}'s forward simulation actually requires a notion slightly different from
%% determinism (namely, that the source language is \emph{receptive} and the target language is
%% \emph{determinate}).  Our formalization supports both \cc{}-style forward simulation and the
%% Pilsner's one.

%%% Local Variables:
%%% mode: latex
%%% TeX-master: "main"
%%% End:
