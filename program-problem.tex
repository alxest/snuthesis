\section{Problems}
\label{sec:program:problem}

However, those modern variants of Hoare logic also have few drawbacks.
First, the step-index complicates the underlying model, which makes soundness result (or even the meaning of a Hoare triple) esoteric.
Second, it supports only partial correctness (\ie cannot prove termination).
In fact, if we change the meaning of Hoare triple to guarantee {\it total correctness} (\ie both the postcondition and termination), the call rule becomes unsound.
\footnote{Here is a simple counter example: $\code{f} := \code{call} \; \code{f}$. Here, one can prove arbitrary pre/postcondition on $\code{f}$ by using the call rule, which in turn implies that $\code{f}$ terminates. This is contradiction.}
%% \footnote{\cite{jung:irisjfp} \todo{Iris jfp 도 full functional, contextual refinement 지원한다고 써놓기는 했는데 이게 total weakestpre라서 invariant 못쓸거임}}
Therefore, to prove total correctness, one needs to employ a separate tool.
%% \footnote{There are also some variants of Hoare logic which proves total correctness, but it is too stringent to be realistic. Consider the verification of library code where clients are unknown.
%% Using total Hoare logic here will rule out potential non-terminating clients (e.g., web server).}
Third, its adequacy results hold only when the whole program is verified with the same Hoare logic.
This is restrictive because we often want to verify each module with different tools (such as model checker).
Finally, it cannot verify a program that communicates with the outside world. Think of a simple REPL, a web server, or an operating system; it is even unclear how to write a Hoare triple for such a program.
%% Moreover, consider verification of a web server or an operating system; it is not realistic to enforce clients or user-level applications to be formally verified with the same Hoare logic.
%% \even{step-index complicates underlying model, spec is very hard to understand, etc...}
%% Third, \todo{gradual abstraction} ...
