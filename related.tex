{\newrevisioncmd
\chapter{\;\;\;\;Related Work}
\label{sec:related}

%% Besides \ccc{}~\cite{beringer:isem,stewart:ccc} and \ccx{}~\cite{gu:dscal,wang:saccx}, there are two
%% other lines of prior work on compositional correctness for CompCert.
%% Now we compare \ccm{} with each of the prior work.  The
%% comparison is summarized in \Cref{table:table1}.  \todo{explain each column.}
%% Left three columns describe characteristics of semantics, and right four columns describe that of
%% proof development.
%% \input{table1}


%% >  extending the comparison between RUSC and contextual refinement: is
%% >  it possible to formulate a variant of full abstraction w.r.t. the set
%% >  \cal{R} and show that it is satisfied by RUSC? This would relate to
%% >  Patterson & Ahmed's recent discussion (ICFP'19) of full-abstraction,
%% >  and the need to complement it by notions that take low-level
%% >  equivalences into consideration.

%% The idea of applying RUSC to the full abstraction setting sounds
%% interesting. We did not have a chance to think about how RUSC can be
%% used to increase modularity of full abstraction proofs yet.  We will
%% think about it and if we have interesting ideas, we will discuss them
%% in the future work section. Thanks!

%% This approach is still closed in the sense that it does not support contexts written in other languages that are
%% not embedded in the combined language.  In contrast, RUSC is open in the sense that it supports
%% arbitrary self-related contexts written in arbitrary interaction semantics, even including
%% mathematical specifications as shown in \Cref{sec:overview-modulelocal}.



% \youngju{semantics를 correctness를 위한 approach로 본다는게 어색한 것 같습니다.}

% \youngju{(i) compositional correctness for \cc{} (ii) compositional
%   compiler correctness 가 자연스러운 것 같고 그냥 compositional
%   correctness는 어색한 것 같습니다.}

% TODO: Amal Ahmed

% - fix된 language, language 자체를 잘 제약해서 contextual refinement
% - closed (multi-)language w/ contextual refinement <-> open language w/ RUSC
% - 우린 context에 spec도 넣을 수 있음 (구체적으로 programming reasoning)


% \todo{
%   할 이야기
%   - semantics 정의 --> syntactic. 하지만 inline assembly 허용
%   - unsoundness --> toy language 다
%   - proof --> contextual equivalence
%   -
% }


%% \section{Conclusion and Future Work}
%% \label{sec:conclusion}
}

%%% Local Variables:
%%% mode: latex
%%% TeX-master: "main"
%%% End:
