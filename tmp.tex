












To overcome this limitation, two extensions of \cc{}, namely \ccx{}
\cite{gu:dscal,wang:saccx} and Compositional CompCert (shortly, \ccc{}) \cite{beringer:isem,stewart:ccc}, have
been developed. Interestingly, they take different approaches to
\emph{two key challenges}:
\begin{enumerate}
\item how to modularly verify each translation of each
module using a different relational memory invariant (shortly, memory relation) and compose the proofs all
together; and
\item how to deal with illegal interference from
arbitrary (handwritten) assembly modules that can invalidate compiler
translations of C modules (\eg not preserving the
callee-save register values).
\end{enumerate}
CCX, CCC, separation logic, etc.




In this dissertation, we develop a novel lightweight verification technique,
called RUSC (Refinement Under Self-related Contexts), and demonstrate
how RUSC can solve the problem without any restrictions but still with
low verification overhead. For this, we develop CompCertM, a full
extension of the latest version of CompCert supporting multi-language
linking. Also, we demonstrate the power of RUSC as a program
verification technique by modularly verifying interesting programs
consisting of C and handwritten assembly against their mathematical
specifications.





